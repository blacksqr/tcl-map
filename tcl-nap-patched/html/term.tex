%  $Id: term.tex,v 1.1 2006/02/27 03:06:43 dav480 Exp $ 
    % NAP Terminology
      \section{Terminology}

    \subsection{Arrays}

    \begin{description}
      \item[scalar]
      array with 0 dimensions i.e. a simple number, character,
      etc.
      \item[vector]
      array with 1 dimension.
      \item[matrix]
      array with 2 dimensions.
      \item[dimension-size]
      number of values along a dimension.
      \item[shape]
      vector of dimension-sizes of array.
      \item[rank]
      number of dimensions (a.k.a. dimensionality) i.e. shape of
      shape.
      \item[row]
      final (least-significant) dimension
      \item[dimension-name]
      name given to dimension e.g. "latitude".
      \item[coordinate-variable (CV)]
      vector (usually sorted) associated with a dimension of the
      same size. CVs are often used to map an array's dimensions to
      physical dimensions such as length and time, thus locating the
      array elements in physical space and time.
    \end{description}
    \subsection{Special Numeric Values}

    \begin{description}
      \item[missing-value (MV)]
      numeric value of data which is abnormal in some way such as:
	  \begin{itemize}
	    \item not applicable (e.g. land point for ocean data)
	    \item not available (e.g. instrument failure or delay in
	    obtaining data)
	    \item result of some illegal operation such as dividing 0 by 0
	    \item undefined for some other reason 
	  \end{itemize}
      \item[infinity ($\infty$)]
      floating-point value representing a value which is too
          large (or small in the case of $-\infty$) to
          represent. This can result from operations such as dividing
          by 0.
      \item[NaN (not-a-number)]
      floating-point value resulting from an illegal operation
          such as dividing 0 by 0.
    \end{description}
