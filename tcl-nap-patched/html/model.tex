%  $Id: model.tex,v 1.4 2006/02/27 08:10:15 dav480 Exp $ 
    % NAP: Data Models
    \section{Data Models}

  \par A 
  \textit{data model} is a mental model of the nature of some data. It
  answers such questions as the following:
  \subsection{What values can the data take?}
Are they all numeric? Are they
  all integers? Is the set of possible values finite? What are the
  minimum and maximum possible values? Are $\infty$ and 
  \textit{NaN} possible values? Do some values have special meanings,
  such as indicating undefined or missing data?
  \subsection{What is the measurement level?}

  \par Data is often classified as follows according to 
  \begin{i}measurement level\end{i}:
  \begin{center}
    \begin{tabular}{|l||p{40 mm}|p{20 mm}|p{20 mm}|p{30 mm}|}
    \hline 
      \textbf{Level} & 
      \textbf{Description} & 
      \textbf{Valid \mbox{Operations}} & 
      \textbf{Measure of Central\ Tendency} & 
      \textbf{Examples}
    \\
    \hline 
    \hline 
      nominal & 
      Values denote categories which have no order & 
      = $\neq$ & 
      mode & 
      \mbox{zip post-code} \mbox{chemical e.g. CO$_{2}$}
    \\
    \hline 
      ordinal & 
      \mbox{Values ordered} \mbox{Differences meaningless} & 
      = $\neq$ $<$ $\le$ $>$ $\ge$ & 
      median & 
      Richter earthquake scale
    \\
    \hline 
      interval & 
      \mbox{Differences valid} \mbox{Quotients meaningless} & 
      $= \neq < \le > \ge + -$ & 
      arithmetic mean & 
      temperature in $^{\circ}$C
    \\
    \hline 
      ratio & 
      Quotients valid & 
      $= \neq < \le > \ge + - \times \div $ & 
      geometric mean & 
      temperature in $^{\circ}$K 
    \\
  \hline
\end{tabular} \\ \par
  \end{center}

  \subsection{How accurate are the values?}

  \par A 
  \textit{measurement error} is the difference between the 
  \textit{true value} and the 
  \textit{measured value}. Measured values can differ from true values
  due to:
  \begin{itemize}
    \item finite precision of instrument
    \item systematic errors (e.g. inadequately calibrated
    instrument)
    \item random errors (due to finite size of sample)
    \item sampling errors (due to non-randomness of sample)
    \item blunders (e.g. a human misreading an instrument)
  \end{itemize}It is desirable to include error estimates with data.
  \subsection{Are the data located in some space?}

  \par A 
  \textit{time series} consists of values located along the time
  dimension. 
  \textit{Geographic} data is located along spatial dimensions such as
  latitude, longitude and altitude and may also have a time dimension.
  Note that longitude is 
  \textit{cyclic}.
  \par The dimensions of the space can have a measurement level of 
  \textit{nominal}. For example, an accounting spreadsheet might have
  columns corresponding to 
  \textit{charge codes} and rows corresponding to 
  \textit{company divisions}.
  \par Data located in a continuous space can be either 
  \textit{gridded} or 
  \textit{scattered}. Both types are discussed in 
  \href{grid.html}{NAP Grids}.
