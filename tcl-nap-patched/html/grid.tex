%  $Id: grid.tex,v 1.4 2006/02/28 22:27:33 dav480 Exp $ 
    % NAP Grids
    \section{Grids}

  \subsection{
    \label{Dims}Dimensions, Coordinate variables and Mappings
  }
Traditional array-processors, such as APL, are based on a
  data-model which has discrete dimensions whose corresponding
  subscripts take integer values. NAP handles such traditional arrays
  in the traditional manner. However NAP is based on a more general
  data-model which also allows non-integer subscript values. These
  values represent distances along dimensions which often correspond to
  the physical dimensions of spacio-temporal spaces, which are of
  course continuous.
  \par If desired, a dimension can have an associated variable called the
  \textit{coordinate variable (CV)}. This is a vector which defines a
  piecewise-linear mapping from subscript to physical dimension.
  \par CVs are convenient when each array dimension maps to a single
  physical dimension. However, the relationship between array
  dimensions and physical dimensions may be more complex. Consider the
  example of a satellite image with dimensions 
  \textit{line} (row) and 
  \textit{pixel} (column). The physical dimension 
  \textit{latitude} depends on both array dimensions ( 
  \textit{line} and 
  \textit{pixel}). There is a matrix (with the same 
  \textit{line} and 
  \textit{pixel} dimensions as the image) giving the latitude at each
  point. There is another similar longitude matrix. These two matrices
  define piecewise-bilinear mappings from line/pixel space to
  latitude/longitude space. The task of warping the image to
  latitude/longitude space is essentially that of defining mappings
  from latitude/longitude space to line/pixel space i.e. the inverses
  of the given mappings. This can be done using the NAP functions 
{\texttt{invert\_grid} and \texttt{invert\_grid\_no\_trim}},
which are discussed in section ~\ref{function:Grid}.

  \subsection{
    \label{Dif}Difference between Grids and Scattered Data
  }

  \par Data in a continuous space with two or more dimensions can be
  either 
  \textit{gridded} or 
  \textit{scattered}. NAP's data-model (with continuous subscripts,
  coordinate variables, etc.) facilitates the processing of gridded
  data.
  \par Let us consider the case of two dimensions i.e. matrices.
  Two-dimensional 
  \textit{gridded} data is aligned in rows and columns, whereas 
  \textit{scattered} data is not. The following examples are intended
  to illustrate the difference between 
  \textit{scattered} and 
  \textit{gridded} 2D data.
  \subsubsection{Example of Scattered 2D Data (Not a Grid)}

  \par
Note that the data are not aligned in rows and columns.

  \par
  \begin{tabular}{|*{9}{p{6 mm}}|}
      \hline 
       & 
       & 
       & 
      $\bullet$
       & 
       & 
       & 
       & 
       & 
    \\
      $\bullet$
       & 
       & 
       & 
       & 
       & 
      $\bullet$
       & 
       & 
       & 
    \\
       & 
       & 
       & 
       & 
       & 
       & 
       & 
       & 
    \\
       & 
       & 
       & 
       & 
      $\bullet$ 
       & 
       & 
       & 
       & 
    \\
       & 
       & 
       & 
       & 
       & 
      $\bullet$
       & 
       & 
       & 
    \\
       & 
       & 
       & 
       & 
       & 
       & 
       & 
      $\bullet$
       & 
    \\
       & 
       & 
       & 
       & 
       & 
       & 
       & 
       & 
    \\
       & 
       & 
       & 
       & 
       & 
       & 
       & 
       & 
    \\
       & 
       & 
       & 
       & 
       & 
       & 
       & 
       & 
    \\
       & 
      $\bullet$
       & 
       & 
       & 
       & 
       & 
       & 
       & 
    \\
       & 
       & 
       & 
       & 
       & 
       & 
       & 
       & 
      $\bullet$
    \\
       & 
       & 
       & 
       & 
       & 
       & 
       & 
       & 
    \\
       & 
       & 
       & 
       & 
       & 
       & 
       & 
       & 
    \\
       & 
       & 
       & 
       & 
       & 
       & 
      $\bullet$
       & 
       & 
    \\
      \hline 
\end{tabular}

  \subsubsection{
    \label{ex2dgrid}Example of 2D Grid
  }

  \par The following example has the grid in black. The green point is not
  on the grid and has non-integer subscript values (2.1, 1.6). The
  coordinate variables are latitude and longitude.

    \begin{tabular}{|lrl|cccccc|}
      \hline 
         & 
         & 
        column & 
        0 & 
        1 & 
	\textcolor{green}{1.6}
         & 
        2 & 
        3 & 
        4
      \\
         & 
         & 
        longitude & 
        30$^{\circ}$E & 
        40$^{\circ}$E & 
	\textcolor{green}{52$^{\circ}$E}
         & 
        60$^{\circ}$E & 
        65$^{\circ}$E & 
        75$^{\circ}$E
      \\
        row & 
        latitude
         & 
         & 
         & 
         & 
         & 
         & 
         & 
      \\
      \hline 
        0 & 
        30$^{\circ}$N & 
         & 
        $\bullet$ & 
        $\bullet$ & 
         & 
        $\bullet$ & 
        $\bullet$ & 
        $\bullet$
      \\
        1 & 
        25$^{\circ}$N & 
         & 
        $\bullet$ & 
        $\bullet$ & 
         & 
        $\bullet$ & 
        $\bullet$ & 
        $\bullet$ 
      \\
        2 & 
        10$^{\circ}$N & 
         & 
        $\bullet$ & 
        $\bullet$ & 
         & 
        $\bullet$ & 
        $\bullet$ & 
        $\bullet$ 
      \\
	\textcolor{green}{2.1}
         & 
	\textcolor{green}{8$^{\circ}$N}
         & 
         & 
         & 
         & 
\textcolor{green}{$\bullet$}
         & 
         & 
         & 
      \\
        3 & 
        10$^{\circ}$S & 
         & 
        $\bullet$ & 
        $\bullet$ & 
         & 
        $\bullet$ & 
        $\bullet$ & 
        $\bullet$
      \\
      \hline 
\end{tabular}

  \subsection{
    \label{Missing}Missing Data
  }
The NAP data model allows any element of an array to have a
  value which is a 
  \textit{missing-value}. Such elements are considered null or missing
  and are treated specially in operations such as arithmetic. Thus
  adding a missing value to anything produces a missing value.
  \par The following example is similar to that above. However four grid
  points are missing. These are shown in red.

    \begin{tabular}{|lrl|cccccc|}
      \hline 
         & 
         & 
        column & 
        0 & 
        1 & 
	\textcolor{green}{1.6}
         & 
        2 & 
        3 & 
        4
      \\
         & 
         & 
        longitude & 
        30$^{\circ}$E & 
        40$^{\circ}$E & 
	\textcolor{green}{52$^{\circ}$E}
         & 
        60$^{\circ}$E & 
        65$^{\circ}$E & 
        75$^{\circ}$E
      \\
        row & 
        latitude
         & 
         & 
         & 
         & 
         & 
         & 
         & 
      \\
      \hline 
        0 & 
        30$^{\circ}$N & 
         & 
        $\bullet$ & 
        $\bullet$ & 
         & 
        $\bullet$ & 
	\textcolor{red}{$\bullet$} & 
        $\bullet$
      \\
        1 & 
        25$^{\circ}$N & 
         & 
        $\bullet$ & 
        $\bullet$ & 
         & 
        $\bullet$ & 
        $\bullet$ & 
        $\bullet$ 
      \\
        2 & 
        10$^{\circ}$N & 
         & 
	\textcolor{red}{$\bullet$}
         & 
        $\bullet$ & 
         & 
        $\bullet$ & 
        $\bullet$ & 
	\textcolor{red}{$\bullet$}
      \\
	\textcolor{green}{2.1}
         & 
	\textcolor{green}{8$^{\circ}$N}
         & 
         & 
         & 
         & 
\textcolor{green}{$\bullet$}
         & 
         & 
         & 
      \\
        3 & 
        10$^{\circ}$S & 
         & 
        $\bullet$ & 
        $\bullet$ & 
         & 
	\textcolor{red}{$\bullet$}
         & 
        $\bullet$ & 
        $\bullet$
      \\
      \hline 
\end{tabular}

  The missing points are treated as if they did not exist, as shown
  in the following:

    \begin{tabular}{|lrl|cccccc|}
      \hline 
         & 
         & 
        column & 
        0 & 
        1 & 
	\textcolor{green}{1.6}
         & 
        2 & 
        3 & 
        4
      \\
         & 
         & 
        longitude & 
        30$^{\circ}$E & 
        40$^{\circ}$E & 
	\textcolor{green}{52$^{\circ}$E}
         & 
        60$^{\circ}$E & 
        65$^{\circ}$E & 
        75$^{\circ}$E
      \\
        row & 
        latitude
         & 
         & 
         & 
         & 
         & 
         & 
         & 
      \\
      \hline 
        0 & 
        30$^{\circ}$N & 
         & 
        $\bullet$ & 
        $\bullet$ & 
         & 
        $\bullet$ & 
	& 
        $\bullet$
      \\
        1 & 
        25$^{\circ}$N & 
         & 
        $\bullet$ & 
        $\bullet$ & 
         & 
        $\bullet$ & 
        $\bullet$ & 
        $\bullet$ 
      \\
        2 & 
        10$^{\circ}$N & 
         & 
         & 
        $\bullet$ & 
         & 
        $\bullet$ & 
        $\bullet$ & 
      \\
	\textcolor{green}{2.1}
         & 
	\textcolor{green}{8$^{\circ}$N}
         & 
         & 
         & 
         & 
\textcolor{green}{$\bullet$}
         & 
         & 
         & 
      \\
        3 & 
        10$^{\circ}$S & 
         & 
        $\bullet$ & 
        $\bullet$ & 
         & 
         & 
        $\bullet$ & 
        $\bullet$
      \\
      \hline 
\end{tabular}

  \par It would be possible to represent scattered data by a grid with
  many missing points. In the extreme, each scattered point would have
  its own row and its own column. There would be only one non-missing
  point in each row. There would be only one non-missing point in each
  column. Of course this would be very inefficient for a matrix of
  significant size.
  \subsection{
    \label{Scattered}Processing Scattered Data
  }

  \par Function 
{\texttt{scattered2grid} } interpolates scattered data onto a grid.
See section
  \ref{nap-function-lib:scattered2grid}
for details.
