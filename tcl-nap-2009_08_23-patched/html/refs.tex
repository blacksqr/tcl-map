%  $Id: refs.tex,v 1.1 2006/02/28 00:09:32 dav480 Exp $ 
    % References
    \section{References}

  \subsection{
    \label{Introduction}Introduction
  }
The following provides a guide to books, papers and web sites
  relevant to NAP. The opinions are those of the author (Harvey
  Davies).
  \subsection{
    \label{tcl}Tcl/Tk
  }
The following are my three favourite Tcl/Tk web sites:
  \begin{description}
    \item[
      \href{http://wiki.tcl.tk}{wiki.tcl.tk}
    ]
    Wiki-based open community site. Excellent starting point.
    \item[
      \href{http://www.tcl.tk}{www.tcl.tk}
    ]
    Main Tcl Developer Xchange site.
    \item[
      \href{http://www.activestate.com/Products/ActiveTcl/}{www.activestate.com/Products/ActiveTcl/}
    ]
    ActiveState provide ActiveTcl here for downloading.
  \end{description}
  \par Despite its age, I consider John Ousterhout's 1994 
  \emph{Tcl and the Tk Toolkit} to still be the best introduction
  to Tcl/Tk. In fact it really is a classic example of clear readable
  technical writing. Ousterhout was the original developer of Tcl. Tcl
  has developed a lot since 1994 and the new features missing from this
  book are discussed in the 
  \href{http://wiki.tcl.tk/103}{Wiki}.
  \par The only detailed up-to-date book is the much less readable 4th
  edition of 
  \emph{Practical Programming in Tcl and Tk} by Brent Welch and
  Ken Jones with Jeffrey Hobbs. This book has the web site 
  \href{http://www.beedub.com/book}{www.beedub.com/book}.
  \subsection{
    \label{NAP}NAP
  }
The home page for 
  \emph{nap} (a.k.a.
  \emph{tcl-nap}) is at 
  \href{http://tcl-nap.sourceforge.net/index.php}{http://tcl-nap.sourceforge.net/index.php}.
The main documentation of NAP is provided by the
  \href{http://tcl-nap.sourceforge.net/nap_users_guide.pdf}{NAP User's Guide},
which is what you are reading.
  \par The only published paper on NAP is Harvey Davies's talk 
  \emph{The NAP (N-Dimensional Array Processor) Extension to Tcl}
  , which was given at the 9th (2002) Annual Tcl/Tk Conference.
Both the
\href{http://aspn.activestate.com/ASPN/Tcl/TclConferencePapers2002/Tcl2002papers/davies-nap/nap.pdf}{original}
and a
  \href{http://tcl-nap.sourceforge.net/nap_paper2002.pdf}{revised} version are available.
  \par NAP was originally developed as part of the CSIRO CAPS project,
  which has the web page 
  \href{http://www.eoc.csiro.au/cats/caps/}{http://www.eoc.csiro.au/cats/caps/}.

  \subsection{
    \label{array}Array Processing
  }
There are two main array processing professional associations in
  the world. The USA has the ACM 
  \emph{Special Interest Group on the APL and J languages}, which has
  the web page 
  \href{http://www.acm.org/sigapl}{www.acm.org/sigapl}. The
  British Computer Society has a specialist group the 
  \emph{British APL Association}, which publishes the journal 
  \emph{Vector}, which has the web page 
  \href{http://www.vector.org.uk}{www.vector.org.uk}. Both
  associations were originally concerned with only APL but now include
  other array processing languages.
  \par The primary site for the J language is 
  \href{http://www.jsoftware.com}{www.jsoftware.com}.
  \subsection{
    \label{file}Array-oriented File Formats
  }
The primary site for netCDF is 
  \href{http://www.unidata.ucar.edu/packages/netcdf/index.html}{www.unidata.ucar.edu/packages/netcdf/index.html}.
  \\The primary site for HDF is 
  \href{http://hdf.ncsa.uiuc.edu}{hdf.ncsa.uiuc.edu}.
  \subsection{
    \label{Arithmetic}Arithmetic
  }
The classic article on floating-point arithmetic (and the IEEE
  Standard 754 for it) is David Goldberg's 
  \href{http://delivery.acm.org/10.1145/110000/103163/p5-goldberg.pdf?key1=103163$\backslash$\&amp;key2=9533216011$\backslash$\&amp;coll=GUIDE$\backslash$\&amp;dl=ACM$\backslash$\&amp;CFID=36776917$\backslash$\&amp;CFTOKEN=647870}{ \emph{What Every Computer Scientist Should Know About Floating-Point Arithmetic} }, ACM Computing Surveys, 1991. A more recent paper is Kahan's
  \href{http://www.cs.berkeley.edu/~wkahan/ieee754status/ieee754.ps} {\emph{Lecture 
    Notes on the Status of IEEE 754} }.
