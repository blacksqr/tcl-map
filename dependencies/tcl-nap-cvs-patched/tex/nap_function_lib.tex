%  $Id: nap_function_lib.tex,v 1.11 2006/06/06 08:09:42 dav480 Exp $ 
    % Nap Library: nap\_function\_lib.tcl

\section{Miscellaneous Functions}
    \label{nap-function-lib}

\subsection{Introduction}
    \label{nap-function-lib-Introduction}

The following functions are defined in the file 
  \texttt{nap\_function\_lib.tcl}.

\subsection{\texttt{color\_wheel(}$n$\texttt{,} $v$\texttt{,} $b$\texttt{)}}
    \label{nap-function-lib-color-wheel}

Square containing color wheel.
\\
$n$ is number rows and columns.
\\
$v$ is desired \emph{value} level.
\\
$b$ is background colour outside circle.

Example:
\\
\texttt{nap "color\_wheel(100,255,3\#150)"}
\\
This produces a 
\texttt{u8} array with shape 
\texttt{\{3 100 100\}} and values from 0 to 255.

\subsection{\texttt{cpi(}$\mathit{array}$[\texttt{,}$i$[\texttt{,}$j$[\texttt{,}$k$ \ldots ]]]\texttt{)}}
    \label{nap-function-lib-cpi}

Cross-product indexing without the need to specify commas for trailing
dimensions that are to be fully selected.
This is useful when the rank of the array can vary.

Examples:
\begin{verbatim}
% [nap "m = {{1 3 5}{6 8 9}}"]
1 3 5
6 8 9
% [nap "cpi(m, {1 0})"]
6 8 9
1 3 5
% [nap "cpi(m, 1, {2 0})"]
9 6
% [nap "cpi(m, , 1)"]
3 8
% [nap "cpi(m,0,)"]
1 3 5
% [nap "cpi(m,0)"]
1 3 5
% [nap "cpi(m)"]
1 3 5
6 8 9
% [nap "cpi(m,)"]
1 3 5
6 8 9
% [nap "cpi(m,,)"]
1 3 5
6 8 9
\end{verbatim}

\subsection{\texttt{cv(}$main\_nao$[\texttt{,} $\mathit{dim\_number}|\mathit{dim\_name}$]\texttt{)}}
    \label{nap-function-lib-cv} 

This is simply an alias for 
  \texttt{coordinate\_variable}.

\subsection{\texttt{derivative(}$a$[\texttt{,} $\mathit{dim\_number}|\mathit{dim\_name}$]\texttt{)}}
      \label{nap-function-lib-derivative} 

Estimate derivative along specified dimension (default is 0) of array $a$.
The result has the same shape as $a$.

Example (assuming \texttt{vector} has dimension (and coordinate variable) \texttt{time}:
\begin{verbatim}
derivative(vector); # result is derivative with respect to time
\end{verbatim}

Examples (assuming \texttt{matrix} has dimensions \texttt{latitude} and \texttt{longitude}):
  \begin{verbatim}
derivative(matrix, 'latitude'); # result is derivative with respect to latitude
derivative(matrix, 0);          # result is derivative with respect to latitude
derivative(matrix);             # result is derivative with respect to latitude
derivative(matrix,'longitude'); # result is derivative with respect to longitude
derivative(matrix, 1);          # result is derivative with respect to longitude
\end{verbatim}

The derivative is based on the
quadratic through 3 points (provided size of dimension is greater than 2, 
if only 2 then based on straight line). 
These points always include the point corresponding to the result.
For interior points, the other 2 are the closest neighbours on each side.
For boundary points, these are the 2 closest neighbours.

Let $D(x)$ be the derivative of the quadratic through points 
$(x_0,y_0)$,
$(x_1,y_1)$
and
$(x_2,y_2)$.
\[ D(x_1) = a_0 y_0 + a_1 y_1 + a_2 y_2 \]
where the coefficients $a_0$, $a_1$, $a_2$ are defined by:
\begin{eqnarray*}
a_0 & = & \frac{x_1 - x_2}{(x_1 - x_0)(x_2 - x_0)} \\
a_1 & = & \frac{1}{x_1 - x_0} - \frac{1}{x_2 - x_1} \\
a_2 & = & \frac{x_1 - x_0}{(x_2 - x_0)(x_2 - x_1)}
\end{eqnarray*}

\subsection{\texttt{fill\_holes(}$x$\texttt{,} [$\mathit{max\_nloops}$]\texttt{)}}
    \label{nap-function-lib-fill-holes}

    $x$ is array to be filled.
    \\
    $\mathit{max\_nloops}$ is maximum number of iterations.
    \\
    Replace missing values by estimates based on means of neighbours.
    If $max\_nloops$ is not specified then \texttt{fill\_holes}
    continues until there are no missing values.

\subsection{\texttt{fuzzy\_floor(}$x$[\texttt{,} $\mathit{eps}$]\texttt{)}}
    \label{nap-function-lib-fuzzy-floor} 

Like 
  \texttt{floor()} except allow for rounding error.
  \\
  $\mathit{eps}$ is tolerance and defaults to 
  \texttt{1e-9}.
  \par Example:
  \\
  \texttt{\% [nap "fuzzy\_floor(\{4.998 4.9998\},1e-3)"]
  \\4 5}

\subsection{\texttt{fuzzy\_ceil(}$x$[\texttt{,} $\mathit{eps}$]\texttt{)}}
    \label{nap-function-lib-fuzzy-ceil}

Like 
  \texttt{ceil()} except allow for rounding error.
  \\
  $\mathit{eps}$ is tolerance and defaults to 
  \texttt{1e-9}.
  \par Example:
  \\
  \texttt{\% [nap "fuzzy\_ceil(\{5.002 5.0002\},1e-3)"]
  \\6 5}

\subsection{\texttt{gets\_matrix(}$\mathit{filename}$[\texttt{,} $\mathit{n\_header\_lines}$]\texttt{)}}
    \label{nap-function-lib-gets-matrix}

Read text file and return NAO matrix whose rows correspond to
  the lines in the file. Ignore:
\begin{bullets}
    \item first $\mathit{n\_header\_lines}$ (default 0) lines
    \item blank lines
    \item lines whose first non-whitespace character is `\texttt{\#}'
\end{bullets}
Examples:
  \\
  \texttt{nap "m1 = gets\_matrix('matrix1.txt')"
  \\nap "m2 = gets\_matrix('matrix2.txt', 3)"; \# Skip 1st 3 lines}

\subsection{\texttt{head(}$x$[\texttt{,} $n$]\texttt{)}}
    \label{nap-function-lib-head}

If $n \ge 0$ then result is 1st $n$ elements of $x$, cycling if $n > \texttt{nels}(x)$.
\\
$n$ defaults to 1.
\\
If $n<0$ then result is 1st $\texttt{nels}(x)+n$ elements of $x$ i.e. drop $-n$ from end.
\\
Examples:
\begin{verbatim}
% [nap "head({3 1 9 2 7})"]
3
% [nap "head({3 1 9 2 7}, 2)"]
3 1
% [nap "head({3 1 9 2 7}, -2)"]
3 1 9
\end{verbatim}

\subsection{\texttt{hsv2rgb(}$\mathit{hsv}$\texttt{)}}
    \label{nap-function-lib-hsv2rgb}

Convert colour in HSV form to RGB.
  \\
$\mathit{hsv}$ is an array whose leading dimension has size 3.

Layer 0 along this dimension corresponds to \emph{hue} as an angle in
  degrees. Angles of any sign or magnitude are allowed.
  Red~=~0, yellow~=~60, green~=~120,
  cyan~=~180, blue~=~$-120$,
  magenta~=~$-60$.

Layer 1 along this dimension corresponds to \emph{saturation} in range 0.0 to 1.0.

Layer 2 along this dimension corresponds to \emph{value}.
  This has the same range as the RGB values, normally either 0.0 to 1.0
  or 0 to 255. If you are casting the result to an integer and want a
  maximum of 255 then set the maximum to say 255.999. Otherwise you
  will get few if any 255s.

The result has the same shape as the argument ($\mathit{hsv}).

See Foley, vanDam, Feiner and Hughes, 
  \textit{Computer Graphics Principles and Practice}, Second Edition, 1990, ISBN
  0201121107 page 593.

Example:
  \\
  \texttt{\% [nap "hsv2rgb \{180.0 0.5 100.0\}"]
  \\50 100 100}

\subsection{\texttt{isMissing(}$x$\texttt{)}}
    \label{nap-function-lib-isMissing}

1 if $x$ missing, 0 if present.
\\
Example:
\\
\texttt{\% [nap "isMissing \{0 \_ 9\}"]
\\
0 1 0}

\subsection{\texttt{isPresent(}$x$\texttt{)}}
    \label{nap-function-lib-isPresent}

0 if $x$ missing, 1 if present.
\\
Example:
\\
\texttt{\% [nap "isPresent \{0 \_ 9\}"]
\\1 0 1}

\subsection{\texttt{magnify\_interp(}$a$\texttt{,} $\mathit{mag\_factor}$\texttt{)}
\label{nap-function-lib-magnify-interp}}

Magnify each dimension of array 
$a$ by factor defined by the corresponding element of 
$\mathit{mag\_factor}$ if this is a vector. If this is a scalar then
every dimension is magnified by the same factor. The new values are
estimated using multi-linear interpolation.

This function can be used to make images larger or smaller.

Example:
\\
\texttt{\% [nap "magnify\_interp(\{\{1 2 3\}\{4 5 6\}\}, \{1 3\})"]
value
\\1.00000 1.33333 1.66667 2.00000 2.33333 2.66667 3.00000
\\4.00000 4.33333 4.66667 5.00000 5.33333 5.66667 6.00000}

\subsection{\texttt{magnify\_nearest(}$a$\texttt{,} $\mathit{mag\_factor}$\texttt{)}}
    \label{nap-function-lib-magnify-nearest}

This function is similar to 
\texttt{magnify\_interp} except that the new values are defined by
the nearest neighbour rather than interpolation.

Example:
\\
\texttt{\% [nap "magnify\_nearest(\{\{1 2 3\}\{4 5 6\}\}, \{1 3\})"]
value
\\1 1 2 2 2 3 3
\\4 4 5 5 5 6 6}

\subsection{\texttt{mixed\_base(}$x$\texttt{,} $b$\texttt{)}}
    \label{nap-function-lib-mixed-base}

Convert scalar value $x$ to mixed base defined by vector $b$.

Following example converts 87 inches to yards, feet and inches:
  \\
  \texttt{\% [nap "mixed\_base(87, \{3 12\})"]
  \\2 1 3}

\subsection{\texttt{nub(}$x$\texttt{)}}
    \label{nap-function-lib-nub}

Result is vector of distinct values in argument (in same order).

Example:
\begin{verbatim}
% [nap "nub {{5 -9 5}{1 1 5}}"]
5 -9 1
\end{verbatim}

\subsection{\texttt{outer(}$\mathit{dyad}$\texttt{,} $y$[, $x$]\texttt{)}}
    \label{nap-function-lib-outer}

Tensor outer-product.
\\
$\mathit{dyad}$ is name of either
\begin{bullets}
\item function with two arguments
\item binary (dyadic) operator
\end{bullets}
$x$ is vector
\\
$y$ is vector defaulting to $x$

Result is cross-product of $x$ and $y$, applying $\mathit{dyad}$ to each combination of $x$ and $y$.
The coordinate variables of the result are $x$ and $y$. 

Following example produces a multiplication table:
  \begin{verbatim}
% [nap "outer('*', 1 .. 5)"]
 1  2  3  4  5
 2  4  6  8 10
 3  6  9 12 15
 4  8 12 16 20
 5 10 15 20 25
\end{verbatim}

\subsection{\texttt{palette\_interpolate(}$\mathit{from}$\texttt{,} $\mathit{to}$)}
    \label{nap-function-lib-palette-interpolate}

Define a palette by interpolating around the HSV (hue,
  saturation, value) colour wheel with both 
$S$ (saturation) and $V$ (value) set to 1.
The arguments $\mathit{from}$ and 
$\mathit{to}$ are angles in degrees which specify the range of $H$ (hue).
Red is $0$, green is $-240$ and blue is $240$.

\subsection{\texttt{scattered2grid(}$\mathit{xyz}$\texttt{,}
$\mathit{ycv}$\texttt{,} $\mathit{xcv}$\texttt{)}}
    \label{nap-function-lib-scattered2grid}

Produce a matrix grid from scattered $(x,y,z)$ data using triangulation.
Grid points within each triangle are defined by interpolating using a plane through the three
vertices of the triangle.

$\mathit{xyz}$ is an $n \times m$ matrix containing data corresponding to 
$n$ points $(x,y,z)$. The number of columns ($m$) must be at least 3.
Columns 0, 1 and 2 contain $x$, $y$ and $z$ values respectively.
Any further columns are ignored.
 
$\mathit{ycv}$ and $\mathit{xcv}$ specify the coordinate-variables for the grid.

The following example defines a grid from the four points $(2,2,0)$, $(6,4,0)$, $(2,4,4)$
and $(4,5,3)$.
Note that the missing values in the
result correspond to points which are outside of both the triangles
produced by the triangulation. You could eliminate these missing
values by defining values at all four corners of the grid.
\begin{verbatim}
% nap "z = scattered2grid({{2 2 0}{6 4 0}{2 4 4}{4 5 3}}, 2 .. 5, 2 .. 6)"
::NAP::1020-1020
% $z
0.00    _    _    _    _
2.00 0.75 0.00    _    _
4.00 2.50 1.50 0.75 0.00
   _    _ 3.00    _    _
% [nap "z(@2, @2)"]; # Check value at x=2, y=2
0
% [nap "z(@4, @6)"]; # Check value at x=6, y=4
0
% [nap "z(@4, @2)"]; # Check value at x=2, y=4
4
% [nap "z(@5, @4)"]; # Check value at x=4, y=5
3
\end{verbatim}

\subsection{\texttt{scaleAxis(}$\mathit{xstart}$\texttt{,}
$\mathit{xend}$[\texttt{,} $\mathit{nmax}$[\texttt{,} $\mathit{nice}$]]\texttt{)}}
    \label{nap-function-lib-scaleAxis}

Find suitable values for an axis of a graph.
These values have a range \emph{within} the interval from $\mathit{xstart}$ to $\mathit{xend}$.
  \\
  $\mathit{xstart}$: First data value
  \\
  $\mathit{xend}$: Final data value
  \\
  $\mathit{nmax}$: Maximum allowable number of elements in result (Default: 10)
  \\
  $\mathit{nice}$: Allowable increments (Default: \texttt{\{1 2 5\}})
  \\Result is the arithmetic progression which:
  \begin{bullets}
    \item is within interval from $\mathit{xstart}$ to $\mathit{xend}$
    \item has same order (ascending/descending) as $\mathit{xstart} \texttt{//} \mathit{xend}$
    \item has increment equal to element of $\mathit{nice}$ times a power \texttt{(-30 .. 30)} of 10
    \item has at least two elements
    \item has no more than $\mathit{nmax}$ elements if possible
    \item has as many elements as possible.
	(Ties are resolved by choosing earlier element in $\mathit{nice}$.)
  \end{bullets}

Example:
  \\
  \texttt{\% [nap "axis = scaleAxis(-370, 580, 10, \{10 20 25
  50\})"] value
  \\-300 -200 -100 0 100 200 300 400 500}

\subsection{\texttt{scaleAxisSpan(}$\mathit{xstart}$\texttt{,}
$\mathit{xend}$[\texttt{,} $\mathit{nmax}$[\texttt{,} $\mathit{nice}$]]\texttt{)}}
    \label{nap-function-lib-scaleAxisSpan}

Find suitable values for an axis of a graph.
These values have a range which \emph{includes}
the interval from $\mathit{xstart}$ to $\mathit{xend}$.
  \\
  $\mathit{xstart}$: First data value
  \\
  $\mathit{xend}$: Final data value
  \\
  $\mathit{nmax}$: Maximum allowable number of elements in result (Default: 10)
  \\
  $\mathit{nice}$: Allowable increments (Default: \texttt{\{1 2 5\}})
  \\Result is the arithmetic progression which:
  \begin{bullets}
    \item includes the interval from $\mathit{xstart}$ to $\mathit{xend}$
    \item has same order (ascending/descending) as $\mathit{xstart} \texttt{//} \mathit{xend}$
    \item has increment equal to element of $\mathit{nice}$ times a power \texttt{(-30 .. 30)} of 10
    \item has at least two elements
    \item has no more than $\mathit{nmax}$ elements if possible
    \item has as many elements as possible.
	(Ties are resolved by choosing earlier element in 
    $\mathit{nice}$.)
  \end{bullets}

Example:
  \\
  \texttt{\% [nap "axis = scaleAxisSpan(-370, 580, 10, \{10 20 25
  50\})"] value
  \\-400 -200 0 200 400 600}

\subsection{\texttt{range(}$a$\texttt{)}}
    \label{nap-function-lib-range}

Result is 2-element vector containing minimum and maximum of
  array 
  $a$.
  \\Example:
  \\
  \texttt{\% [nap "range \{\{9 -1 -5\}\{2 9 3\}\}"]
  \\-5 9}

\subsection{\texttt{tail(}$x$[\texttt{,} $n$]\texttt{)}}
    \label{nap-function-lib-tail}

If $n \ge 0$ then result is final $n$ elements of $x$, cycling if $n > \texttt{nels}(x)$.
\\
$n$ defaults to 1.
\\
If $n < 0$ then result is final $\texttt{nels}(x)+n$ elements of $x$
i.e. drop $-n$ from start.

Example:
\begin{verbatim}
% [nap "tail({3 1 9 2 7})"]
7
% [nap "tail({3 1 9 2 7}, 2)"]
2 7
% [nap "tail({3 1 9 2 7}, -2)"]
9 2 7
\end{verbatim}

