%  $Id: ooc_meta.tex,v 1.8 2006/10/27 08:05:37 dav480 Exp $ 
    % OOC Methods: Metadata

\section{OOC Methods which return Metadata}
    \label{ooc-meta}

\subsection{Introduction}
    \label{ooc-meta-intro}

The following code defines vectors `\texttt{x}' and `\texttt{y}' for use in the examples below:
  \begin{verbatim}
% nap "x = 0 .. 2 ... 0.5"
::NAP::58-58
% nap "y = x ** 2"
::NAP::61-61
% $y val -format %0.3f
0.000 0.250 1.000 2.250 4.000
\end{verbatim}

\subsection{Method \texttt{coordinate}}
    \label{ooc-meta-coordinate}

$ooc\_name$ \texttt{coordinate}
[$dim\_name$\texttt{|}$dim\_number$]
[$dim\_name$\texttt{|}$dim\_number$]
$\ldots$

This returns the OOC-names of the coordinate variables of selected
  dimensions. If no dimensions are specified then the effect is the
  same as `$ooc\_name$ \texttt{coo} 0 1 2 $\ldots$ $rank-1$'.

Example (continued from above):
  \begin{verbatim}
% $y set coo x
% $y coo
::NAP::58-58
% [$y coo]
0 0.5 1 1.5 2
\end{verbatim}

\subsection{Method \texttt{count}}
    \label{ooc-meta-count}

$ooc\_name$ \texttt{count} [\texttt{-keep}]

This returns the reference count.
  

Example (continued from above):
  \begin{verbatim}
% $x count
2
\end{verbatim}

Note that the reference count is 2 because this NAO is referenced
by both
\begin{bullets}
    \item Tcl variable 
    \texttt{x}
    \item coordinate variable pointer of NAO 
    \texttt{::NAP::61-61}
\end{bullets}

\subsection{Method \texttt{datatype}}
    \label{ooc-meta-datatype}

  $ooc\_name$ \texttt{datatype}

This returns the data-type.
  
Example (continued from above):
  \begin{verbatim}
% $x dat
f64
\end{verbatim}

\subsection{Method \texttt{dimension}}
    \label{ooc-meta-dimension}

$ooc\_name$ \texttt{dimension} [$dim\_number$] [$dim\_number$] $\ldots$

This returns the dimension names.

  `$ooc\_name$ \texttt{di}'
  is equivalent to:
  `$ooc\_name$ \texttt{di} 0 1 2 $\ldots$ $rank-1$'

Example (continued from above):
  \begin{verbatim}
% $y dim
x
\end{verbatim}

\subsection{Method \texttt{format}}
    \label{ooc-meta-format}

  $ooc\_name$ \texttt{format}

This returns the C format for printing the NAO.
  
Example (continued from above):
  \begin{verbatim}
% $y set format %.4f
% $y format
%.4f
% $y value
0.0000 0.2500 1.0000 2.2500 4.0000
\end{verbatim}

\subsection{Method \texttt{header}}
    \label{ooc-meta-header}

$ooc\_name$ \texttt{header} [\texttt{-keep}]

This returns similar information to the following (but using a
  different format):
  \\
  $ooc\_name$ 
  \texttt{ooc}
  \\
  $ooc\_name$ 
  \texttt{datatype}
  \\
  $ooc\_name$ 
  \texttt{missing}
  \\
  $ooc\_name$ 
  \texttt{count}
  \\
  $ooc\_name$ 
  \texttt{unit}
  \\
  $ooc\_name$ 
  \texttt{shape}
  \\
  $ooc\_name$ 
  \texttt{dimension}
  \\
  $ooc\_name$ 
  \texttt{coordinate}

Example (continued from above):
  \begin{verbatim}
% $y header
::NAP::61-61  f64  MissingValue: NaN  References: 1  Unit: (NULL)
Dimension 0   Size: 5      Name: x         Coordinate-variable: ::NAP::58-58
\end{verbatim}

\subsection{Method \texttt{label}}
    \label{ooc-meta-label}

$ooc\_name$ \texttt{label}

This returns the label (title, etc.) of the NAO.
  
Example (continued from above):
  \begin{verbatim}
% $y set label "areas of squares"
% $y label
areas of squares
\end{verbatim}

\subsection{Method \texttt{link}}
    \label{ooc-meta-link}

  $ooc\_name$ 
  \texttt{link}

This returns the OOC-name of the link NAO.
  
Example (continued from above):
  \begin{verbatim}
% $y set link [nap 42]
% [$y link]
42
\end{verbatim}

\subsection{Method \texttt{missing}}
    \label{ooc-meta-missing}

  $ooc\_name$ 
  \texttt{missing}

This returns the missing value. This is the value used to indicate
  null or undefined data.
  
Example (continued from above):
  \begin{verbatim}
% $y miss
NaN
\end{verbatim}

\subsection{Method \texttt{ooc}}
    \label{ooc-meta-ooc}

$ooc\_name$ \texttt{ooc} [\texttt{-keep}]

This returns the OOC-name of the NAO.

Example (continued from above):
  \begin{verbatim}
$y ooc
::NAP::61-61
\end{verbatim}

\subsection{Method \texttt{rank}}
    \label{ooc-meta-rank}

  $ooc\_name$ 
  \texttt{rank}
  \\
  

This returns the rank (number of dimensions).
  
Example (continued from above):
  \begin{verbatim}
% $y rank
1
\end{verbatim}

\subsection{Method \texttt{sequence}}
    \label{ooc-meta-sequence}

$ooc\_name$ \texttt{sequence} [\texttt{-keep}]

This returns the sequence number of the NAO. E.g. 42 for 
  \texttt{::NAP::42-9}

Example (continued from above):
  \begin{verbatim}
% $y seq
61
\end{verbatim}

\subsection{Method \texttt{shape}}
    \label{ooc-meta-shape}

  


  $ooc\_name$ 
  \texttt{shape}
  \\
  

This returns the shape, which is a vector of dimension sizes.
  
Example (continued from above):
  \begin{verbatim}
% $y shape
5
\end{verbatim}

\subsection{Method \texttt{slot}}
    \label{ooc-meta-slot}

$ooc\_name$ \texttt{slot} [\texttt{-keep}]

This returns the slot number of the NAO. E.g. 9 for 
  \texttt{::NAP::42-9}

Example (continued from above):
  \begin{verbatim}
% $y sl
61
\end{verbatim}

\subsection{Method \texttt{step}}
    \label{ooc-meta-step}

$ooc\_name$ \texttt{step}

This returns a code which indicates whether step sizes of a vector
  are equal, and if not, their sign. Nap uses this information for
  efficiency. It indicates whether a vector (not relevant for other
  ranks) is monotonically ascending/descending, and if so whether it is
  an arithmetic progression (AP). The result code is one of following
  strings:
\begin{bullets}
    \item `\texttt{+-}': at least one positive step and one negative step
    \item `\texttt{$>$= 0}': all steps $>$= 0
    \item `\texttt{$<$= 0}': all steps $<$= 0
    \item `\texttt{AP}': equal steps (except final one which may be shorter)
	i.e. {\em Arithmetic Progression}
\end{bullets}
  
Example (continued from above):
  \begin{verbatim}
% $x step
AP
% $y step
>= 0
\end{verbatim}

\subsection{Method \texttt{unit}}
    \label{ooc-meta-unit}

  $ooc\_name$ 
  \texttt{unit}

This returns the unit of measure.
This may be used in the future to support arithmetic with automatic unit conversion.
At the moment Nap makes only limited use of the unit attribute.
For example coordinate variables with the unit
\texttt{degrees\_east} are recognised at longitudes and treated differently.

Example (continued from above):
  \begin{verbatim}
% $y set unit seconds
% $y unit
seconds
\end{verbatim}

