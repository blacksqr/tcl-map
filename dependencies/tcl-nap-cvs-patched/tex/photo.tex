%  $Id: photo.tex,v 1.3 2006/04/06 09:54:47 dav480 Exp $ 
    % Nap Photo Image Format

\section{Nap Photo Image Format}
    \label{photo}

  Nap defines a new photo image format. This enables one to
  use Tk's `\texttt{image create photo}' command to produce a photo
  image from a NAO. One can also use the photo image write operation to
  produce a NAO from a photo image.
  

The data type of the NAO is normally 
  \texttt{u8} (8-bit unsigned integer). Any other type will be
  converted to 
  \texttt{u8}.
  

The rank can be 2 or 3. A matrix gives a grey-scale image. Colour
  requires three dimensions. In this case there normally are three
  layers corresponding to red, green and blue. If there are only two
  layers then the first is used for both red and green (which together
  give yellow).
  

The name of the new photo image format is `\texttt{NAO}'.
  

The following example (input only shown) creates and displays a
  grey-scale photo image from a 
  \texttt{u8} matrix:
  \begin{verbatim}
nap "u = u8(reshape(0 .. 255, 2 # 255))"
set i [image create photo -format NAO -data $u]
button .b -image $i
pack .b
\end{verbatim}

  

The following example (input only shown) creates and displays a
  colour photo image from a three-dimensional 
  \texttt{u8} array. It then writes this image to a GIF file named
  `\texttt{n.gif}'.
  \begin{verbatim}
destroy .b
nap "u = u8(reshape({32768#0 65536#255}, {3 2#256}))"
set i [image create photo -format NAO -data $u]
button .b -image $i
pack .b
$i write n.gif -format GIF
\end{verbatim}

  

The following example (input and output shown) first creates a
  photo image by reading this GIF file named `\texttt{n.gif}'. Then a new NAO is created and assigned the
  name `\texttt{abc}'.
  \begin{verbatim}
% set pi [image create photo -file n.gif]
image8
% $pi write abc -format NAO
% $abc header
::NAP::2790-2790  u8  MissingValue: (NULL)  References: 1  Unit: (NULL)
Dimension 0   Size: 3      Name: (NULL)    Coordinate-variable: (NULL)
Dimension 1   Size: 256    Name: (NULL)    Coordinate-variable: (NULL)
Dimension 2   Size: 256    Name: (NULL)    Coordinate-variable: (NULL)
\end{verbatim}

