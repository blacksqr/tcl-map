%  $Id: ooc_data.tex,v 1.7 2006/10/27 08:04:56 dav480 Exp $ 
    % OOC Methods: Data

\section{OOC Methods which return Data Values (with or without metadata)}
    \label{ooc-data}

\subsection{Method \texttt{all}}
    \label{ooc-data-all}

  $ooc\_name$ 
  \texttt{all} 
  [\texttt{-format} $format$]
  [\texttt{-columns} $int$]
  [\texttt{-lines} $int$]
  [\texttt{-list}]
  [\texttt{-missing} $text$]
  [\texttt{-keep}]

This provides both data and metadata from a NAO. However it does
  not provide 
  \emph{all} information despite the name!

The following switches are allowed:
  \\
  \texttt{-format} 
  $format$: C format (default: Use internal format if any,
  else \texttt{""} meaning automatic)
  \\
  \texttt{-columns} 
  $int$: maximum number of columns (default: 6) (\texttt{-1}: no limit)
  \\
  \texttt{-lines} 
  $int$: maximum number of lines (default: 20) (\texttt{-1}: no limit)
  \\
  \texttt{-list}: print in tcl list form (using braces) e.g. `\texttt{\{1 9 2\}}'
  \\
  \texttt{-missing} 
  $text$: text printed for missing value (default:
  "\_")
  \\
  \texttt{-keep}: Do not delete NAO with reference count of 0

The \texttt{-keep} switch is only useful in special situations where one wants to retain a
NAO which would otherwise be deleted.
An example is a debugging tool which displays NAOs but should not destroy them.

The \texttt{all} method provides the same information as the two commands:
  \\
  $ooc\_name$ 
  \texttt{header}
  \\
  $ooc\_name$ 
  \texttt{value} 
  \texttt{-format} $format$ 
  \texttt{-columns} $int$ 
  \texttt{-lines} $int$ 
  \texttt{-missing} 
  $text$

\subsubsection{Example}

  \begin{verbatim}
% [nap "{3#2 2#_ -9}"] all -miss n/a
::NAP::39-39  i32  MissingValue: -2147483648  References: 0  Unit: (NULL)
Dimension 0   Size: 6      Name: (NULL)    Coordinate-variable: (NULL)
Value:
2 2 2 n/a n/a -9
\end{verbatim}

\subsection{Method \texttt{value}}
    \label{ooc-data-value}

  $ooc\_name$ 
  \texttt{value} 
  [\texttt{-format} $format$]
  [\texttt{-columns} $int$]
  [\texttt{-lines} $int$]
  [\texttt{-list}]
  [\texttt{-missing} $text$]
  [\texttt{-keep}]

This returns data values. The default value is 
  \texttt{-1} for both the switches 
  \texttt{-columns} and 
  \texttt{-lines}, giving the entire array.
  

The following switches are allowed:
  \\
  \texttt{-format} 
  $format$: C format (default: Use internal format if any,
  else \texttt{""} meaning automatic)
  \\
  \texttt{-columns} 
  $int$: maximum number of columns (default: 
  \texttt{-1} i.e. no limit)
  \\
  \texttt{-lines} 
  $int$: maximum number of lines (default: 
  \texttt{-1} i.e. no limit)
  \\
  \texttt{-list}: print in tcl list form (using braces) e.g. `\texttt{\{1 9 2\}}'
  \\
  \texttt{-missing} 
  $text$: text printed for missing value (default:
  "\_")
  \\
  \texttt{-keep}: Do not delete NAO with reference count of 0

\subsubsection{Example}

  \begin{verbatim}
% nap "y = (0 .. 2 ... 0.5) ** 2"
::NAP::61-61
% $y val -format %0.3f
0.000 0.250 1.000 2.250 4.000
\end{verbatim}

\subsection{Default method}
    \label{ooc-data-default-method}

  $ooc\_name$ 
  [\texttt{-format} $format$]
  [\texttt{-columns} $int$]
  [\texttt{-lines} $int$]
  [\texttt{-list}]
  [\texttt{-missing} $text$]
  [\texttt{-keep}]

This returns data values in a similar fashion to the 
  \texttt{value} method, except that default line and column limits
  restrict the size.
  

The following switches are allowed:
  \\
  \texttt{-format} 
  $format$: C format (default: Use internal format if any,
  else \texttt{""} meaning automatic)
  \\
  \texttt{-columns} 
  $int$: maximum number of columns (default: 
  \texttt{6}) (\texttt{-1}: no limit)
  \\
  \texttt{-lines} 
  $int$: maximum number of lines (default: 
  \texttt{20}) (\texttt{-1}: no limit)
  \\
  \texttt{-list}: print in tcl list form (using braces) e.g. `\texttt{\{1 9 2\}}'
  \\
  \texttt{-missing} 
  $text$: text printed for missing value (default: 
  \texttt{"\_"})
  \\
  \texttt{-keep}: Do not delete NAO with reference count of 0

\subsubsection{Examples}

The following example shows why and how to use
  switch 
  \texttt{-columns} (abbreviated to 
  \texttt{-c}):
  \begin{verbatim}
% nap "m = reshape(0 .. 99, {10 12})"
::NAP::50-50
% $m
 0  1  2  3  4  5 ..
12 13 14 15 16 17 ..
24 25 26 27 28 29 ..
36 37 38 39 40 41 ..
48 49 50 51 52 53 ..
60 61 62 63 64 65 ..
72 73 74 75 76 77 ..
84 85 86 87 88 89 ..
96 97 98 99  0  1 ..
 8  9 10 11 12 13 ..
% $m -c -1
 0  1  2  3  4  5  6  7  8  9 10 11
12 13 14 15 16 17 18 19 20 21 22 23
24 25 26 27 28 29 30 31 32 33 34 35
36 37 38 39 40 41 42 43 44 45 46 47
48 49 50 51 52 53 54 55 56 57 58 59
60 61 62 63 64 65 66 67 68 69 70 71
72 73 74 75 76 77 78 79 80 81 82 83
84 85 86 87 88 89 90 91 92 93 94 95
96 97 98 99  0  1  2  3  4  5  6  7
 8  9 10 11 12 13 14 15 16 17 18 19
\end{verbatim}

  

The following example shows how to use switch 
  \texttt{-format} (abbreviated to 
  \texttt{-f}) to include a dollar prefix and display two decimal
  places:
  \begin{verbatim}
% [nap "{15 3.2 999}"] -f {$%.2f}
$15.00 $3.20 $999.00
\end{verbatim}

  

The following example shows how to use switch 
  \texttt{-list}:
  \begin{verbatim}
% [nap "reshape(1.5 .. -1.5, {2 5})"] -list
{
{  1.5  0.5 -0.5 -1.5  1.5 }
{  0.5 -0.5 -1.5  1.5  0.5 }
}
\end{verbatim}

\subsection{Format Conversion Strings}
    \label{ooc-data-format-strings}

The 
  \texttt{-format} option (or NAO internal 
  \emph{format} field) specifies a 
  \emph{format conversion string} similar to that used in the standard
  Tcl 
  \texttt{format} command (which is based on the ANSI C 
  \texttt{sprintf()} function). Such strings have the form
  \\
  \texttt{\%}[$flags$][$width$][\texttt{.}$precision$]$char$
  \\where:
  \begin{itemize}
    \item 
    $flags$ is a string containing any of the following
    characters in any order:
    \\
    \texttt{-}: left justify
    \\
    \texttt{+}: include sign
    \\
    \emph{space}: include space prefix if no sign
    \\
    \texttt{0}: leading zeros
    \\
    \texttt{\#}: alternate form
    \item 
    $width$ is the minimum field width
    \item 
    $precision$ is the
    \begin{itemize}
      \item minimum number of digits for 
      \texttt{d}, 
      \texttt{i}, 
      \texttt{o}, 
      \texttt{x}, 
      \texttt{X} or 
      \texttt{u} conversions.
      \item number of digits after `.' for 
      \texttt{e}, 
      \texttt{E} or 
      \texttt{f} conversions.
      \item number of significant digits for 
      \texttt{g} or 
      \texttt{G} conversions.
    \end{itemize}
    \item 
    $char$ specifies conversion as in the following table:

    \begin{tabular}{|l|l|}
      \hline 
          $char$ & \textbf{Convert to}
      \\
        \hline 
        \hline 
          \texttt{d}, \texttt{i} & signed decimal integer
        \\
        \hline 
          \texttt{o} & unsigned octal integer
        \\
        \hline 
          \texttt{x}, \texttt{X} & unsigned hexadecimal integer
        \\
        \hline 
          \texttt{u} & unsigned decimal integer
        \\
        \hline 
          \texttt{c} & character
        \\
        \hline 
          \texttt{f} & decimal number in form [\texttt{-}]$mmm$\texttt{.}$ddd$,
	\\
	& where number of $d$s is specified by the $precision$.
        \\
	  & Default $precision$ is 6; $precision$ of 0 suppresses the `\texttt{.}'.
        \\
        \hline 
          \texttt{e}, \texttt{E}
	  & decimal number in form [\texttt{-}]$m$\texttt{.}$dddddd$\texttt{e}[\texttt{-}]$xx$
	  or [\texttt{-}]$m$\texttt{.}$dddddd$\texttt{E}[\texttt{-}]$xx$,
	  \\
	  & where number of $d$s is specified by the $precision$.  
	  \\
	  & Default $precision$ is 6; $precision$ of 0 suppresses the `\texttt{.}'.
        \\
        \hline 
          \texttt{g}, \texttt{G} & \texttt{\%e} or \texttt{\%E} are used if $exponent$ $<$ $-$4 or $exponent$ $\ge$ $precision$;
	  \\
	  & otherwise \texttt{\%f} is used.
	  \\
	  & Trailing zeros and decimal points are suppressed.
        \\
    \hline
\end{tabular}

  \end{itemize}
  

The following example displays the same data using each of these
  codes. Note that (unlike C and the standard Tcl 
  \texttt{format} command) any data-type can be displayed with any
  code.
  \begin{verbatim}
% foreach code {d i o x X u c f E e g G} {
    puts "$code: [[nap "88 .. 92"] -f "%$code"]"
}
d: 88 89 90 91 92
i: 88 89 90 91 92
o: 130 131 132 133 134
x: 58 59 5a 5b 5c
X: 58 59 5A 5B 5C
u: 88 89 90 91 92
c: XYZ[\
f: 88.000000 89.000000 90.000000 91.000000 92.000000
E: 8.800000E+01 8.900000E+01 9.000000E+01 9.100000E+01 9.200000E+01
e: 8.800000e+01 8.900000e+01 9.000000e+01 9.100000e+01 9.200000e+01
g: 88 89 90 91 92
G: 88 89 90 91 92
\end{verbatim}

