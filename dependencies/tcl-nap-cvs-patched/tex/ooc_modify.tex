%  $Id: ooc_modify.tex,v 1.12 2007/10/24 10:33:00 dav480 Exp $ 
    % OOC Methods: Modify

\section{OOC Methods which Modify NAO}

\subsection{Method \texttt{draw}}
    \label{ooc-modify-draw}

$ooc\_name$ \texttt{draw} $xy$ [$value$]

This draws a polyline in the NAO 
  $ooc\_name$, which must be type 
  \texttt{f32} in the current version. It sets data elements on the
  polyline defined by NAO 
  $xy$ to the value of scalar NAO 
  $value$ (default: missing value). NAO 
  $xy$ can be:
\begin{bullets}
    \item matrix with 2 rows, row 0 is x values, row 1 is y values
    \item vector of y values with coordinate variable (CV) of x
    values
    \item vector of y values without CV (x defaults to 0 1 2 3 $\ldots$)
\end{bullets}
The polyline is not closed, so to draw a polygon, the first point should be duplicated at the end.

Example:
\begin{verbatim}
% [nap "z = reshape(0f32, 2 # 8)"] draw "{{2 2 4 4}{2 4 4 2}}" 1
% $z value
0 0 0 0 0 0 0 0
0 0 0 0 0 0 0 0
0 0 1 0 1 0 0 0
0 0 1 0 1 0 0 0
0 0 1 1 1 0 0 0
0 0 0 0 0 0 0 0
0 0 0 0 0 0 0 0
0 0 0 0 0 0 0 0
\end{verbatim}

\subsection{Method \texttt{fill}}
    \label{ooc-modify-fill}

$ooc\_name$ \texttt{fill} $xy$ [$value$]

This fills a polygon in the NAO 
  $ooc\_name$, which must be type 
  \texttt{f32} in the current version. It sets data elements within
  the polygon defined by NAO 
  $xy$ to the value of the scalar NAO 
  $value$ (default: missing value).
  \\NAO 
  $xy$ can be:
\begin{bullets}
    \item matrix with 2 rows, row 0 is x values, row 1 is y values
    \item vector of y values with coordinate variable (CV) of x
    values
    \item vector of y values without CV (x defaults to 0 1 2 3 $\ldots$)
\end{bullets}
The polygon is closed (unlike \texttt{draw} method).

Example:
  \begin{verbatim}
% [nap "z = reshape(0f32, 2 # 8)"] fill "{{2 2 4 4}{2 4 4 2}}" 1
% $z value
0 0 0 0 0 0 0 0
0 0 0 0 0 0 0 0
0 0 1 1 1 0 0 0
0 0 1 1 1 0 0 0
0 0 1 1 1 0 0 0
0 0 0 0 0 0 0 0
0 0 0 0 0 0 0 0
0 0 0 0 0 0 0 0
\end{verbatim}

\subsection{Method \texttt{set}}
    \label{ooc-modify-set}

  $ooc\_name$ 
  \texttt{set} 
  $attribute$ $arg$ $arg$ $\ldots$

This modifies the NAO attribute specified by 
  its name $attribute$. The sub-methods corresponding to these
  attributes are discussed below in separate sections. Note that the
  name of each attribute is the same as that of the method which
  returns its value.

\subsubsection{set coordinate}
    \label{ooc-modify-set-coordinate}

$ooc\_name$ \texttt{set coordinate} $coord\_var$ $coord\_var$ $coord\_var$ $\ldots$

This sets one or more coordinate variables.
  \\
  $coord\_var$ can be a name, OOC-name or 
  \texttt{""}.
  \\If 
  $coord\_var$ is a valid name then this is also used as
  dimension name if this is undefined.
  \\If 
  $coord\_var$ is 
  \texttt{""} then any existing coordinate variable is
  removed.
  \\If the number of 
  $coord\_var$ arguments $<$ rank then trailing values default
  to 
  \texttt{""}. Thus the following command removes all
  coordinate variables:
\\
$ooc\_name$ set coo

An example is provided in section \ref{ooc-meta-coordinate}.

\subsubsection{set count}
    \label{ooc-modify-set-count}

  $ooc\_name$ \texttt{set count} [$int$] [\texttt{-keep}]

This sets or increments the reference count. One situation where
  this facility is needed is where a GUI window points to a NAO and
  must be retained until the window is destroyed.
  \\
  \texttt{-keep}: Do not delete NAO (even if new value of count is 0 or less)
  \\
If $int$ is signed then add it to reference count.
  \\
If $int$ is unsigned then set reference count to $int$.
  \\
If $int$ not specified then add 1 to reference count (i.e.  treat as `+1')

\subsubsection{set datatype}
    \label{ooc-modify-set-datatype}

  $ooc\_name$ \texttt{set datatype} [$datatype$]

This changes the data-type without any change to the (binary) data.
The new rank is 1.
If the new data-type is larger than the old then any partial new element at the end is discarded.
  \\
If $datatype$ is not specified then it defaults to the original (when NAO created) data-type.

The following example creates a NAO with 
data-type \texttt{c8},
changes the data-type to \texttt{u8},
then changes it back to \texttt{c8}.
  \begin{verbatim}
% nap "c = 'ABC'"
::NAP::13-13
% $c all
::NAP::13-13  c8  MissingValue: (NULL)  References: 1
Dimension 0   Size: 3      Name: (NULL)    Coordinate-variable: (NULL)
Value:
ABC
% $c set datatype u8
% $c all
::NAP::13-13  u8  MissingValue: (NULL)  References: 1
Dimension 0   Size: 3      Name: (NULL)    Coordinate-variable: (NULL)
Value:
65 66 67
% $c set datatype
% $c all
::NAP::13-13  c8  MissingValue: (NULL)  References: 1
Dimension 0   Size: 3      Name: (NULL)    Coordinate-variable: (NULL)
Value:
ABC
\end{verbatim}

\subsubsection{Set dimension}
    \label{ooc-modify-set-dimension}

  $ooc\_name$ 
  \texttt{set dimension} 
  $dim\_name$ 
  $dim\_name$ 
  $dim\_name$ $\ldots$

This sets one or more dimension names.
  \\If 
  $dim\_name$ is a tcl name pointing to a NAO then this also
  defines the coordinate variable if this is undefined.
  \\If 
  $dim\_name$ is 
  \texttt{""} then any existing dimension name is
  removed.
  \\If the number of 
  $dim\_name$s $<$ rank then trailing values default to 
  \texttt{""}. Thus the following command removes all
  dimension names:
\\
$ooc\_name$ set dim

Example (continued from section \ref{ooc-meta}):
  \begin{verbatim}
% $x set dim time
% $x dim
time
\end{verbatim}

\subsubsection{set format}
    \label{ooc-modify-set-format}

$ooc\_name$ \texttt{set format} [$string$]

This sets the C format for printing. The default is NULL i.e. no
  format.
  

An example is provided in section \ref{ooc-meta-format}.

\subsubsection{set label}
    \label{ooc-modify-set-label}

$ooc\_name$ \texttt{set label} [$string$]

This sets the label (title). The default is NULL i.e. no
  label.
  

An example is provided in section \ref{ooc-meta-label}.

\subsubsection{Set link}
    \label{ooc-modify-set-link}

$ooc\_name$ \texttt{set link} [$nao$]

This sets the link slot number to point to a NAO
  \\The default is NULL i.e. no link.
  
As discussed in section \ref{nao}, this enables additional information to be attached to a NAO.
For example the linked NAO could contain an error estimate.
It could also be a boxed NAO pointing to any number of other NAOs (which might correspond
to the attributes of a netCDF variable).

An example is provided in section \ref{ooc-meta-link}.

\subsubsection{Set missing}
    \label{ooc-modify-set-missing}

$ooc\_name$ \texttt{set missing} [$value$]

This sets the missing value. The default is NULL i.e. no missing
  value.
  
Example (continued from section \ref{ooc-meta}):
  \begin{verbatim}
% $x set miss 0
% $x
_ 0.5 1 1.5 2
\end{verbatim}

Note that the value of 0 is now treated as missing.

\subsubsection{Set shape}
    \label{ooc-modify-set-shape}

$ooc\_name$ \texttt{set shape} [$shape$]

This sets the shape of the NAO to that defined by the vector expression $shape$.
The default shape corresponds to a vector containing all the original (when NAO created) elements.
  
Example:
  \begin{verbatim}
% [nap "m = {{0 2 4 6 8}{1 3 5 7 9}}"]
0 2 4 6 8
1 3 5 7 9
% $m set shape "{2 2 2}"
% $m
0 2
4 6

8 1
3 5
% $m set shape
% $m value
0 2 4 6 8 1 3 5 7 9
\end{verbatim}

\subsubsection{Set unit}
    \label{ooc-modify-set-unit}

$ooc\_name$ \texttt{set unit} [$unit$]

This sets the unit of measure. The default is NULL i.e. no
  unit.
  

An example is provided in section \ref{ooc-meta-unit}.

\subsubsection{Set value}
    \label{ooc-modify-set-value}

$ooc\_name$ \texttt{set value} [$value$] [$index$]

This sets the value of data elements selected by NAO 
  $index$ to new values copied from successive elements of NAO
  $value$.
  

If 
  $value$ is absent or 
  \texttt{""} it is treated as the null (missing) value.
  Elements of 
  $value$ are recycled if necessary.
  

If 
  $index$ is absent or 
  \texttt{""} it is treated as the whole array. 
The $index$ expression can include the operators `\texttt{,}' (giving 
{cross-product-indexing} as discussed in section \ref{indexing-Cross-product-index})
  and `\texttt{@@}' (giving 
{indirect indexing} as discussed in section \ref{indexing-indirect-indexing}).
  However {full-indexing} (see section \ref{indexing-Full-index}) is not yet implemented.
  

Note that the Nap assignment operator `\texttt{=}' does not allow its left operand to be indexed,
  as in `\texttt{v(2) = 9}'. However the `\texttt{set value}' method does provide a way to modify all
  or part of an array. The following example sets element 2 of 
  \texttt{v} to 9:
  \begin{verbatim}
% nap "v = {3 6 4 7}"
::NAP::52-52
% $v set value 9 2
% $v
3 6 9 7
\end{verbatim}

  

The following example changes elements 1, 2 and 3. Note how the
  elements of 
  $value$ are recycled.
  \begin{verbatim}
% $v set value "{8 5}" "1 .. 3"
% $v
3 8 5 8
\end{verbatim}

  

The following example illustrates indirect indexing:
  \begin{verbatim}
% $v set coo "10 .. 40 ... 10"
% $v set value 0 "@@{20 40}"
% $v
3 0 5 0
\end{verbatim}

  

The following example illustrates cross-product indexing:
  \begin{verbatim}
% [nap "matrix = {{0 2 4}{1 3 5}}"]
0 2 4
1 3 5
% $matrix set v 9 0,2; # Note that 'value' can be abbreviated to 'v'
% $matrix
0 2 9
1 3 5
% $matrix set v "-1 .. -9" ",{0 2}"
% $matrix
-1  2 -2
-3  3 -4
\end{verbatim}

