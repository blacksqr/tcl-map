%  $Id: ack.tex,v 1.8 2007/01/01 23:31:25 dav480 Exp $ 
    % Acknowledgments

\section{Acknowledgments}

  Ken Iverson (who died in 2004) is the father of array
  processing. Nap does not use the radical mathematical conventions of
  his languages 
  \href{http://www.acm.org/sigapl/}{APL}, and 
  \href{http://www.jsoftware.com/}{J}, but he must be
  acknowledged as the source of the fundamental concepts upon which NAP
  is based. Nap has also adopted various J conventions such as that for
  floating-point constants (e.g. allowing 0.5 to be written in rational
  form as 
  \texttt{1r2} and allowing powers of $\pi$ as in 
  \texttt{1p1}).
  \par Nap would never have happened if the author (Harvey Davies) had
  not worked with Rhys Francis and Ian Mathieson on development of the
  data-parallel modelling language 
  \textit{DPML} and learned from them things like 
  \textit{yacc}. This project met a premature death but we did learn a
  lot. It is hoped to implement unit calculus (automatic unit
  conversion and definition) in Nap as was done in DPML.

The other strong influence on Nap was 
  \href{http://www.unidata.ucar.edu/packages/netcdf/index.html}{netCDF}.
  It is amazing that a portable array file format did not exist until
  Russ Rew and the late Glenn Davis developed netCDF in the early
  1990s. This work was based on 
  \href{http://nssdc.gsfc.nasa.gov/cdf/cdf-home.html}{CDF}, which
  was developed by Michael Gough and Lloyd Treinish. 
Further details of the history of CDF and netCDF are available at
\\
\href{http://www.unidata.ucar.edu/software/netcdf/credits.html}
{http://www.unidata.ucar.edu/software/netcdf/credits.html}.
Nap is designed to
  read and write netCDF (and the similar 
  \href{http://www.hdfgroup.org/}{HDF}) files and stores data in
  memory in stuctures called 
  \textit{NAOs}, which have similar properties to netCDF variables.
  \par
    My CSIRO colleague Peter Turner made significant
  contributions to the Nap code. He wrote the C code for:
\begin{bullets}
    \item NAO photo-image handler
    \item 
    \textit{draw} and 
    \textit{fill} OOC methods
    \item morphological functions (moving\_range, dilate, erode)
\end{bullets}
He also made significant contributions to the following Tcl
  library files:
\begin{bullets}
    \item 
      \texttt{browse\_var.tcl}
    \item 
      \texttt{caps\_nap\_menu.html}
    \item 
      \texttt{colour.tcl}
    \item 
      \texttt{hdf.tcl}
    \item 
      \texttt{pal.tcl}
    \item 
      \texttt{plot\_nao.tcl}
    \item 
      \texttt{proc\_lib.tcl}
\end{bullets}

The \texttt{nap\_land\_flag} command 
(see section \ref{land-flag-nap-land-flag})
is based on code which was originally written by Dr.
  Chris. Mutlow at the Rutherford Appleton Laboratory in England. This
  code has been adapted for Nap by Peter Turner and Harvey Davies.
  \par Past and present members of The Caps development group (Ian Grant,
  Edward King, Jenny Lovell, Paul Tildesley, Peter Turner and Chris
  Rathbone) have provided ideas, support and feedback over the years.
  Mark Collier and Janice Bathols were early users of Nap for
  processing results from atmospheric modelling. Their success and
  enthusiasm has has done a great deal to convert others to 
  \textit{napism}. 
  \par Dr Takeshi Enomoto, Earth Simulator Center, maintains the 
  \href{http://fink.sourceforge.net/pdb/package.php/nap}{Nap Fink site}, which provides distributions for the 
  \textit{Mac OS X} and 
  \textit{Darwin} platforms.
