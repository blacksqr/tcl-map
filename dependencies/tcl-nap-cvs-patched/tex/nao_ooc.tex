%  $Id: nao_ooc.tex,v 1.7 2006/09/20 08:48:49 dav480 Exp $ 
    % NAOs, OOCs and Nap command

\section{NAOs, OOCs and \texttt{nap} command}

  The standard Tcl 
  \texttt{expr} command produces a single (scalar) number. For
  example:
  \begin{verbatim}
% expr "2 * 3.5"
7.0
\end{verbatim}

Note that the Tcl result of this command is simply the text
`\texttt{7.0}'.
  

However the 
  \texttt{nap} command often produces an array containing millions
  of numbers. It is not practical to store, process and display
  millions of numbers as text. It is far better to store and process
  them in binary form. The binary values are stored in a memory object
  called an 
  \emph{N-dimensional Array Object (NAO)}, which also includes other
  information such as the number of elements and the data-type. Nap is
  designed to efficiently handle large arrays, but let us begin
  demonstrating it on the above simple scalar expression:
  \begin{verbatim}
% nap "2 * 3.5"
::NAP::16-16
\end{verbatim}

What is this strange Tcl result `\texttt{::NAP::16-16}'? It is the ID of the NAO result and
  also the name of the command that is used to examine the NAO and make
  changes to it. Such a command is called an 
  \emph{Object-Oriented Command (OOC)}. The ID is called the 
  \emph{OOC-name}. Continuing the above example, let's execute
  the OOC by typing its name `\texttt{::NAP::16-16}':
  \begin{verbatim}
% ::NAP::16-16
7
\end{verbatim}

This displays the value in the NAO.
  

However the following attempt to repeat the command fails because
  the NAO and its associated OOC were automatically deleted at the end
  of the first execution of the OOC. Nap detected the fact that this
  NAO was not referenced by anything and it was therefore treated as 
  \emph{use once and then discard}.
  \begin{verbatim}
% ::NAP::16-16
invalid command name "::NAP::16-16"
\end{verbatim}

  

Tcl syntax allows a command to include another command within
  square brackets 
  \texttt{[]}. First the bracketed command is executed and its Tcl
  result replaces it. Then the modified whole command is executed. So
  the above commands can be simplified by enclosing the 
  \texttt{nap} command in brackets as follows:
  \begin{verbatim}
% [nap "2 * 3.5"]
7
\end{verbatim}

  

Nap expressions can contain the assignment operator `\texttt{=}' with a Tcl variable name on its left and any
  expression on its right. For example:
  \begin{verbatim}
% nap "result = 2 * 3.5"
::NAP::24-24
\end{verbatim}

The following shows that this sets the Tcl variable 
  \texttt{result} to the string value `\texttt{::NAP::24-24}'.
  \begin{verbatim}
% set result
::NAP::24-24
\end{verbatim}

Tcl syntax replaces `\texttt{\$}
  $name$' by the contents of variable 
  $name$. So in our example `\texttt{\$result}' is replaced by `\texttt{::NAP::24-24}', as shown in the following:
  \begin{verbatim}
% $result
7
\end{verbatim}

The fact that this NAO is referenced by something (the variable
`\texttt{result}') means that it is not deleted after it
  executes. So we can repeat the command:
  \begin{verbatim}
% $result
7
\end{verbatim}

We can also use variable names within Nap expressions. For
example:
  \begin{verbatim}
% [nap "result + 4"]
11
\end{verbatim}

Note that no `\texttt{\$}' is needed before a variable name in a Nap expression.
Variables can also contain numeric strings, as in:
  \begin{verbatim}
% set offset 8.2
8.2
% [nap "result + offset"]
15.2
\end{verbatim}

  

An OOC can have arguments. The following example demonstrates the
  argument 
  \texttt{all}, which requests additional information about the
  NAO.
  \begin{verbatim}
% $result all
::NAP::24-24  f64  MissingValue: NaN  References: 1
Value:
7
\end{verbatim}

  

The following additional information is provided:
  \begin{bullets}
    \item 
    \emph{OOC-name} 
    \texttt{::NAP::24-24}
    \item 
    \emph{Data-type} 
    \texttt{f64} (64-bit floating-point)
    \item 
    \emph{Missing-value} 
    \texttt{NaN} (special value for missing data)
    \item 
    \emph{Reference-count} The value is 
    \texttt{1} because there is one variable (\texttt{result}) pointing to this NAO.
    If it were 0 the NAO would be deleted.
  \end{bullets}
