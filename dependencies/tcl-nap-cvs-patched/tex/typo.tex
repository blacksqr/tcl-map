%  $Id: typo.tex,v 1.3 2006/04/06 09:54:47 dav480 Exp $ 
    % Nap Typographic Conventions

\section{Typographic Conventions}

Hyperlinks are blue, as in
\href{http://tcl-nap.sourceforge.net/index.php}{tcl-nap}.
Internal references are red, as in section \ref{overview}.
Both can be clicked on if you are reading this on a screen rather than paper.
Try it.

Consider the following example:
\\
\texttt{count(}$x$[\texttt{,} $r$]\texttt{)}

The font used for `\texttt{count(,)}' indicates this is 
    \textit{literal text}. In other words this is exactly what appears
    on the screen (which could be either output or typed input). Note
    that `$x$' 
    and `$r$' are in italics (with slanted font), which
    indicates these are 
    \textit{formal argument names} rather than literal text. You
    replace such names with whatever is desired.
    
 Optional arguments are indicated using two alternative
    conventions. In the above example they are enclosed in brackets,
    indicating that `\texttt{,} $r$' is optional. This convention is common in
    computing documentation. Use has also been made of the the
    alternative (commonly used in Tcl documentation) of surrounding the
    optional component with question marks. For example:
    \\
    \texttt{count(}$x$ ?\texttt{,} $r$?\texttt{)}
    
 Alternatives are indicated with a vertical bar `\texttt{|}', as
    in the following:
    \\
    \texttt{nap\_info bytes|sequence}
    \\
    which indicates that the argument can be either `\texttt{bytes}' or `\texttt{sequence}'.
