%  $Id: geog.tex,v 1.2 2006/03/02 01:44:59 dav480 Exp $ 
    % NAP Library: Geographic
    \section{Geographic Functions}

  \subsection{
    \label{Introduction}Introduction
  }
The following functions are defined in the files 
  \texttt{geog.tcl} and 
  \texttt{land.tcl}. Note that 
  \texttt{geog.tcl} also defines 
  \href{geog-proc.html}{Geographic Procedures}.
  \subsection{
    \label{Wind}Divergence and Vorticity of Wind
  }
Functions 
  \texttt{div\_wind(}$u$\texttt{,} $v$\texttt{)}
and 
  \texttt{vorticity\_wind(}$u$\texttt{,} $v$\texttt{)}
give the divergence (second$^{-1}$) and vorticity (second$^{-1}$)
  of a 2D wind, where
  \\
  $u$ is matrix containing zonal (x-component i.e. from west
  to east) wind in metres/sec.
  \\
  $v$ is matrix containing meridional (y-component i.e. from
  south to north) wind in metres/sec.
  \\Coordinate variables of 
  $u$ and 
  $v$ are latitude ($^{\circ}$N) and longitude ($^{\circ}$E).
  \par Let 
  $\theta$ be latitude in radians, 
  $\phi$ be longitude in radians, 
  $r$ be radius of earth in metres.

  The divergence of a 2D wind is defined as
\[
  \frac{\partial u}{\partial x} + \frac{\partial v}{\partial y}
\]
Function \texttt{div\_wind} uses the equivalent formula
\[
  \frac{\frac{\partial u}{\partial \phi} + \frac{\partial(v \cos \theta)}{\partial \theta})}
  {r cos \theta}
\]
  The vorticity of a 2D wind is defined as
\[
  \frac{\partial v}{\partial x} - \frac{\partial u}{\partial y}
\]
  Function 
  \texttt{vorticity\_wind} uses the equivalent formula
\[
  \frac{\frac{\partial v}{\partial \phi} - \frac{\partial(u \cos \theta)}{\partial \theta})}
  {r cos \theta}
\]

  \subsection{
    \label{latlon}Things related to Latitude and Longitude
  }

  \subsubsection{
    \label{area-on-globe} \texttt{area\_on\_globe(}$latitude$\texttt{,} $longitude$\texttt{)}
  }
Given latitude and longitude vectors, this function calculates a
  matrix whose values are fractions of the Earth's surface area
  corresponding to cells defined by the functions 
  \ref{merid-bounds} \texttt{merid-bounds(} $longitude$ \texttt{)} and 
  \ref{zone-bounds} \texttt{zone-bounds(} $latitude$ \texttt{)} .
  \par Example:
  \begin{verbatim}
% [nap "lats = 90 .. -90 ... -45"]
90 45 0 -45 -90
% [nap "lons = -180 .. 180 ... 45"] value
-180 -135 -90 -45 0 45 90 135 180
% [nap "area = area_on_globe(lats, lons)"] value -format %.4f
0.0046 0.0092 0.0092 0.0092 0.0092 0.0092 0.0092 0.0092 0.0046
0.0156 0.0313 0.0313 0.0313 0.0313 0.0313 0.0313 0.0313 0.0156
0.0221 0.0442 0.0442 0.0442 0.0442 0.0442 0.0442 0.0442 0.0221
0.0156 0.0313 0.0313 0.0313 0.0313 0.0313 0.0313 0.0313 0.0156
0.0046 0.0092 0.0092 0.0092 0.0092 0.0092 0.0092 0.0092 0.0046
\end{verbatim}

  \subsubsection{
    \label{fix-longitude} \texttt{fix\_longitude(}$longitude$\texttt{)}
  }
Adjust elements of longitude vector by adding or subtracting
  multiple of 360 to ensure:
  \\$-$180 $\le$ 
  $x$ 
  $_{0}$ $<$ 180
  \\0 $\le$ 
  $x$ 
  $_{i+1}$$-$ 
  $x$ 
  $_{i}$ $<$ 360
  \\Also ensure unit is 
  \texttt{degrees\_east}.
  \subsubsection{
    \label{merid-bounds} \texttt{merid\_bounds(}$longitude$\texttt{)}
  }
This function defines reasonable boundary longitudes from a
  vector giving the central longitudes of each cell. The range of
  longitudes should not exceed 360$^{\circ}$ but may be much smaller.
  \par The result is a vector with one more element than the argument.
  The midpoints between adjacent argument values define all the
  elements of the result except the first and final ones.
  \par The first and final values are defined as follows. Trial outer
  cells are added with the same width as their adjacent cell. If these
  outer cells overlap (due to global wrap-around) then both outer
  boundaries are set to the value (plus an integer multiple of 360$^{\circ}$)
  midway between the first and final values of the argument.
  \par Examples:
  \begin{verbatim}
% [nap "merid_bounds {-180 -90 0 30 90 180}"] value; #
global-range
-180 -135 -45 15 60 135 180
% [nap "merid_bounds {10 30 40 50 60}"] value; # local-range
0 20 35 45 55 65
\end{verbatim}

  \subsubsection{
    \label{merid-wt} \texttt{merid\_wt(}$longitude$\texttt{)}
  }
Calculate normalised (so sum of weights is 1) meridional weights
  from longitudes. These weights are proportional to the widths of the
  cells defined by the above function 
  \ref{merid-bounds} \texttt{merid-bounds(} $longitude$ \texttt{)} .
  \par Examples:
  \begin{verbatim}
% [nap "merid_wt {-180 -90 0 30 90 180}"] value; #
global-range
0.125 0.25 0.166667 0.125 0.208333 0.125
% [nap "merid_wt {10 30 40 50 60}"] value; # local-range
0.307692 0.230769 0.153846 0.153846 0.153846
\end{verbatim}

  \subsubsection{
    \label{zone-bounds} \texttt{zone\_bounds(}$latitude$\texttt{)}
  }
This function defines reasonable boundary latitudes from a
  vector giving the central latitudes of each cell.
  \par The result is a vector with one more element than the argument. If
  the argument is an arithmetic progression (AP) then the midpoints
  between adjacent argument values define all the elements of the
  result except the first and final ones, otherwise each of these
  boundaries separates two equal 
  \textit{surface areas} between adjacent argument values. The two
  outer boundaries are defined using steps (in latititude or area)
  equal to their adjacent step, with the obvious limit in magnitude of
  90$^{\circ}$.
  \par Examples:
  \begin{verbatim}
% [nap "zone_bounds(-80 .. 80 ... 40)"] value; # global-range
AP
-90 -60 -20 20 60 90
% [nap "zone_bounds(90 .. -90 ... -30)"] value; #
global-range AP
90 75 45 15 -15 -45 -75 -90
% [nap "zone_bounds{-80 -40 -10 10 40 80}"] value; #
global-range non-AP
-90 -54.4687 -24.0929 0 24.0929 54.4687 90
% [nap "zone_bounds(10 .. 60 ... 10)"] value; # local-range
AP
5 15 25 35 45 55 65
\end{verbatim}

  \subsubsection{
    \label{zone-wt} \texttt{zone\_wt(}$latitude$\texttt{)}
  }
Calculate normalised (so sum weights of is 1) zonal weights from
  latitudes. These weights are proportional to the surface areas of the
  cells (of longitude width 360$^{\circ}$) defined by the above function 
  \ref{zone-bounds} \texttt{zone-bounds(} $latitude$ \texttt{)} . If the Earth's radius is 
  $R$ then the area between two latitudes 
  $L_1$ and 
  $L_2$ is 
  $2\pi R^2(\sin{L_1} - \sin{L_2})$.
  \par Examples:
  \begin{verbatim}
% [nap "zone_wt(-80 .. 80 ... 40)"] value; # global-range AP
0.0669873 0.262003 0.34202 0.262003 0.0669873
% [nap "zone_wt(90 .. -90 ... -30)"] value; # global-range AP
0.0170371 0.12941 0.224144 0.258819 0.224144 0.12941 0.0170371
% [nap "zone_wt{-80 -40 -10 10 40 80}"] value; # global-range
non-AP
0.0931011 0.20279 0.204109 0.204109 0.20279 0.0931011
% [nap "zone_wt(10 .. 60 ... 10)"] value; # local-range AP
0.209562 0.199962 0.184286 0.16301 0.136782 0.106398
\end{verbatim}

  \subsection{
    \label{land}Land, Sea and Coast
  }
The functions 
  \texttt{is\_land}, 
  \texttt{is\_coast} and 
  \texttt{fraction\_land} are based on the 
  \href{land-flag.html}{ \texttt{nap-land-flag} } command. This uses data which has an accuracy of 0.01$^{\circ}$.
  \par These three functions take an optional argument 
  $data\_dir$, which specifies the directory containing the 
  \texttt{nap\_land\_flag} data files. This argument is not needed
  when the standard data directory is used.
  \subsubsection{
    \label{is-land} \texttt{is\_land(}$latitude$\texttt{,} $longitude$[\texttt{,} $data\_dir$]\texttt{)}
  }
Produce a land/sea mask in the form of an 
  \texttt{i8} (8-bit signed integer) matrix with 1 for land and 0
  for sea.
  \par The arguments 
  $latitude$ and 
  $longitude$ can be scalars, vectors or matrices. If they are
  either both vectors, or one is a vector and the other is a scalar,
  then they are used as the coordinate variables of the matrix result.
  Otherwise the result has the same shape and coordinate variables as
  the one of higher rank.
  \par Note that each element of the result corresponds to just the
  single point defined by its latitude and longitude. It is often
  better to test multiple points, which can be done using function 
  \ref{fraction-land} \texttt{fraction-land(} $latitude$ \texttt{,} $longitude$[\texttt{,} $nlat$ \texttt{,} $nlon$ \texttt{,} $data-dir$] \texttt{)}.
  \par The following example produces a matrix with 
  \texttt{0} for sea and 
  \texttt{1} for land:
  \begin{verbatim}
% [nap "lats = 90 .. -90 ... -45"]
90 45 0 -45 -90
% [nap "lons = -180 .. 180 ... 45"] value
-180 -135 -90 -45 0 45 90 135 180
% [nap "isLand = is_land(lats, lons)"] value
0 0 0 0 0 0 0 0 0
0 0 1 0 1 1 1 1 0
0 0 0 0 0 0 0 0 0
0 0 0 0 0 0 0 0 0
1 1 1 1 1 1 1 1 1
\end{verbatim}

The following shows how this can be used to mask out (make
missing) the sea points in the matrix named 
  \texttt{area} defined in the the above example of 
  \ref{area-on-globe} \texttt{area-on-globe(} $latitude$ \texttt{,} $longitude$ \texttt{)}:
  \begin{verbatim}
% [nap "isLand ? area : _"] value -format %.4f
     _      _      _      _      _      _      _      _      _
     _      _ 0.0313      _ 0.0313 0.0313 0.0313 0.0313      _
     _      _      _      _      _      _      _      _      _
     _      _      _      _      _      _      _      _      _
0.0046 0.0092 0.0092 0.0092 0.0092 0.0092 0.0092 0.0092 0.0046
\end{verbatim}

The following masks out the land points:
  \begin{verbatim}
% [nap "isLand ? _ : area"] value -format %.4f
0.0046 0.0092 0.0092 0.0092 0.0092 0.0092 0.0092 0.0092 0.0046
0.0156 0.0313      _ 0.0313      _      _      _      _ 0.0156
0.0221 0.0442 0.0442 0.0442 0.0442 0.0442 0.0442 0.0442 0.0221
0.0156 0.0313 0.0313 0.0313 0.0313 0.0313 0.0313 0.0313 0.0156
     _      _      _      _      _      _      _      _      _
\end{verbatim}

The following alternative method defines a mask consisting of
zeros and missing values. Masking is done by adding this mask.
  \begin{verbatim}
% [nap "mask = {_ 0}(isLand)"] value
_ _ _ _ _ _ _ _ _
_ _ 0 _ 0 0 0 0 _
_ _ _ _ _ _ _ _ _
_ _ _ _ _ _ _ _ _
0 0 0 0 0 0 0 0 0
% [nap "area + mask"] value -format %.4f
     _      _      _      _      _      _      _      _      _
     _      _ 0.0313      _ 0.0313 0.0313 0.0313 0.0313      _
     _      _      _      _      _      _      _      _      _
     _      _      _      _      _      _      _      _      _
0.0046 0.0092 0.0092 0.0092 0.0092 0.0092 0.0092 0.0092 0.0046
\end{verbatim}

  \subsubsection{
    \label{is-coast} \texttt{is\_coast(}$latitude$\texttt{,} $longitude$[\texttt{,} $nlat$\texttt{,} $nlon$\texttt{,} $data\_dir$]\texttt{)}
  }
This function produces an 
  \texttt{i8} (8-bit signed integer) matrix with 1 for coast and 0
  otherwise.
  \par Cells are defined around each latitude and longitude using the
  functions 
  \ref{zone-bounds} \texttt{zone-bounds(} $latitude$ \texttt{)} and \ref{merid-bounds} \texttt{merid-bounds(} $longitude$ \texttt{)} . Each cell is divided into 
  $nlat$ rows and 
  $nlon$ columns. A cell is considered land if more than half
  its 
  $nlat$ � 
  $nlon$ points are classified as land by function 
  \texttt{is\_land}.
  \par The result is 1 if a cell is land and has at least one adjacent
  sea cell to the north, south, east or west. The coordinate variables
  of the result are copies of the vector arguments 
  $latitude$ and 
  $longitude$.
  \par The optional argument 
  $nlat$ defaults to 8.
  \\The optional argument 
  $nlon$ defaults to 
  $nlat$.
  \\If both these arguments are omitted then each cell contains 64
  points, which is enough for most purposes. (A single point is often
  adequate for coastline overlays on maps, etc.)
  \subsubsection{
    \label{fraction-land} \texttt{fraction\_land(}$latitude$\texttt{,} $longitude$[\texttt{,} $nlat$\texttt{,} $nlon$\texttt{,} $data-dir$]\texttt{)}
  }
This function produces an 
  \texttt{f32} (f32-bit floating-point) matrix containing the
  estimated proportion of land in each cell.
  \par Cells are defined around each latitude and longitude using the
  functions 
  \ref{zone-bounds} \texttt{zone-bounds(} $latitude$ \texttt{)} and 
  \ref{merid-bounds} \texttt{merid-bounds(} $longitude$ \texttt{)} . Each cell is divided into 
  $nlat$ rows and 
  $nlon$ columns. The result is defined by the proportion of
  the 
  $nlat$ � 
  $nlon$ points which are classified as land by function 
  \texttt{is\_land}.
  \par The coordinate variables of the result are copies of the vector
  arguments 
  $latitude$ and 
  $longitude$.
  \par The optional argument 
  $nlat$ defaults to 8.
  \\The optional argument 
  $nlon$ defaults to 
  $nlat$.
  \\If both these arguments are omitted then each cell contains 64
  points, which is enough for most purposes.
  \\The minimum resolution is 0.01$^{\circ}$ even if the specified 
  $nlat$ or 
  $nlon$ would give less. The value of 
  $nlat$ or 
  $nlon$ is automatically reduced if its specified value would
  give a resolution less than 0.01$^{\circ}$.
  \par The following example produces a similar (but much more accurate)
  land/sea mask to that produced in the above example for 
  \ref{is-land} \texttt{is-land(} $latitude$ \texttt{,} $longitude$[\texttt{,} $data-dir$] \texttt{)}:
  \begin{verbatim}
% [nap "lats = 90 .. -90 ... -45"]
90 45 0 -45 -90
% [nap "lons = -180 .. 180 ... 45"] value
-180 -135 -90 -45 0 45 90 135 180
% [nap "fraction_land(lats, lons)"] value -format %0.2f
0.05 0.14 0.28 0.55 0.08 0.17 0.34 0.19 0.11
0.06 0.39 0.67 0.13 0.52 0.88 1.00 0.50 0.17
0.00 0.00 0.22 0.38 0.47 0.45 0.17 0.23 0.00
0.00 0.00 0.09 0.13 0.03 0.05 0.00 0.25 0.02
0.25 0.59 0.69 0.47 0.81 0.95 0.95 0.98 0.47
% [nap "fraction_land(lats, lons) > 0.5f32"] value
0 0 0 1 0 0 0 0 0
0 0 1 0 1 1 1 0 0
0 0 0 0 0 0 0 0 0
0 0 0 0 0 0 0 0 0
0 1 1 0 1 1 1 1 0
\end{verbatim}

  \subsection{
    \label{Reading}Reading Geographic Files
  }

  \subsubsection{
    \label{acof2boxed} \texttt{acof2boxed(} $filename$ \texttt{)}
  }
Read ascii cof (.acof) coastline file and return boxed nao.
  \par Example:
  \\
  \texttt{nap "in = acof2boxed('abc.acof')"}
  \subsubsection{
    \label{get-gridascii} \texttt{get\_gridascii(}$filename$[\texttt{,} $unit$]\texttt{)}
  }
Read a file in ARC/INFO GRIDASCII format.
  \par The optional argument 
  $unit$ specifies the unit of x and y. If this argument is
  omitted then:
  \\Dimension 
  $x$ is longitude and has unit 
  \texttt{"degrees\_east"}.
  \\Dimension 
  $y$ is latitude and has unit 
  \texttt{"degrees\_north"}.
  \par Examples:
  \\
  \begin{verbatim}
% nap "in = get_gridascii('abc.gridascii')"; # x is
longitude, y is latitude
% nap "in = get_gridascii('abc.gridascii', 'metres')"
\end{verbatim}

