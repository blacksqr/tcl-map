%  $Id: data_type.tex,v 1.2 2006/03/01 00:57:58 dav480 Exp $ 
    % Data Types
    \section{Data Types}

  \subsection{
    \label{data-types}NAP Data-types
  }
A NAO can have any of the following data-types:
  \\ \par \begin{tabular}{|l|l|}
    \hline 
      \textbf{Name} & 
      \textbf{Description} 
    \\
    \hline 
    \hline 
        \texttt{c8}
       & 
      8-bit character 
    \\
    \hline 
        \texttt{i8}
       & 
      8-bit signed integer
    \\
    \hline 
        \texttt{i16}
       & 
      16-bit signed integer
    \\
    \hline 
        \texttt{i32}
       & 
      32-bit signed integer
    \\
    \hline 
        \texttt{u8}
       & 
      8-bit unsigned integer
    \\
    \hline 
        \texttt{u16}
       & 
      16-bit unsigned integer
    \\
    \hline 
        \texttt{u32}
       & 
      32-bit unsigned integer
    \\
    \hline 
        \texttt{f32}
       & 
      32-bit floating-point
    \\
    \hline 
        \texttt{f64}
       & 
      64-bit floating-point
    \\
    \hline 
        \texttt{ragged}
       & 
      compressed
    \\
    \hline 
        \texttt{boxed}
       & 
      slot numbers
    \\
  \hline
\end{tabular} \\ \par
  

The 
  \texttt{ragged} type provides an efficient way of storing arrays
  with many missing values around the edges.
  

A 
  \texttt{boxed} NAO contains slot numbers pointing to other NAOs.
  This allows one to construct arrays (normally vectors) of arrays.
  Boxed vectors are generated by the 
  \href{op.html\#Link}{Link Operators `\texttt{...}' and `\texttt{,}'}. Boxed vectors are used
  \begin{itemize}
    \item to pass multiple arguments to a function
    \item to return multiple results from a function
    \item in 
    \emph{cross-product indexing}
    \item to generate 
    \emph{arithmetic progressions} with step sizes other than 1 and
    -1
  \end{itemize}Boxed vectors can be unpacked using function 
  \texttt{open\_box(} 
  $x$ 
  \texttt{)}, which is described in 
  \href{function.html\#Special-Data-types}{Functions related to Special Data-types}.
  \subsection{
    \label{constant}Data-type of Constants
  }

  

A scalar constant can contain a data-type suffix, as in:
  \begin{verbatim}
% [nap "3.7f32"] all
::NAP::53-53  f32  MissingValue: NaN  References: 0
Value:
3.7
% [nap "123u8"] all
::NAP::55-55  u8  MissingValue: (NULL)  References: 0
Value:
123
\end{verbatim}

The following examples show that if there is no such suffix then
floating-point values are treated as 
  \texttt{f64} while integer values are treated as 
  \texttt{i32}:
  \begin{verbatim}
% [nap "3.7"] all
::NAP::57-57  f64  MissingValue: NaN  References: 0
Value:
3.7
% [nap "123"] all
::NAP::58-58  i32  MissingValue: -2147483648  References: 0
Value:
123
\end{verbatim}

  \subsection{
    \label{functions}Data-type Conversion Functions
  }
There is a function with the name of each data-type except 
  \texttt{ragged} and 
  \texttt{boxed}. Each such function converts its argument to that
  data-type. For example:
  \begin{verbatim}
% [nap "f64(123)"] all
::NAP::62-62  f64  MissingValue: NaN  References: 0
Value:
123
% [nap "c8(123)"] all; # ascii character 123
::NAP::66-66  c8  MissingValue: (NULL)  References: 0
Value:
{
% [nap "f32({123 -1.2 0})"] all; # vector with 3 elements
::NAP::70-70  f32  MissingValue: NaN  References: 0
Dimension 0   Size: 3      Name: (NULL)    Coordinate-variable: (NULL)
Value:
123 -1.2 0
\end{verbatim}

  

The parentheses 
  \texttt{()} in these three examples are not needed. Deleting
  them:
  \begin{verbatim}
% [nap "f64 123"] all
::NAP::26-26  f64  MissingValue: NaN  References: 0
Value:
123
% [nap "c8 123"] all
::NAP::29-29  c8  MissingValue: (NULL)  References: 0
Value:
{
% [nap "f32{123 -1.2 0}"] all
::NAP::33-33  f32  MissingValue: NaN  References: 0
Dimension 0   Size: 3      Name: (NULL)    Coordinate-variable: (NULL)
Value:
123 -1.2 0
\end{verbatim}

  \subsection{
    \label{result}Data-type of result of operation
  }

  

Many operations (defined by an operator or a function) produce a
  result whose data-type matches that of their operands/arguments (if
  these all have the same data-type). The following examples illustrate
  this for the 
  \emph{subtract} operator:
  \begin{verbatim}
% [nap "3 - 1"] all
::NAP::57-57  i32  MissingValue: -2147483648  References: 0  Unit:
(NULL)
Value:
2
% [nap "3f32 - 1f32"] all
::NAP::61-61  f32  MissingValue: NaN  References: 0  Unit: (NULL)
Value:
2
% [nap "3u8 - 1u8"] all
::NAP::65-65  u8  MissingValue: 255  References: 0  Unit: (NULL)
Value:
2
\end{verbatim}

  

What happens if the operands differ in data-type? Let's try
  adding 
  \texttt{f32} and 
  \texttt{f64} values:
  \begin{verbatim}
% [nap "234f32 + 3.5f64"] all
::NAP::40-40  f64  MissingValue: NaN  References: 0
Value:
237.5
\end{verbatim}

The result is of type 
  \texttt{f64}, so there is no loss of precision. Next let's
  try adding 
  \texttt{i32} and 
  \texttt{f32} values:
  \begin{verbatim}
% [nap "234 + 3.5f32"] all
::NAP::54-54  f64  MissingValue: NaN  References: 0  Unit: (NULL)
Value:
237.5
\end{verbatim}

  

Why is the result 
  \texttt{f64} rather than 
  \texttt{f32}? This prevents possible loss of precision, as
  in:
  \begin{verbatim}
% [nap "f32(123456789) + 5f32"] all -format %d
::NAP::48-48  f32  MissingValue: NaN  References: 0  Unit: (NULL)
Value:
123456800
\end{verbatim}

This is due to the fact that an 
  \texttt{i32} value has 31 bits (9.3 digits) of precision, whereas
  an 
  \texttt{f32} value has only 24 bits (7.2 digits) of precision. So
  both operands must be converted to 
  \texttt{f64} before the addition takes place. This is shown by:
  \begin{verbatim}
% [nap "123456789 + 5f32"] all -format %d
::NAP::43-43  f64  MissingValue: NaN  References: 0  Unit: (NULL)
Value:
123456794
\end{verbatim}

  

Some operations produce a result whose data-type is independent of
  the types of the operands. In particular, relational and logical
  operators always produce an 
  \texttt{i8} result with value 1 for true and 0 for false. The
  following illustrates the `\texttt{$>$}' operator:
  \begin{verbatim}
% [nap "9 > 8"] all
::NAP::43-43  i8  MissingValue: (NULL)  References: 0
Value:
1
\end{verbatim}

