%  $Id: choose_file.tex,v 1.6 2006/09/28 13:59:57 dav480 Exp $ 
    % choose\_file GUI

\section{choose\_file GUI}
    \label{choose-file}

  \subsection{Introduction}
    \label{choose-file-Introduction}

The 
  \texttt{choose\_file} GUI is used to select and open input files.
  It contains the following three lines:
\begin{bullets}
    \item 
    \emph{Filename} button and spinbox widget
    \item 
    \emph{Directory} button and entry widget
    \item 
    \emph{Glob Filter} entry widget
\end{bullets}
If you are already in the desired directory then the spinbox may
  be all you need. The 
  \emph{file selection dialog} GUI can be used to select both the
  directory and the file. The 
  \emph{directory (folder) selection dialog} GUI is useful when the
  directory contains many files matching the filter.
  
The 
  \texttt{choose\_file} GUI is defined by the file 
  \texttt{choose\_file.tcl}.

  \subsection{Instructions}
    \label{choose-file-Instructions}

  \begin{enumerate}
    \item If any field is too narrow then resize the window by dragging
    its edge.
    \item Select an input file using any of the following:
\begin{bullets}
      \item Type into any of the three entry fields.
      \item Press the 
      \texttt{Filename} button to use the 
      \emph{file selection dialog} GUI.
      \item Press the 
      \texttt{Directory} button to use the 
      \emph{directory (folder) selection dialog} GUI.
      \item Click on the spinbox arrows (or press the keyboard up/down
	  keys) to spin through the files matching the filter.
\end{bullets}
    \item Press the \texttt{Open} button to open the file. This can also be done by pressing the 
	  \texttt{Enter} (a.k.a.  \texttt{Return}) key in the \texttt{Filename} entry field.
  \end{enumerate}

\subsection{Usage}
\label{choose-file-Usage}

The following \texttt{choose\_file} command creates a new temporary GUI window,
accepts input from the user and
then returns (as the result) the pathname of the specified input file:
\\
\texttt{choose\_file} ?$\mathit{parent}$? ?$\mathit{filter}$? ?$\mathit{geometry}$?
\begin{simpleitems}
    \item $\mathit{parent}$: Parent window (default: \texttt{"."})
    \item $\mathit{filter}$: Initial glob filter (default: \texttt{"*"})
    \item $\mathit{geometry}$: Value as follows (default: \texttt{"NW"}):
	\begin{simpleitems}
	    \item \texttt{""}   : Pack in parent 
		(If parent is \texttt{""} then create toplevel anywhere)
	    \item \texttt{"NE"} : North-west corner of toplevel at north-east
		corner of parent (cannot be \texttt{""})
	    \item \texttt{"NW"} : North-west corner of toplevel at north-west
		corner of parent (cannot be \texttt{""})
	    \item \texttt{"SE"} : North-west corner of toplevel at south-east
		corner of parent (cannot be \texttt{""})
	    \item \texttt{"SW"} : North-west corner of toplevel at south-west
		corner of parent (cannot be \texttt{""})
	    \item other: Normal Tk geometry string.
		If parent is \texttt{""} then pack in it, else ignore parent.
	\end{simpleitems}
\end{simpleitems}
