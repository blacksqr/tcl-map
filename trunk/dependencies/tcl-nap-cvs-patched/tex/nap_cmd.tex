%  $Id: nap_cmd.tex,v 1.6 2006/06/06 08:09:42 dav480 Exp $ 
    % The Nap command

\section{The \texttt{nap} command}

\subsection{Introduction}

The standard Tcl command\  `\texttt{expr}' is based on C conventions for operators and
    functions. Nap expressions use similar conventions and can include
    any of the following tokens separated by white-space
    characters:
    \begin{itemize}
      \item Operands
      \begin{itemize}
        \item OOC-names
        \item names of Tcl variables (may include namespaces)
        \item constants
        \begin{itemize}
          \item numeric scalar
          \item numeric array
          \item string
        \end{itemize}
      \end{itemize}
      \item Operators (including assignment operator `\texttt{=}')
      \item Parentheses (`\texttt{()}')
      \item Function names
      \begin{itemize}
        \item built-in functions
        \item names (may include namespaces) of Tcl procedures defining
        Nap functions
      \end{itemize}
      \item Tcl substitution characters (`\texttt{[]\$}')
    \end{itemize}
    

If the 
    \texttt{nap} command has multiple arguments then these are
    concatenated. Thus it is not always necessary to enclose the
    expression by quote (\texttt{"}) characters, but this practice is recommended
    because it
\begin{bullets}
      \item prevents Tcl from removing braces (\texttt{\{\}})
      \item allows standard Tcl (\texttt{\$[]}) substitution within braces
      \item allows multi-line expressions.
\end{bullets}

\subsection{Substitution}

Like `\texttt{expr}', Nap does the Tcl substitution defined by
    any brackets and dollars (`\texttt{[]\$}') remaining after normal command parsing.
    

However, unlike `\texttt{expr}', Nap also substitutes for Tcl variable
    names that are not preceded by a `\texttt{\$}' (except where the name is the left operand of
    the assignment operator `\texttt{=}'). The value of the Tcl name is treated as a
    Nap expression, which is evaluated and the OOC-name of the result
    replaces the name. This substitution is repeated (up to eight
    times) until a single OOC-name is generated. The expressions in the
    following example include:
\begin{bullets}
      \item Tcl variable 
      \texttt{length} containing the string `\texttt{3.5}"
      \item Tcl variable 
      \texttt{breadth} defined by Nap to contain `\texttt{::NAP::13-13}"
      \item Nap constants 
      \texttt{2} and 
      \texttt{10}
      \item Tcl variable 
      \texttt{area} containing the string `\texttt{length * breadth}"
\end{bullets}
    \\
    \begin{verbatim}
% set length 3.5
3.5
% nap "breadth = 2"
::NAP::13-13
% [nap "2 * (length + breadth)"]
11
% set area "length * breadth"
length * breadth
% [nap "10 * area"]
70
\end{verbatim}

    

Each constant is replaced by the OOC-name of a NAO representing
    its value. After substitution, the expression consists of
    OOC-names, operators, function names and parentheses.

\subsection{Namespaces}

Nap allows names which include namespaces, as in:
    \begin{verbatim}
% namespace eval ::mySpace {}; # create namespace "mySpace"
% nap "::mySpace::x = 8"
::NAP::13-13
% [nap "3 + ::mySpace::x"]
11
\end{verbatim}

\subsection{Functions}

Function arguments can be enclosed by parentheses (as required
    by many other languages), but these parentheses are not required by
    the syntax. A name (which cannot be a Tcl variable name or it would
    have been substituted) followed by a OOC-name is treated as a
    function name. Thus the following two commands are equivalent:
    \begin{verbatim}
% [nap "sin(3.14)"]
0.00159265
% [nap "sin 3.14"]
0.00159265
\end{verbatim}

\subsection{Indexing}

Tcl array indices are enclosed by parentheses (`\texttt{()}'), while C uses brackets (`\texttt{[]}'). Nap requires neither, since indexing is
    simply implied by adjacent OOC-names. Thus the following two
    commands (which give elements 1, 0, 2 and 0 of the vector 
    \texttt{\{5 7 6\}}) are equivalent:
    \begin{verbatim}
% [nap "{5 7 6}({1 0 2 0})"]
7 5 6 5
% [nap "{5 7 6}{1 0 2 0}"]
7 5 6 5
\end{verbatim}

\subsection{Parenthesising Function Arguments and Indices}

As explained above, there is no syntactic need for parentheses
    around single function arguments and array indices. However, since
    most other computer languages do require such parentheses, it may
    aid human readability to include them.
