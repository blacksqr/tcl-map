%  $Id: hdf_netcdf.tex,v 1.2 2007/01/01 23:31:26 dav480 Exp $ 

\section{HDF and netCDF File Formats} 
    \label{hdf-netcdf}

  \href{http://www.hdfgroup.org/}{HDF} and 
  \href{http://www.unidata.ucar.edu/packages/netcdf/index.html}{netCDF}
  are similar array-oriented file formats.
Such files are popular in earth sciences such as meteorology and oceanography.
They contain data referenced by
symbol tables containing the names, data-types and dimensions of entities called
\emph{variables} in netCDF and \emph{scientific data sets (SDSs)} in HDF.
Each variable (SDS) can also have attributes such as a label, a
  format, a unit of measure and a missing-value. 

The design of Nap has been strongly influenced by HDF and netCDF.
Nap stores data in memory structures called \emph{n-dimensional array objects} (\emph{NAOs})
that have similar attributes to those of HDF SDSs and netCDF variables
(e.g. label, format, unit of measure, missing-value). 
Nap has powerful high-level facilities for HDF and netCDF I/O.

Nap does not currently support the new 
\href{http://www.hdfgroup.org/whatishdf5.html}{HDF5}
file format.
Note that
\href{http://www.unidata.ucar.edu/software/netcdf/netcdf-4/}{netCDF-4} software
supports this HDF5 file format (as well as the traditional netCDF format)
and is due for official release in late 2006.
It is planned to support HDF5 and netCDF-4 in a future version of Nap.
