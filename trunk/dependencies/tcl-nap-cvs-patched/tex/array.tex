%  $Id: array.tex,v 1.6 2006/09/20 08:48:49 dav480 Exp $ 
    % Arrays

\section{Arrays}
    \label{array}
    \label{array-Arrays}

\subsection{Introduction}
    \label{array-Introduction}

    Nap stores data in 
  \emph{NAOs}. A NAO is essentially an object based on the concept of
  an 
  \emph{n-dimensional array}. The number of dimensions is called the 
  \emph{rank}. A scalar is simply an array of rank 0.
  

Dimensions can have names and 
  \emph{coordinate variables (CVs)}. A CV maps a subscript to another
  variable such as a physical dimension (e.g. time). Arrays with CVs
  are called 
  \emph{grids}.
  

Nap provides powerful and efficient facilities for processing
  arrays, including:
\begin{bullets}
    \item 
    \emph{elemental} (element by element) extension of scalar
    arithmetic
    \item array operations such as sum, concatenation and reshaping
    \item indexing in a variety of ways
    \item searching (inverse indexing)
\end{bullets}

\subsection{Vectors}
\label{array-Vectors}

  An array with exactly one dimension is called a 
  \emph{vector}.
  
Nap array constants are enclosed in braces (`\texttt{\{\}}').
The following example doubles a vector containing four elements:
  \begin{verbatim}
% [nap "2 * {3.4 -0.1 _ 7}"]
6.8 -0.2 _ 14
\end{verbatim}

Note that an underscore (`\texttt{\_}') represents a missing element.
  

The ordinary scalar arithmetic operators and functions are applied
  to arrays in an elemental fashion, as shown by:
  \begin{verbatim}
% [nap "{1 _ 4 -6} + {2 0 -9 _}"]
3 _ -5 _
% [nap "sqrt(-{1 _ 9 -6} + 10)"]
3 _ 1 4
\end{verbatim}

The function 
  \texttt{sum} has the functionality of the mathematical
  `$\sum$'. When applied to a vector it returns the sum of
  the elements (ignoring any that are missing), as shown by:
  \begin{verbatim}
% [nap "sum{1 _ 9 -6 3}"]
7
\end{verbatim}

  

The following example applies the concatenation operator `\texttt{//}' to vectors and scalars:
  \begin{verbatim}
% [nap "{1 _ 9} // 7 // {-3 8}"]
1 _ 9 7 -3 8
\end{verbatim}

  

The operators `\texttt{..}' and `\texttt{...}' generate 
  \emph{arithmetic progression (AP)} vectors. For example:
  \begin{verbatim}
% [nap "3 .. 7"]
3 4 5 6 7
% [nap "7 .. 3"]
7 6 5 4 3
% [nap "0 .. 10 ... 2.5"]; # From 0 to 10 with step of 2.5
0 2.5 5 7.5 10
% [nap "5 ... 0 .. 10"]; # 5 elements from 0 to 10
0 2.5 5 7.5 10
\end{verbatim}

  

An 
  \emph{index} specifies a position within an array, commencing from
  0. If two operands (representing NAOs) are adjacent then Nap treats
  the right operand as the index of the left operand. For example:
  \begin{verbatim}
% nap "abc = {4 9 8 2}"
::NAP::283-283
% [nap "abc(2)"]
8
% [nap "abc 0"]; # parentheses not needed
4
% [nap "abc{2 0 3 2 1}"]; # vector index
8 4 2 8 9
\end{verbatim}

  

Some computer languages allow indexed variable names on the left
  of the assignment operator `\texttt{=}'. So one might expect to be able to modify
  element 2 of 
  \texttt{abc} as follows:
  \begin{verbatim}
% nap "abc(2) = 5"
abc(2) = 5
       ^
syntax error, unexpected '=', expecting $end
Error at line 1768 of file napParse.tab.c
\end{verbatim}

  

Nap does not allow indexed variable names on the left of the
  assignment operator `\texttt{=}'. However the `\texttt{set value}' OOC method does provide a way to modify
  elements of a NAO. Element 2 of 
  \texttt{abc} can be changed to 5 as follows:
  \begin{verbatim}
% $abc set value 5 2
% $abc
4 9 5 2
\end{verbatim}

  

The following examples show how several elements (or the entire
  array) can be changed using a single OOC:
  \begin{verbatim}
% $abc set value "{13 11}" "{0 2}"
% $abc
13 9 11 2
% $abc set value -3
% $abc
-3 -3 -3 -3
% $abc set value "{13 11}"
% $abc
13 11 13 11
\end{verbatim}

  

The default OOC displays only the first six elements, as in:
  \begin{verbatim}
% [nap "1 .. 9"]
1 2 3 4 5 6 ..
\end{verbatim}

  

The `\texttt{value}' OOC method displays all elements, as in:
  \begin{verbatim}
% [nap "1 .. 9"] value
1 2 3 4 5 6 7 8 9
\end{verbatim}

\subsection{Matrices}
    \label{array-Matrices}

An array with exactly two dimensions is called a \emph{matrix}.
A matrix constant has two levels of braces, as in:
  \begin{verbatim}
% [nap "{{2 4 8}{0 -1 9}}"] all
::NAP::298-298  i32  MissingValue: -2147483648  References: 0
Dimension 0   Size: 2      Name: (NULL)    Coordinate-variable: (NULL)
Dimension 1   Size: 3      Name: (NULL)    Coordinate-variable: (NULL)
Value:
 2  4  8
 0 -1  9
\end{verbatim}

It is often convenient to use a separate line for each row, as in:
  \begin{verbatim}
% nap "mat = {
{2 1.5 7}
{0  -2 1.1}
}"
::NAP::301-301
% $mat all
::NAP::301-301  f64  MissingValue: NaN  References: 1
Dimension 0   Size: 2      Name: (NULL)    Coordinate-variable: (NULL)
Dimension 1   Size: 3      Name: (NULL)    Coordinate-variable: (NULL)
Value:
 2.0  1.5  7.0
 0.0 -2.0  1.1
\end{verbatim}

  

The function `\texttt{sum}' gives column and row sums, as in:
  \begin{verbatim}
% [nap "sum(mat)"]
2 -0.5 8.1
% [nap "sum(mat, 1)"]
10.5 -0.9
\end{verbatim}

  

Matrices (and higher rank arrays) can be indexed using either 
  \emph{cross-product indexing} or 
  \emph{full indexing}. A cross-product index of a matrix has the
  form `$rows$, $columns$', where 
  $rows$ and 
  $columns$ are scalars or vectors. Here are two examples of
  cross-product indexing:
  \begin{verbatim}
% [nap "mat(0,2)"]
7
% [nap "mat(0 .. 1, {0 2})"]
2.0 7.0
0.0 1.1
\end{verbatim}

  

A full index of a matrix is an array with two columns
  corresponding to 
  \emph{row} and 
  \emph{column}. Here are two examples of full indexing:
  \begin{verbatim}
% [nap "mat{0 2}"]
7
% [nap "mat{{0 2}{0 0}{1 3}}"]
7 2 0
\end{verbatim}

  

The function `\texttt{transpose}' swaps the row and column dimensions.
  E.g.
  \begin{verbatim}
% $mat
 2.0  1.5  7.0
 0.0 -2.0  1.1
% [nap "transpose mat"]
 2.0  0.0
 1.5 -2.0
 7.0  1.1
\end{verbatim}

  

The function `\texttt{reshape}' uses the elements of the first argument
  (recycled if necessary) to produce an array whose shape is specified
  by the second argument. If there is only one argument then this is
  reshaped to a vector with the same number of elements. E.g.
  \begin{verbatim}
% [nap "reshape(mat, {3 4})"]
 2.0  1.5  7.0  0.0
-2.0  1.1  2.0  1.5
 7.0  0.0 -2.0  1.1
% [nap "reshape mat"]
2 1.5 7 0 -2 1.1
% [nap "reshape(9, {2 5})"]
9 9 9 9 9
9 9 9 9 9
\end{verbatim}

  

The concatenation operator `\texttt{//}' can be applied to matrices. For example:
  \begin{verbatim}
nap "m33 = {{4 6 8}{3 5 9}{0 4 1}}"
::NAP::14-14
% $m33
4 6 8
3 5 9
0 4 1
% $mat
 2.0  1.5  7.0
 0.0 -2.0  1.1
% [nap "m33 // mat"]
 4.0  6.0  8.0
 3.0  5.0  9.0
 0.0  4.0  1.0
 2.0  1.5  7.0
 0.0 -2.0  1.1
\end{verbatim}

  

The laminate operator `\texttt{///}' joins its operands over a new dimension. The
  following example creates a matrix from two vectors:
  \begin{verbatim}
% [nap "{9 3 5} /// {1 7 2}"]
9 3 5
1 7 2
% [nap "{9 3 5} // {1 7 2}"]; # Note difference
9 3 5 1 7 2
\end{verbatim}

\subsection{Higher Rank Arrays}
    \label{array-Higher}

A NAO can have up to 32 dimensions. The following example
  creates a 3-dimensional array, sums it along each dimension, then
  extracts an element using cross-product indexing.
  \begin{verbatim}
% nap "a3D = reshape(0 .. 99, {2 3 4})"
::NAP::47-47
% $a3D 
 0  1  2  3
 4  5  6  7
 8  9 10 11

12 13 14 15
16 17 18 19
20 21 22 23
% [nap "sum(a3D)"]
12 14 16 18
20 22 24 26
28 30 32 34
% [nap "sum(a3D, 1)"]
 6 22 38
54 70 86
% [nap "sum(a3D, 2)"]
12 15 18 21
48 51 54 57
% [nap "a3D(0,2,1)"]
9
\end{verbatim}

\subsection{Grids}
    \label{array-Grids}

Grids are arrays with 
  \emph{coordinate variables (CVs)}. Dimensions often correspond to
  physical dimensions such as time and length. A CV maps a subscript to
  another variable, which is typically a physical dimension such as
  time.
  

The following example creates a NAO corresponding to the grid
  described in section \ref{grid-Missing} (Missing Data).
Note how the `\texttt{set coo}' OOC method is used to attach the CVs and
  name the dimensions.
  \begin{verbatim}
% nap "latitude = {30 25 10 -10}"
::NAP::74-74
% nap "longitude = {30 40 60 65 75}"
::NAP::76-76
% nap "grid2d = {
{15 17 10  _ 21}
{16 14 18 18 19}
{ _ 11 12 11  _}
{10  9  _ 12 11}
}"
::NAP::78-78
% $grid2d set coo latitude longitude
% $grid2d all
::NAP::78-78  i32  MissingValue: -2147483648  References: 1
Dimension 0   Size: 4      Name: latitude  Coordinate-variable: ::NAP::74-74
Dimension 1   Size: 5      Name: longitude  Coordinate-variable: ::NAP::76-76
Value:
15 17 10  _ 21
16 14 18 18 19
 _ 11 12 11  _
10  9  _ 12 11
\end{verbatim}

Grids can be indexed using 
  \emph{indirect indexing}, which is indexing via CVs. Continuing the
  above example, let us extract the value corresponding to a latitude
  of 25 and a longitude of 75:
  \begin{verbatim}
% [nap "grid2d(@25, @75)"]; # indirect indexing
19
% [nap "grid2d(1, 4)"]; # direct indexing
19
\end{verbatim}

  

Nap allows fractional subscripts, which correspond to interpolated
  values. The following produces a value corresponding to row 0.5
  (27.5$^{\circ}$N) and column 1.5 (50$^{\circ}$E):
  \begin{verbatim}
% [nap "grid2d(0.5, 1.5)"]; # direct indexing
14.75
% [nap "grid2d(@27.5, @50)"]; # indirect indexing
14.75
% expr (17+10+14+18) * 0.25; # check arithmetic
14.75
\end{verbatim}

\subsection{Strings}
    \label{array-Strings}

String constants are enclosed in apostrophes `\texttt{''}' or grave accents `\texttt{``}' and generate character vectors. For example:
  \begin{verbatim}
% nap "message = 'hello world'"
::NAP::329-329
% $message all
::NAP::329-329  c8  MissingValue: (NULL)  References: 1
Dimension 0   Size: 11     Name: (NULL)    Coordinate-variable: (NULL)
Value:
hello world
% [nap "`don't`"]
don't
\end{verbatim}

  

The following example uses the library function `\texttt{gets\_matrix}' to
    read the text file `\texttt{abc.txt}' into a NAO referenced
    by `\texttt{in}'.
  \begin{verbatim}
% nap "in = gets_matrix('abc.txt')"
::NAP::349-349
% $in all
::NAP::349-349  f64  MissingValue: NaN  References: 1
Dimension 0   Size: 2      Name: (NULL)    Coordinate-variable: (NULL)
Dimension 1   Size: 3      Name: (NULL)    Coordinate-variable: (NULL)
Value:
 2  9 -3
 0 12  7
\end{verbatim}

