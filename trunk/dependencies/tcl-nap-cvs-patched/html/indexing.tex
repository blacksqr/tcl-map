%  $Id: indexing.tex,v 1.1 2006/03/01 05:59:17 dav480 Exp $ 
    % NAP Indexing
    \section{Indexing}

  \subsection{
    \label{Introduction}Introduction
  }

  


  \emph{Indexing} is the process of extracting elements from arrays.
  NAP extends this concept to the estimation (using interpolation) of
  values 
  \emph{between} the elements.
  

An index can appear:
  \begin{itemize}
    \item within a NAP expression
    \item as an argument of an OOC. E.g. method 
    \texttt{set\ value} takes an argument that specifies
    which elements are to be modified
    \item as an argument of commands 
    \texttt{nap\_get hdf} and 
    \texttt{nap\_get netcdf}, specifying positions within a
    file
  \end{itemize}
  

NAP provides powerful indexing (subscripting) facilities. The
  subscript origin is 0 (as in other aspects of Tcl such as lists). The
  rightmost dimension is the least significant (varies fastest). Here
  is a simple example of a vector indexed by a scalar:
  \begin{verbatim}
% nap "vector = {2 -5 9 4}"
::NAP::14-14
% [nap "vector(2)"]
9
\end{verbatim}

  \subsubsection{
    \label{Syntax}Indexing Syntax
  }
NAP syntax specifies that indexing is implied by two adjacent
  NAOs, with the base array on the left and the index on the right.
  Thus it is not necessary to parenthesise an index that is simply a
  constant or variable-name. However parentheses may make the code
  clearer to humans, who are likely to be familiar with languages where
  this is required.
  

This syntax means that the above example can be rewritten without
  parentheses as:
  \begin{verbatim}
% [nap "vector 2"]
9
\end{verbatim}

It also means that any non-scalar expression (including a
constant of course) can be indexed, as shown by:
  \begin{verbatim}
% [nap "{2 -5 9 4} 2"]
9
% [nap "({2 -5 9 4} + 10) 2"]
19
\end{verbatim}

  \subsubsection{
    \label{Dimension-Position}Dimension-Position
  }
A 
  \emph{dimension-position} is a scalar value defining the position
  along a dimension. Fractional values are valid and represent
  positions 
  \emph{between} the array elements. Values at non-integral positions
  are estimated using n-dimensional linear interpolation. The following
  demonstrates this (continuing the above example):
  \begin{verbatim}
% [nap "vector 2.5"]
6.5
\end{verbatim}

Note that the dimension-position 
  \texttt{2.5} is halfway between 
  \texttt{2} (corresponding to the value 
  \texttt{9}) and 
  \texttt{3} (corresponding to the value 
  \texttt{4}). Thus the value is estimated to be 
  \mbox{
    \texttt{0.5 * 9.0 + 0.5 * 4.0 = 4.5 + 2.0 = 6.5}
  } using ordinary one-dimensional linear interpolation.
  

If 
  $n$ is the dimension-size and 
  $p$ the position, then 
  \mbox{0 $\le$ 
  $p$ $<$ 
  $n$.} Values between 
  $n$-1 and 
  $n$ are defined by treating position 
  $n$ as equivalent to 0. This gives wraparound useful with
  cyclic dimensions such as longitude. Thus
  \begin{verbatim}
% [nap "vector 3.1"]
3.8
\end{verbatim}

Note that the dimension-position 
  \texttt{3.1} is 10\% of the distance between 
  \texttt{3} (corresponding to the value 
  \texttt{4}) and 
  \texttt{4} (equivalent to 
  \texttt{0} and corresponding to the value 
  \texttt{2}). Thus the value is estimated to be 
  \mbox{
    \texttt{0.9 * 4.0 + 0.1 * 2.0 = 3.6 + 0.2 = 3.8}
  }
  \subsubsection{
    \label{Subscript}Subscript
  }

  \emph{Dimension-positions} are always specified via 
  \emph{subscripts}. A 
  \emph{subscript} is similar to a 
  \emph{dimension-position} except that there are no size limits. If 
  $s$ is the subscript and 
  $n$ is the dimension-size, then the dimension-position 
  $p$ is defined by 
  \mbox{
  $s$ 
  \texttt{\%} 
  $n$,} the remainder after dividing 
  $s$ by 
  $n$.
  

Thus in our example subscript 
  \texttt{6} is treated as 
  \texttt{6\%4} = 
  \texttt{2}. So we get
  \begin{verbatim}
% [nap "vector 6"]
9
\end{verbatim}

It also means that negative values can be use to index backward
from the end, as shown by:
  \begin{verbatim}
% [nap "vector(-1)"]
4
% [nap "vector(-2)"]
9
% [nap "vector(-3)"]
-5
\end{verbatim}

  \subsubsection{
    \label{Elemental-Index}Elemental Index
  }
An 
  \emph{elemental index} is a vector of 
  $rank$ subscripts, specifying the subscripts of an element
  of an array. The following example creates a matrix 
  \texttt{mat} and illustrates the use of elemental indices to
  extract individual elements.
  \begin{verbatim}
% nap "mat = {{1.5 0 7}{2 -4 -9}}"
::NAP::60-60
% $mat
 1.5  0.0  7.0
 2.0 -4.0 -9.0
% [nap "mat {0 1}"]
0
% [nap "mat {1 -1}"]
-9
% 
% [nap "mat {0.5 1.5}"]
-1.5
\end{verbatim}

The value corresponding to the index 
  \texttt{\{0.5\ 1.5\}} is estimated, using bilinear
  interpolation, to be 
  \mbox{
    \texttt{0.25\ *\ 0.0\ +\ 0.25\ *\ 7.0\ +\ 0.25\ *\ (-4.0)\ +\ 0.25\ *\ (-9.0)
    = -1.5}
  }
  \subsection{
    \label{Index}Index
  }
An 
  \emph{index} is an array defining one or more elemental indices.
  The following table lists the four types, which are explained in the
  sections below:
  \\ \par \begin{tabular}{|l|l|}
    \hline 
      \textbf{Index Type} & 
      \textbf{Rank of Indexed Array}
    \\
    \hline 
      shape-preserving & 
      1
    \\
    \hline 
      vector-flip & 
      1
    \\
    \hline 
      full & 
      2 or more
    \\
    \hline 
      cross-product & 
      2 or more
    \\
  \hline
\end{tabular} \\ \par
  \subsubsection{
    \label{Shape-Preserving}Shape-Preserving
  }

  \emph{Shape-preserving} indexing is used to index a vector. The
  shape of the result is the same as that of the index. The following
  example shows how the previously defined variable 
  \texttt{vector} can be indexed by
  \begin{itemize}
    \item a scalar to produce a scalar
    \item a vector to produce a vector
    \item a matrix to produce a matrix:
  \end{itemize}
  \begin{verbatim}
% $vector
2 -5 9 4
% [nap "vector(2)"] all
::NAP::57-57  i32  MissingValue: -2147483648  References: 0  Unit:
(NULL)
Value:
9
% [nap "vector({2 2.5 2})"] all
::NAP::61-61  f32  MissingValue: NaN  References: 0  Unit: (NULL)
Dimension 0   Size: 3      Name: (NULL)    Coordinate-variable: (NULL)
Value:
9 6.5 9
% [nap "vector({
{1 0 2.5}
{-1 2 1}
})"] all

::NAP::67-67  f32  MissingValue: NaN  References: 0  Unit: (NULL)
Dimension 0   Size: 2      Name: (NULL)    Coordinate-variable: (NULL)
Dimension 1   Size: 3      Name: (NULL)    Coordinate-variable: (NULL)
Value:
-5.0  2.0  6.5
 4.0  9.0 -5.0
\end{verbatim}

The 
  \emph{shape-preserving} property means one can use a vector to
  define a mapping. The following example maps 0 to 4, 1 to 1, 2 to 9
  and 3 to 4:
  \begin{verbatim}
% [nap "{4 1 9 4} {
{2 1 2 0}
{3 3 0 1}
}"]
9 1 9 4
4 4 4 1
\end{verbatim}

The following example uses the same technique to implement a
simple substitution cipher (mapping space to R, A to X, B to B, C to T,
$\ldots$ as shown) to encrypt the message `HELLO\ WORLD'
as `A\ HHVREVZHC' which is then decrypted.
  \begin{verbatim}
% nap "plain   = ' ABCDEFGHIJKLMNOPQRSTUVWXYZ'"
::NAP::63-63
% nap "cipher  = 'RXBTC MUAFGWHYIVJKZDLNOEPQS'"
::NAP::64-64
% [nap "plain((plain @@ cipher)(plain @@ 'HELLO
WORLD'))"]; # encrypt
A HHVREVZHC
% [nap "cipher((cipher @@ plain)(cipher @@ 'A
HHVREVZHC'))"]; # decrypt
HELLO WORLD
\end{verbatim}

  \subsubsection{
    \label{Vector-flip}Vector-Flip
  }

  

It is often necessary to reverse the order of elements in a
  vector. One could use 
  \emph{shape-preserving} indexing, as in:
  \begin{verbatim}
% [nap "{2 4 6 8}(3 .. 0)"]
8 6 4 2
\end{verbatim}

  

NAP provides the 
  \emph{niladic} operator `\texttt{-}' to specify such reversal (or 
  \emph{flipping}). (A 
  \emph{niladic} operator is one without any operands.) Thus one can
  simplify the above example to:
  \begin{verbatim}
% [nap "{2 4 6 8}(-)"]
8 6 4 2
\end{verbatim}

  

Such an index of a vector, consisting of just `\texttt{-}', is called a 
  \emph{vector-flip}. Note that 
  \ref{Cross-product-index}cross-product-indexing also allows
  the niladic `\texttt{-}' to specify flipping of one or more
  dimensions.
  

What does the niladic `\texttt{-}' generate? Let's see:
  \begin{verbatim}
% [nap "-"] all
::NAP::62-62  f32  MissingValue: NaN  References: 0  Unit: (NULL)
Value:
-Inf
\end{verbatim}

It generates a scalar 32-bit NAO with the value 
  \emph{negative infinity}! Indexing treats such a NAO as meaning
  `flip$\backslash$\&amp;quot. So the above indexing example could also (but
  less conveniently) be written as:
  \begin{verbatim}
% [nap "{2 4 6 8}(-1if32)"]
8 6 4 2
\end{verbatim}

  \subsubsection{
    \label{Full-index}Full-index
  }
A 
  \emph{full-index} is an array specifying a separate elemental index
  for every element of the result. The shape of the index is the shape
  of the result with 
  $r$ (the rank of the indexed array) appended. Each row of
  the index contains a vector of 
  $r$ elements defining an elemental index.
  

The following example shows how the previously defined variable 
  \texttt{mat} can be indexed by
  \begin{itemize}
    \item a vector to produce a scalar
    \item a matrix to produce a vector
  \end{itemize}
  \begin{verbatim}
% $mat
 1.5  0.0  7.0
 2.0 -4.0 -9.0
% [nap "mat {0.5 1.5}"] all
::NAP::148-148  f32  MissingValue: NaN  References: 0  Unit: (NULL)
Value:
-1.5
% [nap "mat {
{0.5 1.5}
{0 1}
{-1 -1}
}"] all
::NAP::157-157  f32  MissingValue: NaN  References: 0  Unit: (NULL)
Dimension 0   Size: 3      Name: (NULL)    Coordinate-variable: (NULL)
Value:
-1.5 0 -9
\end{verbatim}

Note that 
  \emph{shape-preserving} indexing is similar to applying 
  \emph{full} indexing to a vector (if this were allowed). The
  shape-preserving-index is the hypothetical full-index reshaped to
  omit the final redundant dimension of size 1.
  \subsubsection{
    \label{Cross-product-index}Cross-product-index
  }
A 
  \emph{cross-product-index} is a boxed vector containing 
  \emph{rank} elements pointing to scalars, vectors, nulls and 
  \ref{Vector-flip}flips. The cross-product combination of
  this vector defines the elemental indices of the indexed array.
  

A cross-product-index is usually defined using the operator `\texttt{,}$\backslash$\&amp;quot.
    This allows the left and/or right operand to
  be omitted and such 
  \emph{null} (missing) operands are treated as 
  \mbox{`\texttt{0..(}$n$\texttt{-1)}',} where 
  $n$ is the dimension-size. Scalar operands produce no
  corresponding dimension in the result. A flip (dimension reversal) is
  normally represented by the niladic `\texttt{-}' operator,
    which is equivalent to 
  \mbox{`\texttt{(}$n$\texttt{-1)..0}'.}
  

The following examples again use the previously defined variable 
  \texttt{mat}. We begin by repeating the first 
  \emph{full-index} example above and then we provide the 
  \emph{cross-product-index} equivalent:
  \begin{verbatim}
% $mat
 1.5  0.0  7.0
 2.0 -4.0 -9.0
% [nap "mat({0.5 1.5})"] all
::NAP::196-196  f32  MissingValue: NaN  References: 0  Unit: (NULL)
Value:
-1.5
% [nap "mat(0.5,1.5)"] all
::NAP::204-204  f32  MissingValue: NaN  References: 0  Unit: (NULL)
Value:
-1.5
\end{verbatim}

The next example shows how the previously defined variable 
  \texttt{mat} can be indexed by the cross-product of two vectors
  to produce a matrix, then provides the equivalent 
  \emph{full-index}:
  \begin{verbatim}
% $mat
 1.5  0.0  7.0
 2.0 -4.0 -9.0
% [nap "mat({1 0},{2 0 -1 0})"] all
::NAP::174-174  f64  MissingValue: NaN  References: 0  Unit: (NULL)
Dimension 0   Size: 2      Name: (NULL)    Coordinate-variable: (NULL)
Dimension 1   Size: 4      Name: (NULL)    Coordinate-variable: (NULL)
Value:
-9.0  2.0 -9.0  2.0
 7.0  1.5  7.0  1.5
% [nap "mat({
{{1 2}{1 0}{1 -1}{1 0}}
{{0 2}{2 0}{2 -1}{2 0}}
})"] all
::NAP::180-180  f64  MissingValue: NaN  References: 0  Unit: (NULL)
Dimension 0   Size: 2      Name: (NULL)    Coordinate-variable: (NULL)
Dimension 1   Size: 4      Name: (NULL)    Coordinate-variable: (NULL)
Value:
-9.0  2.0 -9.0  2.0
 7.0  1.5  7.0  1.5
\end{verbatim}

The following example illustrates the effect of a null operand to
`\texttt{,}$\backslash$\&amp;quot. It also shows the difference between a
  scalar operand and a single-element vector containing the same value.
  \begin{verbatim}
% $mat
 1.5  0.0  7.0
 2.0 -4.0 -9.0
% [nap "mat(1,)"] all
::NAP::209-209  f64  MissingValue: NaN  References: 0  Unit: (NULL)
Dimension 0   Size: 3      Name: (NULL)    Coordinate-variable: (NULL)
Value:
2 -4 -9
% [nap "mat({1},)"] all
::NAP::213-213  f64  MissingValue: NaN  References: 0  Unit: (NULL)
Dimension 0   Size: 1      Name: (NULL)    Coordinate-variable: (NULL)
Dimension 1   Size: 3      Name: (NULL)    Coordinate-variable: (NULL)
Value:
 2 -4 -9
\end{verbatim}

The following examples show how the niladic `\texttt{-}' operator is used to flip (reverse) dimensions:
  \begin{verbatim}
% $mat
 1.5  0.0  7.0
 2.0 -4.0 -9.0
% [nap "mat(,-)"]
 7.0  0.0  1.5
-9.0 -4.0  2.0
% [nap "mat(-,)"]
 2.0 -4.0 -9.0
 1.5  0.0  7.0
% [nap "mat(-,-)"]
-9.0 -4.0  2.0
 7.0  0.0  1.5
% [nap "mat(0,-)"]
7 0 1.5
% [nap "mat(-,{2 0 0})"]
-9.0  2.0  2.0
 7.0  1.5  1.5
\end{verbatim}

The following example creates a rank-3 array 
  \texttt{a3d} with shape 
  \texttt{\{2\ 2\ 3\}}, then extracts all of row 0 from both
  layers:
  \begin{verbatim}
% nap "a3d = {
{
{9 1 4}
{0 8 7}
}{
{2 3 5}
{9 6 0}
}
}"
::NAP::215-215
% $a3d all
::NAP::215-215  i32  MissingValue: -2147483648  References: 1  Unit:
(NULL)
Dimension 0   Size: 2      Name: (NULL)    Coordinate-variable: (NULL)
Dimension 1   Size: 2      Name: (NULL)    Coordinate-variable: (NULL)
Dimension 2   Size: 3      Name: (NULL)    Coordinate-variable: (NULL)
Value:
9 1 4
0 8 7

2 3 5
9 6 0
% [nap "a3d(,0,)"] all
::NAP::220-220  i32  MissingValue: -2147483648  References: 0  Unit:
(NULL)
Dimension 0   Size: 2      Name: (NULL)    Coordinate-variable: (NULL)
Dimension 1   Size: 3      Name: (NULL)    Coordinate-variable: (NULL)
Value:
9 1 4
2 3 5
\end{verbatim}

  \subsection{
    \label{indirect-indexing}Indirect Indexing and Unary Operators `\texttt{@}' and `\texttt{@@}'
  }

  \emph{Indirect indexing} is indexing via coordinate variables. An
  example is specifying latitudes and longitudes rather than rows and
  columns. One can convert a latitude of say 20$^{\circ}$S to a subscript using
  `\texttt{lat@-20}', where 
  \texttt{lat} is the latitude coordinate variable. This uses the 
  \emph{binary} interpolated subscript `\texttt{@}' operator discussed in section
  \ref{Interpolated-Subscript}.
  Within an index expression it is
  possible to use the 
  \emph{unary} `\texttt{@}' operator and simply write
  `\texttt{@-20}'. The omitted left operand defaults to the
  corresponding coordinate variable.
  

There are three such unary operators for indirect indexing: `\texttt{@}',
  `\texttt{@@}' and `\texttt{@@@}'.
  However `\texttt{@@@}' is seldom used and will not be mentioned
  further.
  

There were restrictions on the use of these unary operators prior
  to version 5 of NAP. Each subscript had to have the form `\texttt{@} 
  $expr$' or `\texttt{@@} 
  $expr$'. These restrictions have now been lifted, as
  demonstrated at the end of the following 
  \ref{indirect-1d}1D Time-Series Example.
  

However these unary operators can no longer be used with 
  \emph{full indexing}. This facility was seldom if ever used.
  

The unary `\texttt{@}' and `\texttt{@@}'
  operators previously worked by simply creating
  a copy of their operand and attaching to it (via the 
\emph{link slot} discussed in section \ref{nao})
an ancillary NAO containing an integer with the value
  \\$\bullet$ 1 for indirect indexing using `\texttt{@}'
  \\$\bullet$ 2 for indirect indexing using `\texttt{@@}'.
  

The function 
  \texttt{invert\_grid()} still produces an indirect full index with
  such an attached NAO and the indexing code still handles such
  indices. However such NAOs are no longer produced in any other way
  and can be considered an anachronistic kludge.
  

The new version of NAP is based on a parser which produces a 
  \emph{parse tree} and then executes this tree. This allows cleverer
  execution of various things such as indexing, during which the
  coordinate variables are visible and can be used by the unary `\texttt{@}'
  and `\texttt{@@}' operators.
  

The operators `\texttt{@}' and `\texttt{@@}'
  are often used with indexing to interpolate to
  a finer or coarser grid. The operand of `\texttt{@}'
  and `\texttt{@@}' is normally the desired coordinate variable of
  the result. In many case this would happen using normal NAP
  processes, but there are some situations where this would not be the
  case. One example is producing a finer grid using `\texttt{@@}' to produce 
  \emph{nearest-neighbour} values. Another is using `\texttt{@}'
  with longitude wrap-around.
  

NAP ensures that the right-hand operand is used as the coordinate
  variable by attaching it to the result of `\texttt{@}'
  and `\texttt{@@}' using the 
  \href{nao.html}{ \emph{link slot} } . This is discussed further in 
  \href{op.html\#Inverse-Indexing}{Inverse Indexing Operators `\texttt{@}', `\texttt{@@}' 
and `\texttt{@@@}'}. The 
  \ref{indirect-2d}2D Geographic Example shows how `\texttt{@@}' produces 
  \emph{nearest-neighbour} values and such coordinate variables.
  \subsubsection{
    \label{indirect-1d}1D Time-Series Example
  }

  

Suppose we have temperatures at two-hourly intervals from time
  10:00 to 16:00 as follows:
  \begin{verbatim}
% nap "time = 10 .. 16 ... 2"
::NAP::20-20
% nap "temperature = {20.2 21.6 24.9 22.7}"
::NAP::21-21
% $temperature set coord time
\end{verbatim}

We could estimate temperatures every hour during this period
using either the binary or unary `\texttt{@}' as follows:
  \begin{verbatim}
% [nap "temperature(time @ (10 .. 16))"] value; # Use binary
@
20.2 20.9 21.6 23.25 24.9 23.8 22.7
% [nap "temperature(@ (10 .. 16))"] value; # Use unary @
20.2 20.9 21.6 23.25 24.9 23.8 22.7
\end{verbatim}

  

These unary operators can only be used within index expressions.
  Let's see what happens if we use one elsewhere:
  \begin{verbatim}
% nap "@10"
Nap_Indirect: Illegal coordinate variable corresponding to unary
'@'
Error at line 583 of file
/cygdrive/c/dav480/tcl/nap/generic/napMonad.c,m4
expr1: Error with unary operator '@'
Error at line 52 of file c:/dav480/tcl/tcl-nap/generic/eval_tree.c
\end{verbatim}

Note that this unary `\texttt{@}'
operator makes no sense because it is not within
  an index and thus there is no corresponding coordinate variable.
  

The following examples illustrate useful index expressions which
  now work but did not work prior to version 5 of NAP:
  \begin{verbatim}
% [nap "temperature(@@11.5 + 2)"]; # Example 1
22.7
% [nap "temperature(@@11.5 .. @@16.5)"]; # Example 2
21.6 24.9 22.7
% [nap "temperature(>@10.5 .. <@15.5)"]; # Example 3
21.6 24.9
\end{verbatim}

Example 1 gives the second temperature after that closest to time
11:30. Example 2 gives all the temperatures from that closest to time
11:30 to that closest to time 16:30. Example 3 gives all the
temperatures from the first following time 11:30 to the first before
time 15:30.
  \subsubsection{
    \label{indirect-2d}2D Geographic Example
  }

  

The following creates a 3�4 matrix 
  \texttt{temperature}, which has
  \begin{itemize}
    \item unit of 
    \texttt{degC} ($^{\circ}$C).
    \item rows corresponding to latitudes 10$^{\circ}$N, 20$^{\circ}$N and 30$^{\circ}$N
    \item columns corresponding to longitudes 110$^{\circ}$E, 120$^{\circ}$E, 130$^{\circ}$E and
    140$^{\circ}$E
  \end{itemize}
  \begin{verbatim}
% nap "temperature = f32{
{31.5 37.2 32.9 34.0}
{25.1 25.2 29.0 21.9}
{20.5 21.2 21.0 19.9}
}"
::NAP::72-72
% $temperature set unit degC
% nap "latitude = f32{10 20 30}"
::NAP::76-76
% $latitude set unit degrees_north
% nap "longitude = f32(110 .. 140 ... 10)"
::NAP::86-86
% $longitude set unit degrees_east
% $temperature set coo latitude longitude
\end{verbatim}

The following verifies that the main NAO and its coordinate
variables are as expected:
  \begin{verbatim}
% $temperature all
::NAP::72-72  f32  MissingValue: NaN  References: 1  Unit: degC
Dimension 0   Size: 3      Name: latitude  Coordinate-variable:
::NAP::76-76
Dimension 1   Size: 4      Name: longitude  Coordinate-variable:
::NAP::86-86
Value:
31.5 37.2 32.9 34.0
25.1 25.2 29.0 21.9
20.5 21.2 21.0 19.9
% [$temperature coo 0] all
::NAP::76-76  f32  MissingValue: NaN  References: 2  Unit:
degrees_north
Dimension 0   Size: 3      Name: (NULL)    Coordinate-variable: (NULL)
Value:
10 20 30
% [$temperature coo 1] all
::NAP::86-86  f32  MissingValue: NaN  References: 2  Unit: degrees_east
Dimension 0   Size: 4      Name: (NULL)    Coordinate-variable: (NULL)
Value:
110 120 130 140
\end{verbatim}

The following illustrates the use of both direct and indirect
indexing to display the value of 
  \texttt{29} in row 1 and column 2:
  \begin{verbatim}
% [nap "temperature(1,2)"]
29
% [nap "temperature(@20, @130)"]; # latitude=20 longitude=130
29
% [nap "temperature(@@20, @@130)"]
29
% [nap "temperature(1, @130)"]
29
\end{verbatim}

In this case there is a point exactly corresponding to 20$^{\circ}$S,
130$^{\circ}$E, so the operators 
  \texttt{@} and 
  \texttt{@@} give the same result. Let us try the point 21$^{\circ}$S,
  138$^{\circ}$E, which is not a grid point:
  \begin{verbatim}
% [nap "temperature(@21, @138)"]
23
% [nap "temperature(@@21, @@138)"]
21.9
\end{verbatim}

Now we get different results for the two operators. Operator 
  \texttt{@} gives a value estimated using bilinear interpolation.
  Operator 
  \texttt{@@} gives the data value at the nearest row (1) and
  column (3).
  

If the unary operators 
  \texttt{@} and 
  \texttt{@@} did not exist we would have to use their binary
  equivalents as follows:
  \begin{verbatim}
% nap "interpolated_row = coordinate_variable(temperature,0) @ 21"
::NAP::96-96
% $interpolated_row
1.1
% nap "interpolated_col = coordinate_variable(temperature,1) @ 138"
::NAP::103-103
% $interpolated_col
2.8
% [nap "temperature(interpolated_row, interpolated_col)"]
23
% nap "nearest_row = coordinate_variable(temperature,0) @@ 21"
::NAP::112-112
% $nearest_row
1
% nap "nearest_col = coordinate_variable(temperature,1) @@ 138"
::NAP::119-119
% $nearest_col
3
% [nap "temperature(nearest_row, nearest_col)"]
21.9
\end{verbatim}

Say we want to estimate temperatures on a grid with
  \begin{itemize}
    \item latitudes 19$^{\circ}$N, 20$^{\circ}$N and 21$^{\circ}$N
    \item longitudes 121$^{\circ}$E, 122$^{\circ}$E 123$^{\circ}$E and 124$^{\circ}$E
    \\Naming the new matrix 
    \texttt{region\_temperature}, this can be done as follows:
  \end{itemize}
  \begin{verbatim}
% nap "region_temperature = temperature(@(19 .. 21), @(121 .. 124))"
::NAP::147-147
% $region_temperature all
::NAP::147-147  f32  MissingValue: NaN  References: 1  Unit: degC
Dimension 0   Size: 3      Name: latitude  Coordinate-variable:
::NAP::145-145
Dimension 1   Size: 4      Name: longitude  Coordinate-variable:
::NAP::146-146
Value:
26.699 26.998 27.297 27.596
25.580 25.960 26.340 26.720
25.140 25.480 25.820 26.160
% ::NAP::145-145 all
::NAP::145-145  i32  MissingValue: -2147483648  References: 1  Unit:
degrees_north
Dimension 0   Size: 3      Name: (NULL)    Coordinate-variable: (NULL)
Value:
19 20 21
% ::NAP::146-146 all
::NAP::146-146  i32  MissingValue: -2147483648  References: 1  Unit:
degrees_east
Dimension 0   Size: 4      Name: (NULL)    Coordinate-variable: (NULL)
Value:
121 122 123 124
\end{verbatim}

Why has the new longitude coordinate-variable been converted to
data-type 
  \texttt{f32}? NAP recognises 
  \texttt{degrees\_east} as a special unit implying longitude
  characteristics such as
  \begin{itemize}
    \item wrap around to allow interpolation across longitude 180
    \item data-type 
    \texttt{f32}
  \end{itemize}
  

The above use of `\texttt{@}' produces 
  \emph{interpolated} values. The following illustrates the use of
  `\texttt{@@}' to produce 
  \emph{nearest-neighbour} values. Note the use of the original
  operands as the final coordinate variables.
  \begin{verbatim}
% nap "nearest_temperature = temperature(@@(14 .. 16), @@(123 .. 127))"
::NAP::222-222
% $nearest_temperature all
::NAP::222-222  f32  MissingValue: NaN  References: 1  Unit: degC
Dimension 0   Size: 3      Name: latitude  Coordinate-variable:
::NAP::212-212
Dimension 1   Size: 5      Name: longitude  Coordinate-variable:
::NAP::217-217
Value:
37.2 37.2 32.9 32.9 32.9
25.2 25.2 29.0 29.0 29.0
25.2 25.2 29.0 29.0 29.0
% [$nearest_temperature coo 0]
14 15 16
% [$nearest_temperature coo 1]
123 124 125 126 127
\end{verbatim}

