%  $Id: caps_nap_menu.tex,v 1.8 2006/09/28 14:01:36 dav480 Exp $ 
    % Caps/Nap Menu

\section{Caps/Nap Menu}
    \label{caps-nap-menu}

\subsection{Introduction}
\label{caps-nap-menu-Introduction}

The distributed 
  \texttt{wish} startup file `\texttt{wishrc.tcl}'
    (which is copied to file `\texttt{.wishrc}' on unix systems)
    ends with the command
  `\texttt{caps\_nap\_menu}', which displays a GUI menu
  consisting of three buttons labelled 
  \emph{Browse}, 
  \emph{Command} and 
  \emph{Help}. This command `\texttt{caps\_nap\_menu}' is defined by the file
    `\texttt{caps\_nap\_menu.tcl}'. Each of these three buttons is
  used to display a sub-menu detailed in the following sections.

\subsection{Browse}
    \label{caps-nap-menu-Browse}

The 
  \emph{Browse} menu provides access to browsers for Tcl variables
  and a variety of files. These file browsers typically provide
  facilities to:
\begin{bullets}
    \item Select a file using the \texttt{choose\_file} GUI
	(see section \ref{choose-file}).
    \item Select data from this file.
    \item Display this data as text.
    \item Plot this data (as an XY graph or a 2D image) using 
    \texttt{plot\_nao} (see section \ref{plot-nao}).
    \item Print this data.
    \item Convert this data to a NAO.
\end{bullets}
Detailed help is available within each browser.

\subsubsection{Tcl Variables} 
\label{caps-nap-menu-Variables}

    This facility for browsing 
    \emph{Tcl variables} is defined by the file 
    \texttt{browse\_var.tcl}. It can be used to display the names
    and values of all tcl variables (including arrays), but has
    particular facilities for those referencing NAOs. The menu buttons
    have the following functions:
    \begin{description}
      \item[\emph{namespace}]
      Use tree widget to set the namespace.
      \item[\emph{list}]
      List tcl variables in the namespace matching the glob
      pattern. Display a line for each matching tcl variable. This line
      contains the variable's name and value provided there is
      room. Otherwise the line is truncated. You can click on a line to
      display:
    \begin{bullets}
        \item full value of an ordinary tcl variable or array whose line is truncated as above
        \item a NAO as either text or graph/image as specified by the radio-button
    \end{bullets}
      \item[\emph{help}] Display \emph{Help on Tcl Variable Browser}.
      \item[\emph{cancel}] Remove \emph{Tcl Variable Browser} widget.
    \end{description}

\subsubsection{\emph{AVHRR} and \emph{ATSR} Satellite Files}
\label{caps-nap-menu-AVHRR-ATSR}
	This item is only available if the Caps package is loaded.

\subsubsection{\emph{CIF} Files}
\label{caps-nap-menu-CIF}

    CIF files originated at Melbourne University. Their use has
    been largely replaced by netCDF but some data still exists in this
    format. This facility for browsing 
    \emph{CIF} files is defined by the file 
    \texttt{browse\_cif.tcl}.

\subsubsection{\emph{HDF} and \emph{netCDF} Files}
\label{caps-nap-menu-HDF-netCDF}

\paragraph{Introduction \\}
\label{caps-nap-menu-HDF-netCDF-intro}

HDF and netCDF are two similar common file formats used for
data in the form of arrays and grids, especially in the earth
sciences such as Meteorology and Oceanography.
See section \ref{hdf-netcdf} for further information about HDF 
and netCDF.

The term \emph{variable/SDS} is used below.
This means \emph{variable} for netCDF and \emph{SDS} for HDF.

The HDF/netCDF browser is a powerful tool which allows one to
select a variable/SDS or attribute from a displayed tree, then (for
variable/SDS) select a region within it.

The HDF/netCDF browser is defined by the file \texttt{hdf.tcl}.

\paragraph{Instructions}

%  $Id: hdf.tex,v 1.6 2006/09/28 14:00:47 dav480 Exp $ 
    % HDF/netCDF Browser

  \begin{enumerate}
    \item Use the \texttt{choose-file} GUI
    to open an input file.
    Instructions can be displayed by
    pressing this GUI's own 
    \texttt{help} button. Opening the file should result in the
    display of a 
    \emph{file structure tree}.
    \item Use this tree as follows to select either a variable/SDS or an
    attribute. (The default selection is a variable/SDS with the
    maximum number of elements.)
\begin{bullets}
      \item Click on a variable/SDS to select it and display the spatial sampling widget.
      \item Click on a ' \texttt{+}' to display attribute names.
      \item Click on an attribute to select it and display its value.
\end{bullets}
    \item The spatial sampling widget allows you to select part of a
    variable/SDS. (The entire variable/SDS is selected by default.)
    \\Each dimension is represented by a row containing one or two
    lines. The first line represents subscript values. If a coordinate
    variable exists then it is represented on a second line.
    \\Change a subscript using any of the following:
\begin{bullets}
      \item Drag the slider along the scale widget. This is convenient
      for coarse adjustment.
      \item Click on the spinbox arrows or scale troughs.
      \item Press the keyboard up/down keys.
      \item Use the keyboard to enter numbers. Fractional subscript
      values can be used to produce magnification.
      \item On an image, drag the mouse to define a bounding box.
      \item Press the 
      \texttt{Dimension} button to restore all defaults.
      \item Press the 
      \texttt{From}, 
      \texttt{To} or 
      \texttt{Step} column heading button to restore defaults in a
      column.
      \item Press the row heading buttons to toggle a row between
      defaults and saved values.
\end{bullets}
    The values selected along a dimension are defined as follows:
\begin{bullets}
      \item If $\mathit{step} > 0$ then $\mathit{from}$, $\mathit{to}$ and 
	  $\mathit{step}$ define an arithmetic progression.
      \item If $\mathit{step} = 0$ and expression is blank then use single value $\mathit{from}$.
      \item If $\mathit{step} = 0$ and expression is not blank then use this expression.
\end{bullets}
    \item The following buttons along the bottom are used to select an
    action:
    \\
    \texttt{Range}: Display minimum and maximum value.
    \\
    \texttt{Text}: Display start of data as text.
    \\
    \texttt{Graph}: Use \texttt{plot\_nao} to display data as XY graph(s).
    \\
    \texttt{Image}: Use \texttt{plot\_nao} to display data as 2D image(s).
    \\
    \texttt{Animate}: Animate window-sequence produced by 
    \texttt{Graph} or 
    \texttt{Image}.
    \\
    \texttt{NAO}: Create Numeric Array Object.
    \\
    \texttt{Re-read}: Force a read (e.g. after rewriting the file).
    

Select 
    \texttt{Raw} mode if you want the following attributes to be
    ignored:
    \\
    \texttt{scale\_factor, add\_offset, valid\_min, valid\_max,
    valid\_range}.
  \end{enumerate}


\subsubsection{\emph{Image} Files}
\label{caps-nap-menu-Image}

    This facility for 
    \emph{viewing image files} is defined by the file 
    \texttt{vif.tcl}. Standard Tk supports the PPM, PGM and GIF
    formats. If the package 
    \texttt{Img} is installed then it will be used, giving support
    for BMP, XBM, XPM, PNG, JPEG and TIFF as well. The ActiveTcl
    distribution of Tcl/Tk includes 
    \texttt{Img}.

\subsection{Command}
    \label{caps-nap-menu-Command}

The 
  \emph{Command} menu allows the following common Tcl commands to be
  executed with a mouse click:
  \begin{description}
    \item[\label{caps-nap-menu-History} \texttt{history}]
    Display command history. This differs from the command-line 
    \texttt{history} command in that it uses a scrolled window.
    Note that 
    \texttt{tkcon} also provides:
    \begin{bullets}
      \item \emph{History} entry in main menu
      \item \emph{History} entry within \emph{File}/\emph{Save}
    \end{bullets}
    \item[\label{caps-nap-menu-Exit} \texttt{exit}] Quit.
  \end{description}

  \subsection{Help}
    \label{caps-nap-menu-Help}

The 
  \emph{Help} menu provides access to local and Web documentation on:
\begin{bullets}
    \item Caps/Nap Menu
    \item Tcl/Tk
    \item Nap
    \item Caps
\end{bullets}
Note that the Web versions may be more recent than the installed
  local documentation.
