%  $Id: geog_proc.tex,v 1.6 2007/11/09 09:24:30 dav480 Exp $ 
    % Nap Library: Geographic Procedures

\section{Geographic Procedures}
    \label{geog-proc}

\subsection{Introduction}
    \label{geog-proc-Introduction}

The following procedures are defined in the files \texttt{geog.tcl} and \texttt{gshhs.tcl}. 
Note that there are also geographic \emph{functions}, which are described in section \ref{geog}.

\subsection{GSHHS Procedures}
    \label{geog-proc-gshhs}

These supplement the functions described in section \ref{gshhs-functions}.

\subsubsection{\texttt{dmp\_gshhs}
	$\mathit{nrecords\_max}$
	$\mathit{npoints\_max}$
	$\mathit{out\_file\_name}$
	$\mathit{resolution}$
	$\mathit{min\_area}$
	$\mathit{min\_longitude}$
	$\mathit{max\_longitude}$
	$\mathit{min\_latitude}$
	$\mathit{max\_latitude}$
	$\mathit{data\_dir}$
}
    \label{geog-proc-dmp-gshhs}

This dumps (prints) selected data from a GSHHS file.
The arguments are as follows:
\begin{simpleitems}
    \item $\mathit{nrecords\_max}$
	Maximum number of records to dump. (Default: $-1=all$).
    \item $\mathit{npoints\_max}$
	Maximum number of points (in each record) to dump.
	If too many points then print first and last $(\mathit{npoints\_max}+1)/2$ points.
	(Default: $-1=all$).
	Value of $-2$ means print only record count.
    \item $\mathit{out\_file\_name}$ (default: \texttt{""} $=$ standard output)
    \item $\mathit{resolution}$ (default: 25):
	Required accuracy (km).
	This determines which data file is used.
    \item $\mathit{min\_area}$ (default: \texttt{"\{-1 -1 -1 -1\}"}): \\
	This is a four-element vector specifying the minimum area (square km) of:
	\begin{bullets}
	    \item land with sea boundary,
	    \item lake (including inland sea),
	    \item island in lake,
	    \item pond in island in lake.
	\end{bullets}
	If the value is scalar or contains less than 4 elements
	then the final element is repeated to give a 4-element vector.
	A value of $-1$ is treated as $\mathit{resolution}^2/4$.
    \item $\mathit{min\_longitude}$ (default: $-180$):
	Left boundary of region. 
    \item $\mathit{max\_longitude}$ (default: $\mathit{min\_longitude} + 360$):
	Right boundary of region.
    \item $\mathit{min\_latitude}$ (default: $-90$):
	Bottom boundary of region.
    \item $\mathit{max\_latitude}$ (default: $90$):
	Top boundary of region.
    \item $\mathit{data\_dir}$:
	GSHHS data directory (see section \ref{gshhs-functions})
\end{simpleitems}

Example: \\
\begin{verbatim}
% dmp_gshhs 1 4
GSHHS File C:/dav480/tcl/lib/nap6.3/data/gshhs/gshhs_c.b

record 0, 1240 points, level=1, area=79793839.9 km^2
west=-17.533778 east=190.326000 south=-34.830444 north=77.716250
start:
68.993778 69.883722
180.000000 176.081639
end:
68.992917 68.993778
180.001556 180.000000
1 records read
\end{verbatim}

\subsubsection{\texttt{list\_gshhs\_resolutions} $\mathit{data\_dir}$}
    \label{geog-proc-list-gshhs-resolutions}

Return a sorted list of resolutions (km) corresponding to the GSHHS data files in the
specified GSHHS data directory, which has the default value described in
	section \ref{gshhs-functions}.

Example: \\
\begin{verbatim}
% list_gshhs_resolutions
1 5 25
\end{verbatim}

\subsection{Write \emph{ARC/INFO GRIDASCII} file using \texttt{put\_gridascii}}
    \label{geog-proc-put-gridascii}

Procedure 
  \texttt{put\_gridascii} writes a file in ARC/INFO GRIDASCII
  format. All cells must be squares of the same size. Coordinate
  variable 0 (usually latitude) can be either ascending or descending.

The procedure is called as follows: \\
  \texttt{put\_gridascii} $\mathit{expr}$ $\mathit{filename}$ [$\mathit{missing\_value\_string}$]
  \begin{simpleitems}
    \item $\mathit{expr}$
    Nap expression defining a matrix with coordinate variables (normally latitude and longitude)
    \item $\mathit{filename}$
    Pathname of the output file
    \item $\mathit{missing\_value\_string}$
    String representing missing value in file (Default: \texttt{1e9})
  \end{simpleitems}

Example: \\
  \texttt{\% put\_gridascii mat abc.txt -999}

  \subsection{Write \emph{Surfer} file using \texttt{put\_text\_surfer}}
    \label{geog-proc-put-text-surfer}

Procedure 
  \texttt{put\_text\_surfer} writes a file in 
  \emph{Surfer} format. Surfer is a package developed by 
  \emph{Golden Software}.

The procedure is called as follows: \\
  \texttt{put\_text\_surfer} $\mathit{expr}$ [$\mathit{filename}$] [$\mathit{format}$]
  \begin{simpleitems}
    \item $\mathit{expr}$
    Nap expression defining a matrix with coordinate variables
    \item $\mathit{filename}$
    Pathname of the output file. If none then writes to standard output.
    \item $\mathit{format}$
    C format specification of each output element (Default: \texttt{"\%g"})
  \end{simpleitems}

Example: \\
  \texttt{\% put\_text\_surfer mat abc.txt \%8.4f}
