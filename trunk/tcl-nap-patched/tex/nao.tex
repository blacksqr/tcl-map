%  $Id: nao.tex,v 1.6 2006/06/06 08:09:42 dav480 Exp $ 
    % Nap NAOs

\section{N-dimensional Array Objects (NAOs)}
    \label{nao}

  A NAO is a data structure in memory. A NAO consists of the
  following components:
  \begin{description}
    \item[slot number]
    Index of entry in internal table used to provide fast access to
    NAOs.
    \item[sequence number]
    Number (starting from 1) assigned in order of creation of
    NAOs.
    \item[OOC-name]
    Name of object-oriented command associated with NAO. Also used
    as unique identifier of NAO.
    \item[reference count]
    Number of Tcl variables, NAOs, windows, etc. pointing to this
    NAO. If the count decrements to 0 then Nap deletes the NAO and its
    associated OOC.
    \item[nap\_cd]
    pointer to Nap structure created for each interpreter.
    \item[dataType]
    one of the following:

\begin{tabular}{|l|l|}
      \hline 
        \textbf{Name} & \textbf{Description}
      \\
      \hline 
          \texttt{c8} & 8-bit character
      \\
      \hline 
          \texttt{i8} & 8-bit signed integer
      \\
      \hline 
          \texttt{i16} & 16-bit signed integer
      \\
      \hline 
          \texttt{i32} & 32-bit signed integer
      \\
      \hline 
          \texttt{u8} & 8-bit unsigned integer
      \\
      \hline 
          \texttt{u16} & 16-bit unsigned integer
      \\
      \hline 
          \texttt{u32} & 32-bit unsigned integer
      \\
      \hline 
          \texttt{f32} & 32-bit floating-point
      \\
      \hline 
          \texttt{f64} & 64-bit floating-point
      \\
      \hline 
          \texttt{ragged} & compressed
      \\
      \hline 
          \texttt{boxed} & slot numbers (used as pointers to NAOs)
      \\
    \hline
\end{tabular}

See section
\ref{data-type-Nap-Data-types}
(Data Types)
for further details of these types.

    \item[step]
    An efficiency hint for searching vectors. If this component is
    undefined before a search then Nap defines it according to whether
    the vector is
    \begin{bullets}
      \item unordered
      \item sorted into ascending order
      \item sorted into descending order
      \item quasi-arithmetic-progression i.e. a vector all of whose steps
      are equal, except possibly the final one which may be
      shorter.
    \end{bullets}
    \item[mortal]
    This is set FALSE for NAOs (e.g. standard missing values) which
    must not be deleted regardless of reference count.
    \item[format]
    Text containing C format for printing.
    \item[label]
    Text containing title, description of data, etc.
    \item[unit]
    Text defining unit of measure.
    \item[rank]
    number of dimensions.
    \item[nels]
    number of elements = product(shape).
    \item[nbytes]
    number of bytes in NAO. Mainly for debugging.
    \item[link slot]
    slot number (0 = none) of link NAO, which can be used to attach
    further information to a NAO (possibly via a boxed NAO which could
    link to any number of further NAOs).
    \item[next slot]
    used internally to create NAO 
    \emph{death list} when executing a Tcl procedure defining a NAP
    function. (0 = none)
    \item[missing value slot]
    slot number (0=none) of missing-value NAO.
    \item[pointer to missing value]
    pointer to missing value NAO (for fast access).
    \item[pointer to missing value function]
    This function tests whether an element of a NAO is
    missing.
    \item[ragged start slot number]
    slot number (0=none) of vector NAO giving start index of each
    row of ragged array.
    \item[shape]
    sizes of dimensions.
    \item[dimension names]
    names (if any) of dimensions.
    \item[coordinate-variable slot numbers]
    slot numbers (0 = none) of CVs.
    \item[data]
    Main data.
  \end{description}
