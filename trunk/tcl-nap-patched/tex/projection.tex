%  $Id: projection.tex,v 1.8 2007/11/09 07:09:48 dav480 Exp $ 
    % Nap Library: projection.tcl

\section{Map Projection Procedures}
    \label{projection}

\subsection{Introduction}
    \label{projection-Introduction}

The following procedures are defined in the file 
  \texttt{projection.tcl}.
Note that these procedures are now deprecated in favour of the PROJ.4 functions described in
sections \ref{function-Cart-Proj} and \ref{geog-Cartographic}.

\subsection{\texttt{projection} $\mathit{code}$ $p_0$ $p_1$ $p_2$ $\ldots$}

Define functions 
  \texttt{projection\_x} and 
  \texttt{projection\_y} for specified map projection.
  


  $\mathit{code}$ is map projection code (default: 
  \texttt{CylindricalEquidistant}) as follows:
\begin{bullets}
    \item 
    \texttt{CylindricalEquidistant}: 
    $p_0$ = x-origin (default: 
    \texttt{""})
    \item 
    \texttt{Mercator}: 
    $p_0$ = x-origin (default: 
    \texttt{""})
    \item 
    \texttt{NorthPolarEquidistant}: North Polar azimuthal
    equidistant
    \item 
    \texttt{SouthPolarEquidistant}: South Polar azimuthal
    equidistant
    \item 
    \texttt{SouthPolarStereographic}: As used by IASOS
\end{bullets}

  $p_0$ 
  $p_1$ 
  $p_2$ $\ldots$ define parameters of the projection. Some
  projections use 
  $p_0$ to specify an `x-origin'. This is the minimum
  result to be returned by projection\_x. If x-origin is 
  \texttt{""} then there is no defined minimum
  result.
