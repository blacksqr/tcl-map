%  $Id: plot_nao.tex,v 1.9 2006/09/19 00:35:10 dav480 Exp $ 
    % Nap plot\_nao

\section{Visualisation using procedure \texttt{plot\_nao}}
    \label{plot-nao}

\subsection{Introduction}
    \label{plot-nao-Introduction}

A NAO can be visualised using procedure 
  \texttt{plot\_nao}, which can represent it by:
\begin{bullets}
    \item xy graphs (1D and 2D NAOs)
    \item bar-charts and histograms (1D and 2D NAOs)
    \item color-coded z-images and maps (2D NAOs)
    \item RGB z-images and maps (3D NAOs with three layers for red, green
    and blue)
    \item tiled images (multiple color-coded z-images on a page) (3D
    NAOs)
\end{bullets}
If there are additional dimensions then it is possible to
  generate multiple frames which can be animated. See examples.
  
The right mouse button displays a menu. The left mouse button
  saves (x,y,z) values (z-images only) which can be written to a
  file.
  
The Tcl code is in the files 
  \texttt{plot\_nao.tcl} and 
  \texttt{plot\_nao\_procs.tcl}.

\subsection{Usage}
    \label{plot-nao-Usage}

  \texttt{plot\_nao} 
  \emph{expression ?options?}

\subsubsection{Options}
    \label{plot-nao-Options}

Most options can be set using either the above 
  \emph{command-line} or the 
  \emph{options menu}.
Command-line options are as follows:
\begin{simpleitems}
  \item
  \texttt{-barwidth} 
  \emph{float}: (bar chart only) width of bars in x-coordinate units
  (Default: 1.0)
  \item
  \texttt{-buttonCommand} 
  \emph{script}: executed when button pressed with z-plots 
    (Default: 
  \\
\texttt{"lappend\ Plot\_nao::\$\{window\_id\}::save\ [set\ Plot\_nao::\$\{window\_id\}::xyz]"})
  \item
  \texttt{-colors} 
  \emph{list}: Colors of xy graphs or bars. (Default: 
  \texttt{black red green blue yellow orange purple grey aquamarine
  beige})
  \item
  \texttt{-columns} 
  \emph{int}: (tiled-plot only) number of columns of tiles on page
  \item
  \texttt{-dash} 
  \emph{list}: Dash patterns of xy-graph lines. (Default: 
  \texttt{""} i.e. all full lines) Each element is 
  \texttt{""} for full line, 
  \texttt{" "} for no line, or standard Tk dash pattern
  (See entry for \emph{canvas} in Tk manual).
  \item
  \texttt{-discrete} 0 or 1: 1 = Discrete colors between major z
  tick marks. (Default: 0)
  \item
  \texttt{-filename} 
  \emph{name with extension} 
  \texttt{.ps, .gif, .jpeg,} 
  \emph{etc.:} File produced by 
  \texttt{-print} (Default: Print rather than writing a file)
  \item
  \texttt{-fill} 0 or 1: 1 = Scale PostScript to fill page.
  (Default: 0)
  \item
  \texttt{-font\_standard} 
  \emph{font}: Main font. (Default: 
  \texttt{"courier 10"})
  \item
  \texttt{-font\_title} 
  \emph{font}: Font for title. (Default: 
  \texttt{"courier 16"})
  \item
  \texttt{-gap\_height} 
  \emph{int}: height (pixels) of horizontal white gaps (Default: 20)
  \item
  \texttt{-gap\_width} 
  \emph{int}: width (pixels) of vertical white gaps (Default: 20)
  \item
  \texttt{-geometry} 
  \emph{string}: If specified then use to create new toplevel window.
  \\
  (E.g. \texttt{"-geometry\ +0+0"} for top left corner)
  \item
  \texttt{-height} 
  \emph{int}: Desired height (screen units). Not used for tiled-plot.
\begin{bullets}
  \item  Type \texttt{xy/bar}: Height of whole window (Default: automatic)
  \item  Type \texttt{z}: Image height (can be `$\mathit{min}$ $\mathit{max}$' for range)
    (Default: NAO dim if within limits)
\end{bullets}
  \item
  \texttt{-key} 
  \emph{int}: width (pixels) of key. No key if 0 or blank. (Default: 30)
  \item
  \texttt{-labels} 
  \emph{list}: Labels of tiles or xy-graphs
  \item
  \texttt{-menu} 0 or 1: 0 = Start with menu bar at top hidden.
  (Default: 1)
  \item
  \texttt{-orientation P, L} 
  \emph{or} 
  \texttt{A:} P = portrait, L = landscape, A = automatic (Default:
  A)
  \item
  \texttt{-overlay C, L, S, N} 
  or $\texttt{"E}~\mathit{expression}\texttt{"}$: Define overlay. C = coast, L = land, S =
  sea, N = none, E = $\mathit{expr}$ (Default: N)
  \item
  \texttt{-oversize\_prompt} 0 or 1: 1 = prompt if image is larger
  than screen. (Default: 1)
  \item
  \texttt{-ovpal}~$\mathit{expression}$:
    Overlay palette in same form as main palette
  (Default: \texttt{black white red green blue})
  \item
  \texttt{-palette} $\mathit{expression}$:
  Main palette defining color map for 2D image. This
  is matrix with 3 or 4 columns and up to 256 rows. If there are 4
  columns then the first gives color indices in range 0 to 255. Values
  can be whole numbers in range 0 to 255 or fractional values from 0.0
  to 1.0. 
  \texttt{""} = black-to-white. (Default: blue-to-red)
  \item
  \texttt{-paperheight} 
  \emph{distance}: E.g. 
  \texttt{11i} = 11 inch (Default: 
  \texttt{297m} = 297 mm (A4))
  \item
  \texttt{-paperwidth} 
  \emph{distance}: E.g. 
  \texttt{8.5i} = 8.5 inch (Default: 
  \texttt{210m} = 210 mm (A4))
  \item
  \texttt{-parent} 
  \emph{string}: parent window (Default: 
  \texttt{""} i.e. create toplevel window)
  \item
  \texttt{-print} 0 or 1: 1 = automatic print/write (for batch
  processing) (Default: 0)
  \item
  \texttt{-printer} 
  \emph{string}: name (Default: 
  \texttt{env(PRINTER)} if defined, else any printer)
  \item
  \texttt{-range} $\mathit{expression}$:
  defines scaling (Default: auto scaling)
  \item
  \texttt{-rank} 1, 2 or 3: rank of sub-arrays to be displayed
  (Default: 
  \texttt{3 <<< rank}$(\mathit{data})$ )
  \item
  \texttt{-scaling} 0 or 1: 0 = Start with scaling widget hidden.
  (Default: 1)
  \item
  \texttt{-symbols} 
  \emph{list}: Symbol drawn at each point of xy-graph. Can be 
  \texttt{plus, square, circle, cross, splus, scross, triangle} or
  single character (e.g. 
  \texttt{"*"} ) (Default: 
  \texttt{""} i.e. none)
  \item
  \texttt{-title} 
  \emph{string}: title (Default: NAO label (if any) else $\mathit{expression}$
  \item
  \texttt{-type} 
  \emph{string} plot-type (\texttt{bar, tile, xy} or \texttt{z} )
	\\
	If rank is 1 then default type is \texttt{"xy"}
	\\
	If rank is 2 and n\_rows $<$= 8 then default type is \texttt{"xy"}
	\\
	If rank is 2 and n\_rows $>$ 8 then default type is \texttt{"z"}
	\\
	If rank is 3 then default type is \texttt{"z"}
  \item
  \texttt{-width} 
  \emph{int}: Desired width (screen units). Not used for tiled-plot.
\begin{bullets}
  \item  Type \texttt{xy/bar}: Width of whole window (Default: automatic)
  \item  Type \texttt{z}: Image width (can be `$\mathit{min}$ $\mathit{max}$' for range)
    (Default: NAO dim if within limits)
\end{bullets}
  \item
  \texttt{-xaxis} 0 or 1: Draw x-axis? 0 = no, 1 = yes. (Default: 1)
  \item
  \texttt{-xflip} 0 or 1: Flip left-right? 0 = no, 1 = yes.  (Default: 0)
  \item
  \texttt{-xlabel} \emph{string}: x-axis label (Default: name of last (final) dimension)
  \item
  \texttt{-xproc} \emph{string}: name of procedure to format x-axis tick values (Default: none)
  \item
  \texttt{-xticks} $\mathit{expression}$:
  Major tick marks of x-axis (Default: automatic)
  \item
  \texttt{-yaxis} 0 or 1: Draw y-axis? 0 = no, 1 = yes. (Default:
  1)
  \item
  \texttt{-yflip 0, 1, ascending or geog}: Flip upside down?  (Default: \texttt{geog})
    \\
    0 = no,
    \\
    1 = yes, 
    \\
    \texttt{ascending} = `if y ascending', 
    \\
    \texttt{geog} = `if ascending and (y\_dim\_name = \texttt{latitude} or y\_unit = 
	  \texttt{degrees\_north} (or equivalent))' 
  \item
  \texttt{-ylabel} 
  \emph{string}: y-axis label (Default: name of 2nd-last dimension)
  \item
  \texttt{-yproc} 
  \emph{string}: name of procedure to format y-axis tick values
  (Default: none)
  \item
  \texttt{-yticks} $\mathit{expression}$:
  Major tick marks of y-axis (Default: automatic)
  \item
  \texttt{-zlabels} 
  \emph{list}: z-axis labels of values 0, 1, 2, $\ldots$ (Default: none)
  \item
  \texttt{-zticks} $\mathit{expression}$:
  Major tick marks of z-axis (Default:
  automatic)
\end{simpleitems}

\subsection{Examples}
    \label{plot-nao-Examples}

You can cut and paste the following examples into tkcon or wish.

\subsubsection{x-y graphs and bar-charts}
    \label{plot-nao-xy}

  \begin{verbatim}
nap "sales = {
    {2 5 1 3 5 7 9 1 2 9 1 0}
    {9 2 5 5 3 9 2 0 8 8 3 8}
}"
nap "month = 1 .. 12"
$sales set coord "" month
$sales set label "Car sales"
plot_nao sales -labels "Joe Mary" -xtick "1..12" -dash {{} .} -symbols "plus cross"
proc format_x x {lindex {{} Jan Feb Mar Apr May Jun Jul Aug Sep Oct Nov Dec} $x}
plot_nao sales -labels "Joe Mary" -xtick "1..12" -type bar -xproc format_x
\end{verbatim}

\subsubsection{Scattergram}
    \label{plot-nao-scattergram}

  \begin{verbatim}
nap "x = {1.1 3.2 2.0 5.9 7.7 4.5 6.3}"
nap "y = {5.0 4.1 9.9 3.7 1.2 2.1 4.5}"
$y set coo x
plot_nao y -symbols plus -dash " "
\end{verbatim}

\subsubsection{Color-coded z-images}
    \label{plot-nao-z2}

These examples define and use the 2D NAO 
  \texttt{z}.
  \begin{verbatim}
nap "n = 200"
nap "x = n ... 0.0 .. 10.0"
nap "y = x(-)"
nap "z = sin(x) * sin(reshape(n#y, 2#n))"
$z set coo y x
plot_nao z
plot_nao z -zticks "-1 .. 1 ... 0.2" -discrete 1; # discrete colors
plot_nao "nint(z+1)" -zlabels {{zero (0)} {one (1)} {two}}; # labelled values 0, 1, 2
\end{verbatim}

\subsubsection{RGB z-images, tiles and animation}
    \label{plot-nao-z3}

These examples define and use the 3D NAO 
  \texttt{z3d}. This is defined using 
  \texttt{z} defined above.
  \begin{verbatim}
nap "z3d = z /// z*z // z**3 // z**4"
plot_nao z3d; # layer 0 = red, layer 1 = green, layer 2 = blue, layer 3 is ignored
plot_nao z3d -type tile -labels {z {z * z} z**3 z**4} -title "powers of z"
set frames [plot_nao z3d -rank 2]; # create four 2D frames
animate $frames; # animate these frames
\end{verbatim}

\subsubsection{Printing and writing files}
    \label{plot-nao-print}

Note that the 
  \texttt{-print\ 1} option can be used to print and write
  automatically in batch mode operation. It may be necessary to specify
  `\texttt{-geometry\ +0+0}' to ensure the window is
  entirely visible.
  \begin{verbatim}
nap "x = 200 ... 0 .. 4p"
nap "y = sin x"
$y set coo x
plot_nao y -print 1; # Print on default printer
plot_nao y -print 1 -geometry +0+0 -filename sin.ps; # write postscript file sin.ps
plot_nao y -print 1 -geometry +0+0 -filename sin.jpeg; # write JPEG image file sin.jpeg
\end{verbatim}

