%  $Id: bin_io.tex,v 1.11 2007/12/01 05:43:36 dav480 Exp $ 
    % Nap Library: bin\_io.tcl

\section{Binary Input/Output Procedures}
    \label{bin-io}

\subsection{Introduction}
    \label{bin-io-Introduction}

    The following procedures are defined in the file 
  \texttt{bin\_io.tcl}.

\subsection{Reading and Writing Simple Binary Files}
    \label{bin-io-Simple-Binary-Files}

    A simple binary file is a file containing nothing except data of
  a single data type.

\subsubsection{Procedure
\texttt{get\_nao} [$\mathit{fileName}$ [$\mathit{dataType}$ [$\mathit{shape}$]]]}
    \label{bin-io-get-nao}

  Read NAO from binary file.
  \\
  $\mathit{filename}$: file name (default: 
  \texttt{""} which is treated as 
  \texttt{stdin})
  \\
  $\mathit{dataType}$: 
  \texttt{c8}, 
  \texttt{u8}, 
  \texttt{u16}, 
  \texttt{u32}, 
  \texttt{i8}, 
  \texttt{i16}, 
  \texttt{i32}, 
  \texttt{f32} or 
  \texttt{f64}
  \\
  $\mathit{shape}$: shape of result (Default: number of elements until
  end)

\subsubsection{Procedure \texttt{put\_nao} [$napExpr$ [$\mathit{fileName}$]]}
    \label{bin-io-put-nao}

  Write NAO to binary file.
  \\
  $napExpr$: Nap expression to be evaluated in caller
  namespace
  \\
  $\mathit{fileName}$: file name (default: 
  \texttt{stdout})

\subsection{Reading and Writing Fortran Unformatted Files}
    \label{bin-io-Fortran-Unformatted-Files}

  Fortran unformatted files are files consisting of binary records
  preceded and followed by 32-bit byte-counts.

\subsubsection{Procedure
\texttt{get\_bin} $\mathit{dataType}$ [$\mathit{fileId}$ [$\mathit{mode}$]]}
    \label{bin-io-get-bin}

  Read next Fortran binary (unformatted) record.
  \\
  $\mathit{dataType}$: 
  \texttt{c8}, 
  \texttt{u8}, 
  \texttt{u16}, 
  \texttt{u32}, 
  \texttt{i8}, 
  \texttt{i16}, 
  \texttt{i32}, 
  \texttt{f32} or 
  \texttt{f64}
  \\
  $\mathit{fileId}$: Tcl file handle (default: 
  \texttt{stdin})
  \\
  $\mathit{mode}$: 
  \texttt{binary} (default) or 
  \texttt{swap}

\subsubsection{Procedure \texttt{put\_bin} $napExpr$ [$\mathit{fileId}$ [$\mathit{mode}$]]}
    \label{bin-io-put-bin}

  Write Fortran binary (unformatted) record.
  \\
  $napExpr$: Nap expression to be evaluated in caller
  namespace
  \\
  $\mathit{fileId}$: Tcl file handle (default: 
  \texttt{stdout})
  \\
  $\mathit{mode}$: 
  \texttt{binary} (default) or 
  \texttt{swap}

\subsubsection{Example}
    \label{bin-io-Example}

  The following example creates a NAO called 
  \texttt{squares}, writes it to a file, then reads the data from
  this file into a NAO called 
  \texttt{in}.
  \begin{verbatim}
% nap "squares = (0 .. 4)**2"
::NAP::66-66
% $squares all
::NAP::66-66  f32  MissingValue: NaN  References: 1  Unit: (NULL)
Dimension 0   Size: 5      Name: (NULL)    Coordinate-variable: (NULL)
Value:
0 1 4 9 16
% set file [open tmp.bin w]; # open file "tmp.bin" for
writing
file5
% put_bin squares $file; # write data from squares
% close $file
% set file [open tmp.bin]; # open file "tmp.bin" for reading
file5
% nap "in = [get_bin f32 $file]"; # read data into in
::NAP::78-78
% close $file
% $in all
::NAP::78-78  f32  MissingValue: NaN  References: 1  Unit: (NULL)
Dimension 0   Size: 5      Name: (NULL)    Coordinate-variable: (NULL)
Value:
0 1 4 9 16
\end{verbatim}

\subsection{Reading and Writing \texttt{cif} Files}
    \label{bin-io-cif}

The 
  \texttt{cif} (\texttt{conmap} input file) format is one which originated in the
  Melbourne University Department of Meteorology in the days before
  netCDF and HDF. It is now rather obsolete but is still used within
  CSIRO and other Australian organisations. A cif is a Fortran
  unformatted file consisting of one or more frames, each of which
  consists of six records as follows:
\begin{bullets}
    \item number of rows
    \item vertical coordinate variable (often latitude)
    \item number of columns
    \item horizontal coordinate variable (often longitude)
    \item title
    \item main data (matrix)
\end{bullets}

The main input procedure is 
  \texttt{get\_cif}, which reads one or more matrices from each of
  one or more cif files. The main output procedure is 
  \texttt{put\_cif}, which writes a NAO as an entire cif. These
  procedures call the low-level procedures 
  \texttt{get\_cif1} or 
  \texttt{put\_cif1} for each frame. The default 
  $\mathit{mode}$ for 
  \texttt{put\_cif} and 
  \texttt{put\_cif1} is 
  \texttt{auto} which produces a big-endian file regardless of the
  platform.

\subsubsection{Procedure
\texttt{get\_cif} [$\mathit{options}$] $\mathit{pattern}$ [$\mathit{pattern}$ ...]}
    \label{bin-io-get-cif}

  Read one or more matrices from each of one or more cif files
  (whose names are specified by one or more glob patterns). The result
  is 2D if only one frame is read, otherwise it is 3D. Check whether
  byte swapping is needed by examining 1st word in file.
  \\Options:
  \\
  \texttt{-g 0|1}: 
  \texttt{1} (default) for geographic mode, 
  \texttt{0} for non-geographic mode
  \\
  \texttt{-m} 
  \textbf{real}: Input missing value (default: 
  \texttt{-7777777.0})
  \\
  \texttt{-um} 
  \textbf{text}: Units for matrix (default: none)
  \\
  \texttt{-ux} 
  \textbf{text}: Units for x (default: if geographic mode then 
  \texttt{degrees\_east}, else none)
  \\
  \texttt{-uy} 
  \textbf{text}: Units for y (default: if geographic mode then 
  \texttt{degrees\_north}, else none)
  \\
  \texttt{-x} 
  \textbf{text}: Name of dimension x (default: if geographic mode then 
  \texttt{longitude} else x)
  \\
  \texttt{-y} 
  \textbf{text}: Name of dimension y (default: if geographic mode then 
  \texttt{latitude} else x)
  

The following example reads a single-frame cif named 
  \texttt{7.cif} into a NAO called 
  \texttt{in}, then displays it (including the coordinate
  variables).
  \begin{verbatim}
% nap "in = [get_cif 7.cif]"
::NAP::357-357
% $in all
::NAP::357-357  f32  MissingValue: NaN  References: 1  Unit: (NULL)
This data originated from ascii conmap input file 'acif.7'
Dimension 0   Size: 3      Name: latitude  Coordinate-variable: ::NAP::236-236
Dimension 1   Size: 4      Name: longitude  Coordinate-variable: ::NAP::308-308
Value:
 1  1  2 -3
 1  _  3 -4
 2  0  4  5
% [nap "coordinate_variable(in,0)"]
-60 30 60
% [nap "coordinate_variable(in,1)"]
-90 30 90 180
\end{verbatim}

\subsubsection{Procedure \texttt{put\_cif} $napExpr$ [$\mathit{fileName}$ [$\mathit{mode}$]]}
    \label{bin-io-put-cif}

  Write NAO as entire cif.
  \\
  $napExpr$: Nap expression to be evaluated in caller
  namespace
  \\
  $\mathit{fileName}$: file name (default: 
  \texttt{stdout})
  \\
  $\mathit{mode}$: 
  \texttt{auto} (default), 
  \texttt{binary} or 
  \texttt{swap}

\subsubsection{Procedure
\texttt{get\_cif1} [$\mathit{options}$] $\mathit{fileId}$}
    \label{bin-io-get-cif1}

  Read next frame from cif (Conmap Input File).
  \\Options:
  \\
  \texttt{-g 0|1}: 
  \texttt{1} (default) for geographic mode, 
  \texttt{0} for non-geographic mode
  \\
  \texttt{-m} 
  \textbf{real}: Input missing value (default: 
  \texttt{-7777777.0})
  \\
  \texttt{-s 0|1}: 
  \texttt{0} (default) for binary mode, 
  \texttt{1} for swap (byte-swapping) mode
  \\
  \texttt{-um} 
  \textbf{text}: Units for matrix (default: none)
  \\
  \texttt{-ux} 
  \textbf{text}: Units for x (default: if geographic mode then 
  \texttt{degrees\_east}, else none)
  \\
  \texttt{-uy} 
  \textbf{text}: Units for y (default: if geographic mode then 
  \texttt{degrees\_north}, else none)
  \\
  \texttt{-x} 
  \textbf{text}: Name of dimension x (default: if geographic mode then 
  \texttt{longitude} else x)
  \\
  \texttt{-y} 
  \textbf{text}: Name of dimension y (default: if geographic mode then 
  \texttt{latitude} else x)
  

The following example reads the first frame of a cif named 
  \texttt{7.cif} into a NAO called 
  \texttt{in}, then displays it (including the coordinate
  variables).
  \begin{verbatim}
% set f [open 7.cif]
file5
% nap "in = [get_cif1 $f]"
::NAP::218-218
% close $f
% $in all
::NAP::218-218  f32  MissingValue: NaN  References: 1  Unit: (NULL)
This data originated from ascii conmap input file 'acif.7'
Dimension 0   Size: 3      Name: latitude  Coordinate-variable: ::NAP::97-97
Dimension 1   Size: 4      Name: longitude  Coordinate-variable: ::NAP::169-169
Value:
 1  1  2 -3
 1  _  3 -4
 2  0  4  5
% ::NAP::97-97
-60 30 60
% ::NAP::169-169
-90 30 90 180
\end{verbatim}

\subsubsection{Procedure \texttt{put\_cif1} $napExpr$ [$\mathit{fileId}$ [$\mathit{mode}$]]}
    \label{bin-io-put-cif1}

  Write NAO as frame of cif.
  \\
  $napExpr$: Nap expression to be evaluated in caller
  namespace
  \\
  $\mathit{fileId}$: Tcl file handle (default: 
  \texttt{stdout})
  \\
  $\mathit{mode}$: 
  \texttt{auto} (default), 
  \texttt{binary} or 
  \texttt{swap}

\subsection{Writing HDF and netCDF files}
    \label{bin-io-hdf-netcdf}

\subsubsection{Procedure \texttt{put16} $\mathit{napExpr}$
$\mathit{fileName}$ $\mathit{variableName}$}
    \label{bin-io-put16}

  Write automatically-scaled 16-bit variable to netCDF or HDF
  file.
  \\
  $napExpr$: Nap expression to be evaluated in caller
  name-space
  \\
  $\mathit{fileName}$: name of netCDF (\texttt{.nc}) or HDF (\texttt{.hdf}) file
  \\
  $\mathit{variableName}$: name of 16-bit netCDF variable or HDF SDS
  

Example:
  \begin{verbatim}
% put16 "{{1.2e4 -99 _}{_ 101 0}}" m.nc matrix
\end{verbatim}

\subsection{Utilities}
    \label{bin-io-Utilities}

\subsubsection{Procedure \texttt{byte\_order\_mode} $\mathit{mode}$}
    \label{bin-io-byte-order-mode}

Binary I/O may require byte swapping.
Some Nap commands take an argument with the value \texttt{"swap"} to specify \emph{byte swapping}
and \texttt{"binary"} to specify \emph{no byte swapping}.

The need for byte swapping depends on two things.
Is the file big-endian or little-endian?
Is the computer platform big-endian or little-endian?

Procedure \texttt{byte\_order\_mode} allows one to specify whether 
the file is big-endian or little-endian without having to worry about whether
this matches the platform.
The result of this procedure is \texttt{"binary"} or \texttt{"swap"}.

The argument can be the value \texttt{"auto"}, \texttt{"bigEndian"}, \texttt{"binary"},
\texttt{"littleEndian"} or \texttt{"swap"}.
These can be abbreviated to a unique prefix.
Note that \texttt{"auto"} is treated as \texttt{"bigEndian"} for historic reasons.

The following example (run on a little-endian PC) shows how \texttt{byte\_order\_mode} 
can be used with
the \texttt{nap\_get} command (see section \ref{nap-get-Reading-Binary-Data})
and the binary output object-oriented commands (see sections \ref{ooc-write-binary}
and \ref{ooc-write-swap}).
  \begin{verbatim}
% set mode [byte_order_mode bigEndian]
swap
% set f [open a.big w]
file131c788
% [nap "0 .. 3"] $mode $f
% close $f
% set f [open a.big]
file131c788
% [nap_get $mode $f i32]
0 1 2 3
% close $f
\end{verbatim}
