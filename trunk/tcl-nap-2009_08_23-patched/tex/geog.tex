%  $Id: geog.tex,v 1.13 2007/11/09 09:32:08 dav480 Exp $ 
    % Nap Library: Geographic

\section{Geographic Functions}
    \label{geog}

\subsection{Introduction}
    \label{geog-Introduction}

Except where otherwise stated,
the following functions are defined in the file \texttt{geog.tcl}.
Note that this file also defines 
geographic {\em procedures}, as described in section \ref{geog-proc}.

\subsection{Divergence and Vorticity of Wind}
    \label{geog-Wind}

Functions 
  \texttt{div\_wind(}$u$\texttt{,} $v$\texttt{)}
and 
  \texttt{vorticity\_wind(}$u$\texttt{,} $v$\texttt{)}
give the divergence (second$^{-1}$) and vorticity (second$^{-1}$)
  of a 2D wind, where
  \\
  $u$ is matrix containing zonal (x-component i.e. from west
  to east) wind in metres/sec.
  \\
  $v$ is matrix containing meridional (y-component i.e. from
  south to north) wind in metres/sec.
  \\Coordinate variables of 
  $u$ and 
  $v$ are latitude ($^{\circ}$N) and longitude ($^{\circ}$E).
  
 Let 
  $\theta$ be latitude in radians, 
  $\phi$ be longitude in radians, 
  $r$ be radius of earth in metres.

  The divergence of a 2D wind is defined as
\[
  \frac{\partial u}{\partial x} + \frac{\partial v}{\partial y}
\]
Function \texttt{div\_wind} uses the equivalent formula
\[
  \frac{\frac{\partial u}{\partial \phi} + \frac{\partial(v \cos \theta)}{\partial \theta})}
  {r cos \theta}
\]
  The vorticity of a 2D wind is defined as
\[
  \frac{\partial v}{\partial x} - \frac{\partial u}{\partial y}
\]
  Function 
  \texttt{vorticity\_wind} uses the equivalent formula
\[
  \frac{\frac{\partial v}{\partial \phi} - \frac{\partial(u \cos \theta)}{\partial \theta})}
  {r cos \theta}
\]

\subsection{Things related to Latitude and Longitude}
    \label{geog-latlon}

\subsubsection{\texttt{area\_on\_globe(}$latitude$\texttt{,} $longitude$\texttt{)}}
    \label{geog-area-on-globe}

Given latitude and longitude vectors, this function calculates a
  matrix whose values are fractions of the Earth's surface area
  corresponding to cells defined by the functions 
\texttt{merid\_bounds(}$longitude$\texttt{)}
(see section \ref{geog-merid-bounds})
and 
\texttt{zone\_bounds(}$latitude$\texttt{)}
(see section \ref{geog-zone-bounds}).
  
 Example:
  \begin{verbatim}
% [nap "lats = 90 .. -90 ... -45"]
90 45 0 -45 -90
% [nap "lons = -180 .. 180 ... 45"] value
-180 -135 -90 -45 0 45 90 135 180
% [nap "area = area_on_globe(lats, lons)"] value -format %.4f
0.0046 0.0092 0.0092 0.0092 0.0092 0.0092 0.0092 0.0092 0.0046
0.0156 0.0313 0.0313 0.0313 0.0313 0.0313 0.0313 0.0313 0.0156
0.0221 0.0442 0.0442 0.0442 0.0442 0.0442 0.0442 0.0442 0.0221
0.0156 0.0313 0.0313 0.0313 0.0313 0.0313 0.0313 0.0313 0.0156
0.0046 0.0092 0.0092 0.0092 0.0092 0.0092 0.0092 0.0092 0.0046
\end{verbatim}

\subsubsection{\texttt{fix\_longitude(}$longitude$\texttt{)}}
    \label{geog-fix-longitude}

Adjust elements of longitude vector by adding or subtracting
  multiple of 360 to ensure:
\[-180 \le x_{0} < 180\]
\[0 \le x_{i+1} - x_{i} < 360\]
Also ensure unit is \texttt{degrees\_east}.

\subsubsection{\texttt{merid\_bounds(}$longitude$\texttt{)}}
    \label{geog-merid-bounds}

This function defines reasonable boundary longitudes from a
  vector giving the central longitudes of each cell. The range of
  longitudes should not exceed 360$^{\circ}$ but may be much smaller.
  
 The result is a vector with one more element than the argument.
  The midpoints between adjacent argument values define all the
  elements of the result except the first and final ones.
  
 The first and final values are defined as follows. Trial outer
  cells are added with the same width as their adjacent cell. If these
  outer cells overlap (due to global wrap-around) then both outer
  boundaries are set to the value (plus an integer multiple of 360$^{\circ}$)
  midway between the first and final values of the argument.
  
 Examples:
  \begin{verbatim}
% [nap "merid_bounds {-180 -90 0 30 90 180}"] value; # global-range
-180 -135 -45 15 60 135 180
% [nap "merid_bounds {10 30 40 50 60}"] value; # local-range
0 20 35 45 55 65
\end{verbatim}

\subsubsection{\texttt{merid\_wt(}$longitude$\texttt{)}}
    \label{geog-merid-wt}

Calculate normalised (so sum of weights is 1) meridional weights
  from longitudes. These weights are proportional to the widths of the
  cells defined by the above 
(section \ref{geog-merid-bounds})
function 
\\
\texttt{merid\_bounds(}$longitude$\texttt{)}.
  
 Examples:
  \begin{verbatim}
% [nap "merid_wt {-180 -90 0 30 90 180}"] value; # global-range
0.125 0.25 0.166667 0.125 0.208333 0.125
% [nap "merid_wt {10 30 40 50 60}"] value; # local-range
0.307692 0.230769 0.153846 0.153846 0.153846
\end{verbatim}

\subsubsection{\texttt{zone\_bounds(}$latitude$\texttt{)}}
    \label{geog-zone-bounds}

This function defines reasonable boundary latitudes from a
  vector giving the central latitudes of each cell.
  
 The result is a vector with one more element than the argument. If
  the argument is an arithmetic progression (AP) then the midpoints
  between adjacent argument values define all the elements of the
  result except the first and final ones, otherwise each of these
  boundaries separates two equal 
  \textit{surface areas} between adjacent argument values. The two
  outer boundaries are defined using steps (in latititude or area)
  equal to their adjacent step, with the obvious limit in magnitude of
  90$^{\circ}$.
  
 Examples:
  \begin{verbatim}
% [nap "zone_bounds(-80 .. 80 ... 40)"] value; # global-range AP
-90 -60 -20 20 60 90
% [nap "zone_bounds(90 .. -90 ... -30)"] value; # global-range AP
90 75 45 15 -15 -45 -75 -90
% [nap "zone_bounds{-80 -40 -10 10 40 80}"] value; # global-range non-AP
-90 -54.4687 -24.0929 0 24.0929 54.4687 90
% [nap "zone_bounds(10 .. 60 ... 10)"] value; # local-range AP
5 15 25 35 45 55 65
\end{verbatim}

\subsubsection{\texttt{zone\_wt(}$latitude$\texttt{)}}
    \label{geog-zone-wt}

Calculate normalised (so sum of weights is 1) zonal weights from latitudes.
These weights are proportional to the surface areas of the
  cells (of longitude width 360$^{\circ}$) defined by the above 
(section  \ref{geog-zone-bounds})
function 
\texttt{zone\_bounds(}$latitude$\texttt{)}.
If the Earth's radius is 
  $R$ then the area between two latitudes 
  $\theta_1$ and 
  $\theta_2$ is 
  $2\pi R^2(\sin{\theta_1} - \sin{\theta_2})$.
  
 Examples:
  \begin{verbatim}
% [nap "zone_wt(-80 .. 80 ... 40)"] value; # global-range AP
0.0669873 0.262003 0.34202 0.262003 0.0669873
% [nap "zone_wt(90 .. -90 ... -30)"] value; # global-range AP
0.0170371 0.12941 0.224144 0.258819 0.224144 0.12941 0.0170371
% [nap "zone_wt{-80 -40 -10 10 40 80}"] value; # global-range non-AP
0.0931011 0.20279 0.204109 0.204109 0.20279 0.0931011
% [nap "zone_wt(10 .. 60 ... 10)"] value; # local-range AP
0.209562 0.199962 0.184286 0.16301 0.136782 0.106398
\end{verbatim}

\subsection{Land, Water and Shoreline}
    \label{geog-land}

Functions \texttt{is\_land}, \texttt{is\_coast} and \texttt{fraction\_land} produce grid results.
Function \texttt{acof2boxed} produces a boxed array of polyline matrix NAOs.
Function \texttt{get\_gshhs} produces a {\em linked list} of polyline matrix NAOs.

\subsubsection{Functions based on \texttt{nap\_land\_flag}}
\label{land-flag-functions}

The functions 
  \texttt{is\_land}, 
  \texttt{is\_coast} and 
  \texttt{fraction\_land} are based on the 
\texttt{nap\_land\_flag} command described in section \ref{land-flag}.
This uses data which has an accuracy of 0.01$^{\circ}$.
  
These three functions are defined in the file \texttt{land.tcl}.

These functions take an optional argument 
  $data\_dir$, which specifies the directory containing the 
  \texttt{nap\_land\_flag} data files. This argument is not needed
  when the standard data directory is used.

\paragraph{\texttt{is\_land(}$latitude$\texttt{,} $longitude$[\texttt{,} $data\_dir$]\texttt{)}\\}
    \label{geog-is-land}

Produce a land/sea mask in the form of an 
  \texttt{i8} (8-bit signed integer) matrix with 1 for land and 0
  for sea.
  
 The arguments 
  $latitude$ and 
  $longitude$ can be scalars, vectors or matrices. If they are
  either both vectors, or one is a vector and the other is a scalar,
  then they are used as the coordinate variables of the matrix result.
  Otherwise the result has the same shape and coordinate variables as
  the one of higher rank.
  
 Note that each element of the result corresponds to just the
  single point defined by its latitude and longitude. It is often
  better to test multiple points, which can be done using function 
\texttt{fraction\_land(}$latitude$\texttt{,}
$longitude$[\texttt{,} $nlat$\texttt{,} $nlon$\texttt{,} $data\_dir$]\texttt{)}
described in section \ref{geog-fraction-land}.
  
 The following example produces a matrix with 
  \texttt{0} for sea and 
  \texttt{1} for land:
  \begin{verbatim}
% [nap "lats = 90 .. -90 ... -45"]
90 45 0 -45 -90
% [nap "lons = -180 .. 180 ... 45"] value
-180 -135 -90 -45 0 45 90 135 180
% [nap "isLand = is_land(lats, lons)"] value
0 0 0 0 0 0 0 0 0
0 0 1 0 1 1 1 1 0
0 0 0 0 0 0 0 0 0
0 0 0 0 0 0 0 0 0
1 1 1 1 1 1 1 1 1
\end{verbatim}

The following shows how this can be used to mask out (make
missing) the sea points in the matrix named 
  \texttt{area} defined in the the above example 
in section \ref{geog-area-on-globe}:
  \begin{verbatim}
% [nap "isLand ? area : _"] value -format %.4f
     _      _      _      _      _      _      _      _      _
     _      _ 0.0313      _ 0.0313 0.0313 0.0313 0.0313      _
     _      _      _      _      _      _      _      _      _
     _      _      _      _      _      _      _      _      _
0.0046 0.0092 0.0092 0.0092 0.0092 0.0092 0.0092 0.0092 0.0046
\end{verbatim}

The following masks out the land points:
  \begin{verbatim}
% [nap "isLand ? _ : area"] value -format %.4f
0.0046 0.0092 0.0092 0.0092 0.0092 0.0092 0.0092 0.0092 0.0046
0.0156 0.0313      _ 0.0313      _      _      _      _ 0.0156
0.0221 0.0442 0.0442 0.0442 0.0442 0.0442 0.0442 0.0442 0.0221
0.0156 0.0313 0.0313 0.0313 0.0313 0.0313 0.0313 0.0313 0.0156
     _      _      _      _      _      _      _      _      _
\end{verbatim}

The following alternative method defines a mask consisting of
zeros and missing values. Masking is done by adding this mask.
  \begin{verbatim}
% [nap "mask = {_ 0}(isLand)"] value
_ _ _ _ _ _ _ _ _
_ _ 0 _ 0 0 0 0 _
_ _ _ _ _ _ _ _ _
_ _ _ _ _ _ _ _ _
0 0 0 0 0 0 0 0 0
% [nap "area + mask"] value -format %.4f
     _      _      _      _      _      _      _      _      _
     _      _ 0.0313      _ 0.0313 0.0313 0.0313 0.0313      _
     _      _      _      _      _      _      _      _      _
     _      _      _      _      _      _      _      _      _
0.0046 0.0092 0.0092 0.0092 0.0092 0.0092 0.0092 0.0092 0.0046
\end{verbatim}

\paragraph{\texttt{is\_coast(}$latitude$\texttt{,}
$longitude$[\texttt{,} $nlat$\texttt{,} $nlon$\texttt{,} $data\_dir$]\texttt{)}\\}
    \label{geog-is-coast}

This function produces an 
  \texttt{i8} (8-bit signed integer) matrix with 1 for coast and 0
  otherwise.

Cells are defined around each latitude and longitude using the functions 
\texttt{zone\_bounds(}$latitude$\texttt{)} (see section  \ref{geog-zone-bounds}) and
\texttt{merid\_bounds(}$longitude$\texttt{)} (see section  \ref{geog-merid-bounds}).
Each cell is divided into $nlat$ rows and $nlon$ columns.
A cell is considered land if more than half its 
$nlat \times nlon$ points are classified as land by function \texttt{is\_land}.

 The result is 1 if a cell is land and has at least one adjacent
  sea cell to the north, south, east or west. The coordinate variables
  of the result are copies of the vector arguments 
  $latitude$ and 
  $longitude$.
  
 The optional argument 
  $nlat$ defaults to 8.
  \\The optional argument 
  $nlon$ defaults to 
  $nlat$.
  \\If both these arguments are omitted then each cell contains 64
  points, which is enough for most purposes. (A single point is often
  adequate for shoreline overlays on maps, etc.)

\paragraph{\texttt{fraction\_land(}$latitude$\texttt{,}
$longitude$[\texttt{,} $nlat$\texttt{,} $nlon$\texttt{,} $data\_dir$]\texttt{)}\\}
    \label{geog-fraction-land} 

This function produces an 
  \texttt{f32} (32-bit floating-point) matrix containing the
  estimated proportion of land in each cell.

Cells are defined around each latitude and longitude using the functions 
\texttt{zone\_bounds(}$latitude$\texttt{)} (see section  \ref{geog-zone-bounds}) and
\texttt{merid\_bounds(}$longitude$\texttt{)} (see section  \ref{geog-merid-bounds}).
Each cell is divided into $nlat$ rows and $nlon$ columns.
The result is defined by the proportion of the 
$nlat \times nlon$ points which are classified as land by function \texttt{is\_land}.
  
 The coordinate variables of the result are copies of the vector
  arguments 
  $latitude$ and 
  $longitude$.
  
The optional argument $nlat$ defaults to 8.
The optional argument $nlon$ defaults to $nlat$.
If both these arguments are omitted then each cell contains 64
  points, which is enough for most purposes.

The minimum resolution is 0.01$^{\circ}$ even if the specified 
  $nlat$ or 
  $nlon$ would give less. The value of 
  $nlat$ or 
  $nlon$ is automatically reduced if its specified value would
  give a resolution less than 0.01$^{\circ}$.

The following example produces a similar (but much more accurate)
land/sea mask to that produced in the above example 
in section  \ref{geog-is-land} for function \texttt{is\_land}:
\begin{verbatim}
% [nap "lats = 90 .. -90 ... -45"]
90 45 0 -45 -90
% [nap "lons = -180 .. 180 ... 45"] value
-180 -135 -90 -45 0 45 90 135 180
% [nap "fraction_land(lats, lons)"] value -format %0.2f
0.05 0.14 0.28 0.55 0.08 0.17 0.34 0.19 0.11
0.06 0.39 0.67 0.13 0.52 0.88 1.00 0.50 0.17
0.00 0.00 0.22 0.38 0.47 0.45 0.17 0.23 0.00
0.00 0.00 0.09 0.13 0.03 0.05 0.00 0.25 0.02
0.25 0.59 0.69 0.47 0.81 0.95 0.95 0.98 0.47
% [nap "fraction_land(lats, lons) > 0.5f32"] value
0 0 0 1 0 0 0 0 0
0 0 1 0 1 1 1 0 0
0 0 0 0 0 0 0 0 0
0 0 0 0 0 0 0 0 0
0 1 1 0 1 1 1 1 0
\end{verbatim}

\subsubsection{\texttt{acof2boxed(}$\mathit{filename}$[\texttt{,}
$\mathit{min\_longitude}$]\texttt{)}}
    \label{geog-acof2boxed}

Read \emph{ACOF (ascii coastal outline file)} coastline file and return a boxed nao.

Each element of this boxed nao is a 2-row matrix.
Row 0 contains longitudes.
Row 1 contains latitudes.
These coordinates define a polyline shoreline of an island, lake, etc.

The argument 
$\mathit{min\_longitude}$
(default: $-180$) specifies the desired minimum longitude.
This is normally $-180$ (giving longitudes in the range $-180$ to $180$)
or $0$ (giving longitudes in the range $0$ to $360$).

An ACOF file is a text file defining polyline shorelines.
Each polyline commences with two header lines.
The first of these contains arbitrary text (for name of Island, etc.)
which function \texttt{acof2boxed} ignores.
The second line contains the number of points in the polyline.
Each point corresponds to a line containing a longitude and latitude.

The following is a listing of an ACOF file
\texttt{nz\_tas.acof} defining very simple shorelines for Tasmania and New Zealand:

\begin{verbatim}
Tasmania
       5
    143.44    -44.60
    143.44    -41.41
    149.06    -41.41
    149.06    -44.60
    143.44    -44.60
New Zealand
       7
    165.94    -44.60
    165.94    -41.41
    177.19    -41.41
    177.19    -38.23
    171.56    -38.23
    171.56    -44.60
    165.94    -44.60
\end{verbatim}

The following example reads this file:

\begin{verbatim}
% nap "nz_tas = acof2boxed('nz_tas.acof', 0)"; # longitudes from 0 to 360
::NAP::425-425
% [nap "open_box(nz_tas(0))"] value
143.44 143.44 149.06 149.06 143.44
-44.60 -41.41 -41.41 -44.60 -44.60
% [nap "open_box(nz_tas(1))"] value
165.94 165.94 177.19 177.19 171.56 171.56 165.94
-44.60 -41.41 -41.41 -38.23 -38.23 -44.60 -44.60
\end{verbatim}

The following example shows how \texttt{acof2boxed}
can be used in conjunction with \texttt{plot\_nao} to draw shorelines:

\begin{verbatim}
plot_nao u -overlay "E acof2boxed('nz_tas.acof')"
\end{verbatim}

\subsubsection{Functions based on GSHHS}
\label{gshhs-functions}

The following functions are defined in the file \texttt{gshhs.tcl}.
\href{http://www.ngdc.noaa.gov/mgg/shorelines/gshhs.html}
{\emph{GSHHS}}
stands for
\emph{Global Self-consistent Hierarchical High-resolution Shorelines}.
This data was produced by the National Geophysical Data Center (NGDC), which is part of
the U.S. National Oceanic and Atmospheric Administration (NOAA).
GSHHS includes the following data files:

\begin{tabular}{|l|l|r|r|}
\hline
Filename & Description & Resolution & File Size \\
\hline
\texttt{gshhs\_c.b} & crude resolution & 25 km & 0.2 Mb \\
\texttt{gshhs\_l.b} & low resolution & 5 km & 1.2 Mb \\
\texttt{gshhs\_i.b} & intermediate resolution & 1 km & 5.3 Mb \\
\texttt{gshhs\_h.b} & high resolution & 0.2 km & 21.4 Mb \\
\texttt{gshhs\_f.b} & full resolution &   & 90.3 Mb \\
\hline
\end{tabular}

Only the crude, low and intermediate resolution files are distributed with Nap.
If higher resolution is required then copy the required files to the
{\em GSHHS data directory}.

The {\em GSHHS data directory} contains GSHHS files with various resolutions.
This directory can be specified using an optional argument of each function.
The default directory is
    \\
    $\mathit{tcl\_root}$\texttt{/lib/nap*.*/data/gshhs}
    \\
     unless the environment variable
    \texttt{GSHHS\_DATA\_DIR}
     is defined to overide this.

Function \texttt{proj4gshhs}, described in section \ref{geog-proj4gshhs},
overlays a GSHHS coastline on various map projections.

\paragraph{\texttt{get\_gshhs(}$\mathit{resolution}$\texttt{,} 
$\mathit{min\_area}$\texttt{,} 
$\mathit{min\_longitude}$\texttt{,} 
$\mathit{max\_longitude}$\texttt{,} 
$\mathit{min\_latitude}$\texttt{,} 
$\mathit{max\_latitude}$\texttt{,} 
$\mathit{data\_dir}$\texttt{)}\\}
    \label{geog-get-gshhs}

Read GSHHS data to produce a {\em linked list} of \texttt{f64} NAOs.
Each NAO defines the polyline shoreline (of an island, lake, etc.)
and is a matrix with two rows containing longitudes and latitudes respectively.
Each NAO (except the final one) points to the next using the
the {\em link} component described in sections
\ref{ooc-meta-link}, \ref{ooc-modify-set-link} and \ref{nao}.

The arguments are:
\begin{simpleitems}
\item $\mathit{resolution}$ (default: 1):
    Required accuracy (km).
    This determines which data file is used.
\item $\mathit{min\_area}$ (default: \texttt{"\{-1 -1 -1 -1\}"}): \\
This is a four-element vector specifying the minimum area (square km) of:
    \begin{bullets}
	\item land with sea boundary,
	\item lake (including inland sea),
	\item island in lake,
	\item pond in island in lake.
    \end{bullets}
If the value is scalar or contains less than 4 elements
then the final element is repeated to give a 4-element vector.
A value of $-1$ is treated as $\mathit{resolution}^2/4$.
\item $\mathit{min\_longitude}$ (default: $-180$):
    Left boundary of region. 
\item $\mathit{max\_longitude}$ (default: $\mathit{min\_longitude} + 360$):
    Right boundary of region.
\item $\mathit{min\_latitude}$ (default: $-90$):
    Bottom boundary of region.
\item $\mathit{max\_latitude}$ (default: $90$):
    Top boundary of region.
\item $\mathit{data\_dir}$:
    GSHHS data directory (see above)
\end{simpleitems}

An example of the use of \texttt{get\_gshhs}
can be found in the definition of the library function \texttt{proj4gshhs} in the library
file \texttt{proj4.tcl}.

\paragraph{\texttt{read\_gshhs(}$\mathit{channelId}$\texttt{,} 
$\mathit{min\_area}$\texttt{,} 
$\mathit{min\_longitude}$\texttt{,} 
$\mathit{max\_longitude}$\texttt{,} 
$\mathit{min\_latitude}$\texttt{,} 
$\mathit{max\_latitude}$\texttt{)}\\}
    \label{geog-read-gshhs}

Read next record of GSHHS data which satifies the specified conditions.
The first argument is the Tcl I/O channel (file) ID.
The remaining arguments are the same as those described above for function \texttt{get\_gshhs} in
section \ref{geog-get-gshhs}.
The data is read using the \texttt{nap\_get} command.

\subsection{Cartographic Projections}
    \label{geog-Cartographic}

The following functions are defined in the file \texttt{proj4.tcl}
and are based on the built-in cartographic projections functions described in
\ref{function-Cart-Proj}.

\subsubsection{\texttt{proj4gshhs(}$\mathit{in}$[\texttt{,}
$\mathit{projection\_spec}$\texttt{,}
$\mathit{want\_lat\_lon}$\texttt{,}
$\mathit{nx}$\texttt{,}
$\mathit{s}$\texttt{,}
$\mathit{min\_area}$\texttt{,}
$\mathit{resolution}$\texttt{,}
$\mathit{data\_dir}$\texttt{,}
$\mathit{lat\_min}$\texttt{,}
$\mathit{lat\_max}$\texttt{,}
$\mathit{lon\_min}$\texttt{,}
$\mathit{lon\_max}$]\texttt{)}}
    \label{geog-proj4gshhs}

This performs the forward mapping of any PROJ.4 projection.
The input matrix has latitude and longitude as coordinate variables.
GSHHS (see section \ref{gshhs-functions})
shorelines are overlaid unless suppressed by setting $s$ to be an empty vector.
The arguments are:
\begin{simpleitems}
\item $\mathit{in}$ Input matrix
\item $\mathit{projection\_spec}$ PROJ.4 specification. (default: \texttt{'proj=eqc'})
\item $\mathit{want\_lat\_lon}$ Type of coordinate variables of result: \\
    $0=$ northing and easting (This is default), \\
    $1=$ latitude and longitude (Legal only for projections which produce
    rows of constant latitude and columns of constant longitude)
\item $\mathit{nx}$ Number of columns in result (default: number of input columns)
\item $\mathit{s}$ Shoreline value (scalar). No shorelines if s is \texttt{\{\}} (empty vector).
    (default: missing value)
\item $\mathit{min\_area}$ Exclude shoreline polygons whose area (square km)
$< \mathit{min\_area}$ (default: 0)
\item $\mathit{resolution}$ Required shoreline accuracy (km). (default: mean input latitude step)
\item $\mathit{data\_dir}$ GSHHS data directory (default: \texttt{''} giving GSHHS default)
\item $\mathit{lat\_min}$ Minimum latitude in result (default: minimum input latitude)
\item $\mathit{lat\_max}$ Maximum latitude in result (default: maximum input latitude)
\item $\mathit{lon\_max}$ Minimum longitude in result (default: minimum input longitude)
\item $\mathit{lon\_max}$ Maximum longitude in result (default: maximum input longitude)
\end{simpleitems}

\subsection{Reading Miscellaneous Geographic Files}
    \label{geog-Reading}

\subsubsection{\texttt{get\_gridascii(}$\mathit{filename}$[\texttt{,} $\mathit{unit}$]\texttt{)}}
    \label{geog-get-gridascii}

Read a file in ARC/INFO GRIDASCII format.
  
 The optional argument 
  $unit$ specifies the unit of x and y. If this argument is
  omitted then:
  \\Dimension 
  $x$ is longitude and has unit 
  \texttt{"degrees\_east"}.
  \\Dimension 
  $y$ is latitude and has unit 
  \texttt{"degrees\_north"}.
  
 Examples:
  \begin{verbatim}
% nap "in = get_gridascii('abc.gridascii')"; # x is longitude, y is latitude
% nap "in = get_gridascii('abc.gridascii', 'metres')"
\end{verbatim}

