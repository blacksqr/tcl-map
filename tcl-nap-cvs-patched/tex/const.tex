%  $Id: const.tex,v 1.5 2006/09/22 00:37:51 dav480 Exp $ 
    % Nap Constants

\section{Constants}
    \label{const}

\subsection{Introduction}
      \label{const-Introduction}

Nap provides a rich variety of constants. Nap is oriented to
    numeric data but does provide string constants. The data-type can
    be specified as a suffix (except for strings and hexadecimal
    constants). Numeric constants can be scalars (simple numbers) or
    higher-rank arrays.

\subsection{Integer Scalar Constants}
      \label{const-Integer-Scalar-Constants}

An integer scalar constant can be specified in decimal or
    hexadecimal form. The default data-type is 
    \texttt{i32} (32-bit signed integer) for decimal integer
    constants. Octal constants are not allowed from version 3, although
    they were in earlier versions (causing problems with decimal data
    containing leading 0s).
    

Hexadecimal integer constants begin with `\texttt{0x}' and are
    32-bit unsigned integers. A data-type suffix is not allowed for
    hexadecimal constants because some cases would be ambiguous.
    

Examples of integer constants are:
    \begin{verbatim}
% [nap 14] all
::NAP::72-72  i32  MissingValue: -2147483648  References: 0  Unit: (NULL)
Value:
14
% [nap 14u8] all
::NAP::74-74  u8  MissingValue: (NULL)  References: 0  Unit: (NULL)
Value:
14
% [nap 0x14] all
::NAP::80-80  u32  MissingValue: 4294967295  References: 0  Unit: (NULL)
Value:
20
\end{verbatim}

    

The constant `\texttt{\_}' represents an 
    \texttt{i32} NAO whose value and missing-value are both
    -2147483648 (the minimum possible 
    \texttt{i32} value). It provides a convenient way of indicating
    undefined data. Such values are used mainly within array constants
    and will be discussed further in that section.

\subsection{Floating-point Scalar Constants}
      \label{const-Floating-point-Scalar-Constants}

A floating-point scalar constant can represent infinity, NaN or
    a normal finite value. A finite value is represented by a mantissa,
    optionally followed by an exponent. There can be a data-type suffix
    on any floating-point scalar constant. If this suffix is omitted
    the data-type is 
    \texttt{f64} (64-bit float).
    

A mantissa can be written in either decimal or rational form.
    A \emph{scalar} decimal mantissa must not begin or end with a decimal point.
    (However this rule is relaxed within array constants as discussed below.)
    A rational mantissa consists of two integers separated by 
    `\texttt{r}' and represents their ratio. Here are examples of
    floating-point constants without exponents:
    \begin{verbatim}
% [nap 4.0] all
::NAP::82-82  f64  MissingValue: NaN  References: 0  Unit: (NULL)
Value:
4
% [nap 4f32] all
::NAP::83-83  f32  MissingValue: NaN  References: 0  Unit: (NULL)
Value:
4
% [nap 2r3] all
::NAP::85-85  f64  MissingValue: NaN  References: 0  Unit: (NULL)
Value:
0.666667
\end{verbatim}

    

The letter 
    \texttt{e} indicates an exponent with base 10. The letter 
    \texttt{p} indicates an exponent with base $\pi$. Examples of
    constants with exponents are:
    \begin{verbatim}
% [nap 1e4] all
::NAP::89-89  f64  MissingValue: NaN  References: 0  Unit: (NULL)
Value:
10000
% [nap 1e]; # Default exponent is 1
10
% [nap 1p1]
3.14159
% [nap 1p]; # Default exponent is 1
3.14159
% [nap 180p-1]; # degrees in a radian
57.2958
% [nap 1r3p1f32] all; # pi/3
::NAP::95-95  f32  MissingValue: NaN  References: 0  Unit: (NULL)
Value:
1.0472
\end{verbatim}

    

Infinity is represented by 
    \texttt{1i}. NaN is represented by 
    \texttt{1n}. Examples are:
    \begin{verbatim}
% [nap 1i] all
::NAP::101-101  f64  MissingValue: NaN  References: 0  Unit: (NULL)
Value:
Inf
% [nap 1if32] all
::NAP::102-102  f32  MissingValue: NaN  References: 0  Unit: (NULL)
Value:
Inf
% [nap 1n] all
::NAP::104-104  f64  MissingValue: NaN  References: 0  Unit: (NULL)
Value:
_
% [nap 1nf32] all
::NAP::105-105  f32  MissingValue: NaN  References: 0  Unit: (NULL)
Value:
_
\end{verbatim}

\subsection{Numeric Array Constants}
      \label{const-Numeric-Array-Constants}

Tcl uses nested braces (`\texttt{\{\}}') to represent lists.
    Nap uses braces in a
    similar manner to represent n-dimensional constant arrays. The
    elements of array constants have the same form as scalar
    constants except that \emph{array} floating point mantissas can begin or end
    with a decimal point.
    

A vector (1-dimensional array) constant is enclosed by one level
    of braces, as in:
\begin{verbatim}
% [nap "{2 .8 -7.}"] all
::NAP::16-16  f64  MissingValue: NaN  References: 0
Dimension 0   Size: 3      Name: (NULL)    Coordinate-variable: (NULL)
Value:
2 0.8 -7
\end{verbatim}

Note that `\texttt{.8}' begins with a decimal point
and `\texttt{-7.}' ends with one.
    

A matrix (2-dimensional array) constant is enclosed by two
    levels of braces, as in:
    \begin{verbatim}
% [nap "{{1 3 5}{2 4 6}}"] all
::NAP::120-120  i32  MissingValue: -2147483648  References: 0  Unit: (NULL)
Dimension 0   Size: 2      Name: (NULL)    Coordinate-variable: (NULL)
Dimension 1   Size: 3      Name: (NULL)    Coordinate-variable: (NULL)
Value:
1 3 5
2 4 6
\end{verbatim}

    

We could have written each row on a separate line, as in
    \begin{verbatim}
% [nap "{
    {1 3 5}
    {2 4 6}
}"]
1 3 5
2 4 6
\end{verbatim}

    

The following generates a three-dimensional constant:
    \begin{verbatim}
% [nap "{{{1 5 0}{2 2 9}}{{3 0 7}{4 4 9}}}"] all
::NAP::126-126  i32  MissingValue: -2147483648  References: 0  Unit: (NULL)
Dimension 0   Size: 2      Name: (NULL)    Coordinate-variable: (NULL)
Dimension 1   Size: 2      Name: (NULL)    Coordinate-variable: (NULL)
Dimension 2   Size: 3      Name: (NULL)    Coordinate-variable: (NULL)
Value:
1 5 0
2 2 9

3 0 7
4 4 9
\end{verbatim}

    

Elements can be preceded by a `\texttt{+}'
    or `\texttt{-}' sign. Repeated elements and sub-arrays can be
    specified using `\texttt{\#}' which also has a related meaning as an
    operator. The following illustrates such repetition counts:
    \begin{verbatim}
% [nap "{{7 3#5} 2#{9 1 2#4}}"] all
::NAP::131-131  i32  MissingValue: -2147483648  References: 0  Unit: (NULL)
Dimension 0   Size: 3      Name: (NULL)    Coordinate-variable: (NULL)
Dimension 1   Size: 4      Name: (NULL)    Coordinate-variable: (NULL)
Value:
7 5 5 5
9 1 4 4
9 1 4 4
\end{verbatim}

    

Undefined (missing) elements are represented by `\texttt{\_}', as in:
    \begin{verbatim}
% [nap "{1.6 _ 0}"] all
::NAP::133-133  f64  MissingValue: NaN  References: 0  Unit: (NULL)
Dimension 0   Size: 3      Name: (NULL)    Coordinate-variable: (NULL)
Value:
1.6 _ 0
\end{verbatim}

    

It is possible to include data-type suffices on individual
    elements, but it is more convenient to use a data conversion
    function to obtain the desired data-type. For example:
    \begin{verbatim}
% [nap "f32{0 -6 1e9 1p1}"] all
::NAP::137-137  f32  MissingValue: NaN  References: 0  Unit: (NULL)
Dimension 0   Size: 4      Name: (NULL)    Coordinate-variable: (NULL)
Value:
0 -6 1e+09 3.14159
\end{verbatim}

\subsection{String Constants}
      \label{const-String-Constants}

String constants are enclosed by either two apostrophes
    (`\texttt{'}') or two grave accents
    (`\texttt{`}'). String constants have the data-type 
    \texttt{c8} (8-bit character). They are 1-dimensional (vectors)
    but other ranks can be produced using the function 
    \texttt{reshape}. A simple string constant is shown by:
    \begin{verbatim}
% [nap "'Hello world'"] all
::NAP::139-139  c8  MissingValue: (NULL)  References: 0  Unit: (NULL)
Dimension 0   Size: 11     Name: (NULL)    Coordinate-variable: (NULL)
Value:
Hello world
\end{verbatim}

    

Adjacent strings are concatenated as in:
    \begin{verbatim}
% [nap "`can't` ' go'"]
can't go
\end{verbatim}

    % 
