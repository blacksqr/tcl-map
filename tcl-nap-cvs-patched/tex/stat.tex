%  $Id: stat.tex,v 1.11 2006/09/11 04:59:26 dav480 Exp $ 
    % Nap Library: stat.tcl

\section{Statistical Functions}
    \label{stat}

\subsection{Introduction}
    \label{stat-Introduction}

These are defined in \texttt{stat.tcl}.
Note that there are also built-in statistical functions called 
\texttt{correlation} and \texttt{moving\_correlation}, which are described in section
  \ref{function-Correlation}.
The {\em tally} unary operator `\texttt{\#}'
(see section \ref{op-Tally})
also has statistical applications.

\subsection{Simple Statistics}
    \label{stat-Simple-Statistics}

Three examples are provided for each function. The first uses the
  vector 
  \texttt{v} which is defined as follows (note the missing
  values):
  \begin{verbatim}
% nap "v = {12 6 _ 7 3 15 _ 10 18 6}"
\end{verbatim}

The second example produces statistics of each column of a matrix.
  The third example produces statistics of each row of the same matrix.
  This matrix is called 
  \texttt{m} and is defined as follows:
  \begin{verbatim}
% nap "m = {
    {1.5 2.1 1.5 0.2}
    {6.2 4.9   _ 0.2}
}"
\end{verbatim}

\subsubsection{Arithmetic mean: \texttt{am(}$x$[\texttt{,} $\mathit{verb\_rank}$]\texttt{)}}
    \label{stat-am}

  \begin{verbatim}
% [nap "am(v)"]
9.625
% [nap "am(m)"]
3.85 3.5 1.5 0.2
% [nap "am(m,1)"]
1.325 3.76667
\end{verbatim}

\subsubsection{Coefficient of variation (with division by $n$):
\texttt{CV(}$x$[\texttt{,} $\mathit{verb\_rank}$]\texttt{)}}
    \label{stat-CV}

This is the uncorrected standard deviation (\texttt{sd($x$)}) divided by the arithmetic mean.
\begin{verbatim}
% [nap "CV(v)"]
0.495383
% [nap "CV(m)"]
0.61039 0.4 0 0
% [nap "CV(m,1)"]
0.523904 0.684226
\end{verbatim}

\subsubsection{Coefficient of variation (with division by $n-1$):
\texttt{CV1(}$x$[\texttt{,} $\mathit{verb\_rank}$]\texttt{)}}
    \label{stat-CV1}

This is the corrected standard deviation (\texttt{sd1($x$)}) divided by the arithmetic mean.
\begin{verbatim}
% [nap "CV1(v)"]
0.529586
% [nap "CV1(m)"]
0.863221 0.565685 _ 0
% [nap "CV1(m,1)"]
0.604952 0.838002
\end{verbatim}

\subsubsection{Geometric mean: \texttt{gm(}$x$[\texttt{,} $\mathit{verb\_rank}$]\texttt{)}}
    \label{stat-gm}

  \begin{verbatim}
% [nap "gm(v)"]
8.38752
% [nap "gm(m)"]
3.04959 3.2078 1.5 0.2
% [nap "gm(m,1)"]
0.985957 1.82476
\end{verbatim}

\subsubsection{Median: \texttt{median(}$x$[\texttt{,} $\mathit{verb\_rank}$]\texttt{)}}
    \label{stat-median}

  \begin{verbatim}
% [nap "median(v)"]
8.5
% [nap "median(m)"]
3.85 3.5 1.5 0.2
% [nap "median(m,1)"]
1.5 4.9
\end{verbatim}

\subsubsection{Mode: \texttt{mode(}$x$[\texttt{,} $\mathit{verb\_rank}$]\texttt{)}}
    \label{stat-mode}

  \begin{verbatim}
% [nap "mode(v)"]
6
% [nap "mode(m)"]
3.85 3.5 1.5 0.2
% [nap "mode(m,1)"]
1.50625 3.7625
\end{verbatim}

\subsubsection{Percentiles:
\texttt{percentile(}$x$\texttt{,} $pc$[\texttt{,} $\mathit{verb\_rank}$[\texttt{,} $nc$]]\texttt{)}}
    \label{stat-percentile}

  $pc$ = vector of required percentiles
  \\
  $nc$ = number of class intervals (default: 256)

The following examples calculate the following percentiles:
\begin{bullets}
    \item 0 (minimum)
    \item 50 (median)
    \item 100 (maximum)
\end{bullets}
\begin{verbatim}
% [nap "percentile(v, {0 50 100})"]
3 8.50781 18
% [nap "percentile(m, {0 50 100})"]
1.50000 2.10000 1.50000 0.20000
3.85918 3.50547 1.50000 0.20000
6.20000 4.90000 1.50000 0.20000
% [nap "percentile(m, {0 50 100}, 1)"]
0.20000 0.20000
1.50254 4.92266
2.10000 6.20000
\end{verbatim}

\subsubsection{Root mean square: \texttt{rms(}$x$[\texttt{,} $\mathit{verb\_rank}$]\texttt{)}}
    \label{stat-rms}

  \begin{verbatim}
% [nap "rms(v)"]
10.7413
% [nap "rms(m)"]
4.51054 3.76962 1.5 0.2
% [nap "rms(m,1)"]
1.49583 4.56399
\end{verbatim}

\subsubsection{Standard-deviation (with division by $n$):
\texttt{sd(}$x$[\texttt{,} $\mathit{verb\_rank}$]\texttt{)}}
    \label{stat-sd}

  \begin{verbatim}
% [nap "sd(v)"]
4.76806
% [nap "sd(m)"]
2.35 1.4 0 0
% [nap "sd(m,1)"]
0.694172 2.57725
\end{verbatim}

\subsubsection{Standard-deviation (with division by $n-1$):
\texttt{sd1(}$x$[\texttt{,} $\mathit{verb\_rank}$]\texttt{)}}
    \label{stat-sd1}

  \begin{verbatim}
% [nap "sd1(v)"]
5.09727
% [nap "sd1(m)"]
3.3234 1.9799 _ 0
% [nap "sd1(m,1)"]
0.801561 3.15647
\end{verbatim}

\subsubsection{Variance (with division by $n$): \texttt{var(}$x$[\texttt{,}
$\mathit{verb\_rank}$]\texttt{)}}
    \label{stat-var}

  \begin{verbatim}
% [nap "var(v)"]
22.7344
% [nap "var(m)"]
5.5225 1.96 0 0
% [nap "var(m,1)"]
0.481875 6.64222
\end{verbatim}

\subsubsection{Variance (with division by $n-1$):
\texttt{var1(}$x$[\texttt{,} $\mathit{verb\_rank}$]\texttt{)}}
    \label{stat-var1}

  \begin{verbatim}
% [nap "var1(v)"]
25.9821
% [nap "var1(m)"]
11.045 3.92 _ 0
% [nap "var1(m,1)"]
0.6425 9.96333
\end{verbatim}

\subsection{Moving Average:
\texttt{moving\_average(}$x$\texttt{,} $\mathit{shape\_window}$[, $\mathit{step}$]\texttt{)}}
    \label{stat-moving-average}

Move window of specified shape by specified step (can vary for
  each dimension). Result is arithmetic mean of 
  $x$ values in each window. 
  $x$ can have any rank $>$ 0.

$\mathit{shape\_window}$ is either a scalar or a vector with an
element for each dimension. If it is a scalar then it is treated as a vector with 
$\texttt{rank(}x\texttt{)}$
identical elements.

Similarly, $\mathit{step}$ is either a scalar or a vector with an element for
each dimension. If it is a scalar then it is treated as a vector with
$\texttt{rank(}x\texttt{)}$
identical elements.
The default value for $\mathit{step}$ is 1.
If $\mathit{step}$ has the same value as 
$\mathit{shape\_window}$ then the windows do not overlap. The value $-1$
is treated like $1$, except that missing values are prepended and
appended (along this dimension of 
$x$) to produce a result with the same dimension size as $x$.

The following examples illustrate the application of 
\texttt{moving\_average} to a vector.
\begin{verbatim}
% nap "v = {12 6 _ 7 3 15 _ 10 18 5}"
% [nap "moving_average(v, 3)"] value
9 6.5 5 8.33333 9 12.5 14 11
% [nap "moving_average(v, 3, 3)"] value
9 8.33333 14
% [nap "moving_average(v, 3, -1)"] value
9 9 6.5 5 8.33333 9 12.5 14 11 11.5
\end{verbatim}

  

The following examples illustrate the application of 
  \texttt{moving\_average} to a matrix.
  \begin{verbatim}
% nap "m = {
    {1 2 1 4 0 0}
    {5 4 2 9 2 7}
    {0 1 1 3 1 4}
}"
% [nap "moving_average(m, {3 3})"] value
1.88889 3.00000 2.55556 3.33333
% [nap "moving_average(m, {3 3}, {3 3})"] value
1.88889 3.33333
% [nap "moving_average(m, 3, 3)"] value
1.88889 3.33333
% [nap "moving_average(m, 3, -1)"] value
3.00000 2.50000 3.66667 3.00000 3.66667 2.25000
2.16667 1.88889 3.00000 2.55556 3.33333 2.33333
2.50000 2.16667 3.33333 3.00000 4.33333 3.50000
% [nap "moving_average(m, {1 3})"] value
1.333333 2.333333 1.666667 1.333333
3.666667 5.000000 4.333333 6.000000
0.666667 1.666667 1.666667 2.666667
\end{verbatim}

\subsection{Least Squares Regresssion and Curve Fitting}
    \label{least-squares}

Function \texttt{regression} provides the coefficients of a least squares 
linear multiple regression equation.
Function \texttt{fit\_polynomial} provides the coefficients of a least squares polynomial of
specified order.
Function \texttt{polynomial} evaluates a polynomial.

\subsubsection{\texttt{regression(}$x$\texttt{,} $y$\texttt{)}}
    \label{regression}

This function produces an array containing
the coefficients for the regression of $y$ on $x$.
In other words it defines an equation for predicting $y$ from $x$.

The arrays $x$ and $y$ can be vectors or a matrices.
Vectors are treated as matrices with a single column.
The columns of $x$ correspond to the known variables.
The columns of $y$ (and the result) correspond to the predicted variables.
$x$ and $y$ must have the same number of rows.
Each row corresponds to a case (e.g. person).

If $x$ and $y$ are both vectors (of the same length) then the
result is the 2-element vector $a$ which defines the simple linear regression equation
of $Y$ on $X$ given by
\[ Y = a_0 + a_1 X \]

The following example of such simple linear regression is from page 248 of 
\emph{Schaum's Outline of Theory and Problems of Statistics}, M.R. Spiegel, 1961.
The heights of fathers are used to predict the heights of their eldest sons.
\begin{verbatim}
% nap "x = {65 63 67 64 68 62 70 66 68 67 69 71}"; # height (inches) of 12 fathers
::NAP::14-14
% nap "y = {68 66 68 65 69 66 68 65 71 67 68 70}"; # height (inches) of their sons
::NAP::16-16
% [nap "regression(x, y)"]
35.8248 0.476378
\end{verbatim}
Thus the regression line of $Y$ on $X$ is
\[ Y = 35.8248 + 0.476378 X \]

Consider the case where $x$ is an $m \times n$ matrix and $y$ is an $m$-element vector.
The sample size (number of cases) is $m$.
The result is the $(n+1)$-element vector $a$ which defines
the linear multiple regression equation 
\[ Y = a_0 + a_1 X_0 + a_2 X_1 + a_3 X_2 + \cdots + a_n X_{n-1} \]
which predicts a single variable $Y$ from the $n$ variables 
$X_0, X_1, X_2, \ldots , X_{n-1}$.
If we are predicting a single case then these variables can be combined into an
$n$-element vector $X$.
If we are predicting $M$ cases then $X$ is an $M \times n$ matrix.
Thus the above equation can be expressed in matrix form as
\[ Y = (1 \| X) \cdot a \]
where `$\|$' is the \emph{concatenate} operator 
(prepending to the matrix $X$ a column of ones 
which essentially defines a pseudo-variable $X_{-1}$ which is multiplied by $a_0$)
and `$\cdot$' is the \emph{matrix-multiplication} operator.
Note that the above equation is valid for vector $X$ provided we then (as in \emph{nap})
treat `$\cdot$' as the \emph{vector dot (scalar) product} operator.

The following example of multiple linear regression is from page 273 of 
\emph{Schaum's Outline of Theory and Problems of Statistics}, M.R. Spiegel, 1961.
The ages and heights of boys are used to predict their weights.
\begin{verbatim}
% nap "age    = { 8 10  6 11  8  7 10  9 10  6 12  9}"; # years
::NAP::37-37
% nap "height = {57 59 49 62 51 50 55 48 52 42 61 57}"; # inches
::NAP::39-39
% nap "weight = {64 71 53 67 55 58 77 57 56 51 76 68}"; # pounds
::NAP::41-41
% nap "x = transpose(age /// height)"
::NAP::44-44
% [nap "b = regression(x, weight)"]
3.65122 1.50633 0.85461
\end{verbatim}
Thus the least squares regression equation of weight $w$ on age $a$ and height $h$ is
\[ w = 3.65122 + 1.50633 a + 0.85461 h \]

Let us use this equation to predict the weight of a boy who is 9 years old and 54 inches tall.
\begin{verbatim}
% [nap "{1 9 54} . b"]
63.3571
\end{verbatim}
The predicted weight is 63.3571 pounds.

Next let us use this equation to predict the weights of three boys.
The first is again 9 years old and 54 inches tall.
The second is 6 years old and 40 inches tall.
The third is 12 years old and 60 inches tall.
\begin{verbatim}
% [nap "{{1 9 54}{1 6 40}{1 12 60}} . b"]
63.3571 46.8736 73.0038
\end{verbatim}
Thus the predicted weights are 
63.3571, 46.8736 and 73.0038 respectively.

Now consider the case where we are predicting multiple variables from a single variable.
Using the above example, let us predict both height and weight from age.
We could use two separate commands as in the following:
\begin{verbatim}
% [nap "regression(age, height)"]
31.5 2.5
% [nap "regression(age, weight)"]
30.5714 3.64286
\end{verbatim}
Thus the desired regression equations (predicting height $h$ and weight $w$ from age $a$) are
\begin{eqnarray*}
h & = & 31.5    + 2.5 a \\
w & = & 30.5714 + 3.64286 a
\end{eqnarray*}
Now we can use these equations to predict the heights and weights of boys aged 6 and 10:
\begin{verbatim}
% [nap "31.5 + 2.5 * {6 10}"]
46.5 56.5
% [nap "30.5714 + 3.64286 * {6 10}"]
52.4286 67
\end{verbatim}
Thus a boy aged 6 is expected to have a height of 46.5 inches and a weight of 52.4 pounds,
while a boy aged 10 is expected to have a height of 56.5 inches and a weight of 67 pounds.

The following example shows how we can define and use both equations in a single expression:
\begin{verbatim}
% [nap "{{1 6}{1 10}} . regression(age, transpose(height /// weight))"]
46.5000 52.4286
56.5000 67.0000
\end{verbatim}

Here $x$ 
(the first argument of function $\mathit{regression}$)
is the vector $\mathit{age}$ and $y$ 
(the second argument of function $\mathit{regression}$)
is the matrix consisting of the column vectors
$\mathit{height}$ and $\mathit{weight}$.
Let $X$ be the vector containing the values of $x$ for which predictions are required.
Now the desired predictions are again given by the matrix equation
\[ Y = (1 \| X) \cdot a \]

In fact this equation is always valid, including the case where both $x$ and $y$ are matrices.

\subsubsection{\texttt{fit\_polynomial(}$x$\texttt{,} $y$\texttt{,} $n$\texttt{)}
and
\texttt{polynomial(}$c$\texttt{,} $x$\texttt{)}}
    \label{polynomial} 

Function 
\texttt{fit\_polynomial(}$x$\texttt{,} $y$\texttt{,} $n$\texttt{)}
provides the coefficients of a least squares polynomial of 
order $n$ (default 1, giving straight line).
$x$ and $y$ are vectors of the same length.

The following example uses the $\mathit{age}$ and $\mathit{height}$ vectors defined above
to define the least-squares parabola that predicts $\mathit{height}$ from $\mathit{age}$.
\begin{verbatim}
% [nap "c = fit_polynomial(age, height, 2)"]; # Define coefficients of parabola
28.8273 3.13801 -0.0364315
\end{verbatim}
Thus the parabola is
$28.8273 + 3.13801 a - 0.0364315 a^2$

The following uses this parabola to again estimate height at age 6 and 10.
\begin{verbatim}
% nap "a = {6 10}"; # ages in years
::NAP::56-56
% [nap "c(0) + c(1) * a + c(2) * a ** 2"]; # predicted height at ages 6 & 10
46.3439 56.5643
\end{verbatim}
Thus the height at age 6 is estimated to be 46.3439 inches rather than 
the previous linear estimate of 46.5 inches.
And the height at age 10 is estimated to be 56.5643 inches rather than 
the previous linear estimate of 56.5 inches.

Function \texttt{polynomial(}$c$\texttt{,} $x$\texttt)}
evaluates the polynomial defined by the vector of coefficients $c$
(e.g. as given by function \texttt{fit\_polynomial}).
$x$ can have any shape.
The result has the same shape as that of $x$.

The following example uses 
function \texttt{polynomial} to evaluate this parabola again.
\begin{verbatim}
% [nap "polynomial(c, a)"]; # predicted height at ages 6 & 10
46.3439 56.5643
\end{verbatim}

