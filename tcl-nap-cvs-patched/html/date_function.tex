%  $Id: date_function.tex,v 1.2 2006/03/02 01:52:16 dav480 Exp $ 
    % NAP Library: date.tcl functions
      \section{Functions for Dates and Times}

      \subsection{
        \label{Introduction}Introduction
      }
The calendar we use is the 
      \href{http://scienceworld.wolfram.com/astronomy/GregorianCalendar.html}{ Gregorian Calendar} introduced by Pope Gregory XIII in 1582.
      Years divisible by 400 (e.g. 1600, 2000) are leap years but other
      years divisible by 100 (e.g. 1800, 1900) are not. The Gregorian
      Calendar replaced the 
      \href{http://scienceworld.wolfram.com/astronomy/JulianCalendar.html}{ Julian Calendar} introduced by Julius Caesar in 46 BC.
      

A 
      \emph{date} is defined by a 
      \emph{year} (AD for these functions), a 
      \emph{month-of-year} and a 
      \emph{day-of-month}. The following are examples of date
      problems:
      \\How many days are there between two specified dates?
      \\What is the date 
      $n$ days after a specified date?
      \\What day of the week corresponds to a specified date?
      \\Such problems can be solved by converting both ways between
      dates and the (integer) number of days since some fixed date.
      

Let us call a combination of date and time of day a 
      \emph{date/time}. The following are examples of date/time
      problems:
      \\How many seconds are there between two specified
      date/times?
      \\What is the date/time 
      $t$ seconds after a specified date/time?
      \\Such problems can be solved by converting both ways between
      date/time and the time (as a real number of say days or seconds)
      since some fixed date/time.
      

The XXIIIrd International Astronomical Union General Assembly
      (1997) 
      \href{http://www.iers.org/iers/earth/resolutions/UAI-b1.html}{Resolution B1} recommends and defines (using different wording) the
      following terms:
      \begin{description}
        \item[Julian Day Number (JDN)]
        integer assigned to a solar day, counting from 0 assigned
        to the day starting at noon 
        \href{http://tycho.usno.navy.mil/systime.html}{UTC} on
        1st January 4713 BC (Julian Calendar).
        \item[Julian Date (JD)]
        JDN plus the fraction of the day since the preceding noon
        UTC.
        \item[Modified Julian Date (MJD)]
        time (in days) since 0000 hours UTC on 17th November 1858.
        \\MJD = JD $-$ 2400000.5.
      \end{description}
      

This astronomical definition of JDN can be relaxed for civil
      date calculations. Here one simply requires a way to associate a
      date with an integer. So the JDN can be used for civil dates
      commencing at local midnight. Alternatively, one can use the MJD
      corresponding to the preceding midnight. There is a difference of
      2400001 between these integer JDN and MJD values used to identify
      entire days (see example below). Modern dates correspond to a
      7-digit JDN, as shown in the examples below.
      

The file 
      \texttt{date.tcl} defines the four functions 
      \texttt{date2jdn}, 
      \texttt{jdn2date}, 
      \texttt{dateTime2mjd} and 
      \texttt{mjd2dateTime}. These are valid for all Gregorian
      dates from 1st January, 1 AD. The file 
      \texttt{date.tcl} also defines the two formatting procedures 
      \href{date-proc.html\#format-jdn}{ \texttt{format\_jdn} } and 
      \href{date-proc.html\#format-mjd}{ \texttt{format\_mjd} } .
      

Function 
      \texttt{date2jdn} converts Gregorian dates to JDNs.
      \\Function 
      \texttt{jdn2date} converts JDNs to Gregorian dates.
      

Function 
      \texttt{dateTime2mjd} converts Gregorian date/times to MJDs.
      \\Function 
      \texttt{mjd2dateTime} converts MJDs to Gregorian date/times.
      \\Double-precision (64-bit) floating point gives accuracy of
      about a microsecond for years around 2000 AD.
      

These functions do not the handle the 
      \href{http://tycho.usno.navy.mil/leapsec.html}{leap seconds} that occur in 
      \href{http://tycho.usno.navy.mil/systime.html}{UTC}. These
      functions simply assume that each day contains exactly 86400
      seconds (as in 
      \href{http://tycho.usno.navy.mil/systime.html}{UT1}). This
      can cause errors of several seconds in UTC time differences. This
      problem can be solved using a table of leap seconds such as that
      at the end of 
      \href{http://www.iers.org/iers/earth/resolutions/UAI-b1.html}{Resolution B1}.
      \subsection{
        \label{jdn}Dates and Julian Day Numbers (JDNs)
      }

      \subsubsection{
        \label{date2jdn} \texttt{date2jdn(}$ymd$\texttt{)}
      }
This returns the JDNs of the specified Gregorian dates. The
      unit is set to 
      \texttt{JDN}.
      

The argument 
      $ymd$ is an array whose final dimension has the size 3.
      \\Column 0 contains the year AD (positive integer).
      \\Column 1 contains the month of year (1 for Jan, 2 for Feb,
      ...).
      \\Column 2 contains the day of month (0 to 31). Final example
      below illustrates 0.
      

Examples:
      \begin{verbatim}
% [nap "date2jdn{1997 3 21}"] all; # 21st March, 1997. Note
unit
::NAP::112-112  i32  MissingValue: -2147483648  References: 0  Unit:
JDN
Value:
2450529
% [nap "date2jdn{1 1 1}"]; # 1st January, 1 AD
1721426
% [nap "date2jdn{1858 11 16}"]; # 16th November, 1858 AD
2400000
% [nap "date2jdn{{1997 2 28}{1997 3 1}}"]; # 28th February
\&amp; 1st March, 1997
2450508 2450509
% [nap "date2jdn{{2004 8 31}{2004 9 0}}"]; # Both 31st August
2004
2453249 2453249
\end{verbatim}

      

The following example calculates the day of the week for the
      first ten days of September 2004:
      \begin{verbatim}
% [nap "ymd = transpose(2004 // (9 /// 1 .. 10))"]; # 1st to
10th Sept 2004
2004    9    1
2004    9    2
2004    9    3
2004    9    4
2004    9    5
2004    9    6
2004    9    7
2004    9    8
2004    9    9
2004    9   10
% [nap "date2jdn(ymd) % 7"] value; # 0 = Mon, 1 = Tue, 2 =
Wed, ..., 6 = Sun
2 3 4 5 6 0 1 2 3 4
% [nap "1 + date2jdn(ymd) % 7"] value; # 1 = Mon, 2 = Tue,
..., 7 = Sun
3 4 5 6 7 1 2 3 4 5
% [nap "1 + (1 + date2jdn(ymd)) % 7"] value; # 1 = Sun, 2 =
Mon, ..., 7 = Sat
4 5 6 7 1 2 3 4 5 6
\end{verbatim}

      \subsubsection{
        \label{jdn2date} \texttt{jdn2date(}$jdn$\texttt{)}
      }
This is the inverse of function 
      \texttt{date2jdn} above. It converts JDNs to Gregorian dates.
      The argument 
      $jdn$ is an array of JDNs. The shape of the result is
      \\
      \texttt{shape( 
      $jdn$) // 3}
      \\where the final dimension (3) corresponds to year, month,
      day. The result is \{\_ \_ \_\} for any date prior to 1st January, 1
      AD (JDN = 1721426).
      

The following examples are the inverses of the first four
      examples above for 
      \texttt{date2jdn}:
      \begin{verbatim}
% [nap "jdn2date(1721426)"]
1 1 1
% [nap "jdn2date(2400000)"]
1858 11 16
% [nap "jdn2date(2450529)"]
1997 3 21
% [nap "jdn2date{2450508 2450509}"]
1997    2   28
1997    3    1
\end{verbatim}

      \subsection{
        \label{mjd}Date/Times and Modified Julian Dates (MJDs)
      }

      \subsubsection{
        \label{dateTime2mjd} \texttt{dateTime2mjd(}$ymdhms$\texttt{)}
      }
This returns the MJDs corresponding to the specified
      Gregorian date/times. The unit is set to 
      \texttt{MJD}.
      

The argument 
      $ymdhms$ is an array whose final dimension has a size
      from 1 to 6.
      \\Column 0 contains the year AD (positive integer)
      \\Column 1 contains the month of year (1 for Jan, 2 for Feb,
      ...) (Default: 1)
      \\Column 2 contains the day of month (0 to 31) (0: day before
      1st) (Default: 1)
      \\Column 3 contains the hour of day (0 to 23) (default: 0)
      \\Column 4 contains the minute of hour (0 to 59) (default: 0)
      \\Column 5 contains the (possibly fractional) second of
      minute (0 to 60) (default: 0)
      

Examples:
      \begin{verbatim}
% [nap "dateTime2mjd{2004 8 16 14 39 59.5}"] all -f %.7f
::NAP::525-525  f64  MissingValue: NaN  References: 0  Unit: MJD
Value:
53233.6111053
% [nap "dateTime2mjd{
{1858 11 16}
{1858 11 17}
{1858 11 18}
}"]
-1 0 1
% [nap "dateTime2mjd{1 1 1}"];# 1st Jan 0001 AD
-678575
% [nap "dateTime2mjd{2000 1 0 12}"]; # 1200 hours, 31st
December 1999
51543.5
# The following gives the difference between the JDN \&amp; MJD for a
date
% [nap "date2jdn{2004 8 16} - dateTime2mjd{2004 8 16}"] -f %d
2400001
\end{verbatim}

      \subsubsection{
        \label{mjd2dateTime} \texttt{mjd2dateTime(}$mjd$[\texttt{,} $delta$]\texttt{)}
      }
This is the inverse of function 
      \texttt{dateTime2mjd} above. It converts MJDs to Gregorian
      date/times.
      

The argument 
      $mjd$ is an array of MJDs.
      

The optional argument 
      $delta$ controls rounding. The result is rounded to the
      nearest multiple of 
      $delta$ seconds. The default value is 1, which rounds to
      the nearest second. A value of 60 would round to the nearest
      minute. A value of 1e-3 would round to the nearest
      millisecond.
      

The shape of the result is
      \\
      \texttt{shape( 
      $mjd$) // 6}
      \\where the final dimension (6) corresponds to year, month,
      day, hour, minute, second.
      

The following examples are the inverses of the first three
      examples for 
      \texttt{dateTime2mjd}:
      \begin{verbatim}
% [nap "mjd2dateTime(53233.6111053, 0.1)"]; # round to
multiple of 0.1 seconds
2004 8 16 14 39 59.5
% [nap "mjd2dateTime(-1 .. 1)"]
1858   11   16    0    0    0
1858   11   17    0    0    0
1858   11   18    0    0    0
% [nap "mjd2dateTime(-678575)"];# 1st Jan 0001 AD
1 1 1 0 0 0
\end{verbatim}

      \subsection{
        \label{fixed-years}Calendars with Fixed-length Years
      }
Computer models (of climate etc.) sometimes use simplified
      calendars in which every year has the same number of days (e.g.
      360, 365 or 366). The following two functions 
      \texttt{dateTime2days} and 
      \texttt{days2dateTime}, handle such calendars. In each case
      there is a 12-element vector argument specifying the number of
      days in each month of the year. Time is measured relative to the
      start of the base year (year 0) and can be negative.
      \subsubsection{
        \label{dateTime2days} \texttt{dateTime2days(}$ymdhms, ndim$\texttt{)}
      }
This returns the time in days since the start of year 0.
      This function is similar to the above 
      \ref{dateTime2mjd} \texttt{dateTime2mjd}  except that it uses the unconventional calendar defined by 
      $ndim$.
      

The argument 
      $ymdhms$ is an array whose final dimension has a size
      from 1 to 6.
      \\Column 0 contains the (possibly negative) integer year
      relative to year 0
      \\Column 1 contains the month of year (1 for Jan, 2 for Feb,
      ...) (Default: 1)
      \\Column 2 contains the day of month (0 to 31) (0: day before
      1st) (Default: 1)
      \\Column 3 contains the hour of day (0 to 23) (default: 0)
      \\Column 4 contains the minute of hour (0 to 59) (default: 0)
      \\Column 5 contains the (possibly fractional) second of
      minute (0 to 60) (default: 0)
      

The argument 
      $ndim$ is a 12-element vector giving 
      \emph{number of days in month}.
      

Examples:
      \begin{verbatim}
% nap "ndim = {31 28 31 30 31 30 31 31 30 31 30 31}"; #
number of days in month
% [nap "dateTime2days({1 1 1 18}, ndim)"]
365.75
% [nap "dateTime2days({{1 1 1}{1 4 10}{0 1 1}{-1 1 1}},
ndim)"]
365 464 0 -365
\end{verbatim}

      \subsubsection{
        \label{days2dateTime} \texttt{days2dateTime(}$days, ndim[, delta]$\texttt{)}
      }
This converts a time (in days since the start of year 0) to
      a calendar date and time-of-day. This function is similar to the
      above 
      \ref{mjd2dateTime} \texttt{mjd2dateTime}  except that it uses the unconventional calendar defined by 
      $ndim$.
      

The argument 
      $days$ is the (possibly negative) time in days since the
      start of the year 0.
      

The argument 
      $ndim$ is a 12-element vector giving 
      \emph{number of days in month}.
      

The optional argument 
      $delta$ controls rounding. The result is rounded to the
      nearest multiple of 
      $delta$ seconds. The default value is 1, which rounds to
      the nearest second. A value of 60 would round to the nearest
      minute. A value of 1e-3 would round to the nearest
      millisecond.
      

Examples:
      \begin{verbatim}
% nap "ndim = {31 28 31 30 31 30 31 31 30 31 30 31}"; #
number of days in month
% [nap "days2dateTime(365.75, ndim)"]
1 1 1 18 0 0
% [nap "days2dateTime({365 464 0 -365}, ndim)"]
 1  1  1  0  0  0
 1  4 10  0  0  0
 0  1  1  0  0  0
-1  1  1  0  0  0
\end{verbatim}

