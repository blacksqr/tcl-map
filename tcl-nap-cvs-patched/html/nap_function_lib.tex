%  $Id: nap_function_lib.tex,v 1.2 2006/03/01 23:13:38 dav480 Exp $ 
    % NAP Library: nap\_function\_lib.tcl
    \section{Miscellaneous Functions}

  \subsection{
    \label{Introduction}Introduction
  }
The following functions are defined in the file 
  \texttt{nap\_function\_lib.tcl}.
  \subsection{
    \label{color-wheel} \texttt{color\_wheel(}$n$\texttt{,} $v$\texttt{,} $b$\texttt{)} 
  }
Square containing color wheel.
  \\
  $n$ is number rows $\backslash$\&amp; columns
  \\
  $v$ is desired "value" level
  \\
  $b$ is background colour outside circle
  \par Example:
  \\
  \texttt{nap "color\_wheel(100,255,3\#150)"}
  \\This produces a 
  \texttt{u8} array with shape 
  \texttt{\{3 100 100\}} $\backslash$\&amp; values from 0 to 255.
  \subsection{
    \label{cv} \texttt{cv(}$main\_nao$[\texttt{,} $dim\_number$]\texttt{)}
  }
This is simply an alias for 
  \texttt{coordinate\_variable}.
  \subsection{
    \label{derivative} \texttt{derivative(}$a$[\texttt{,} $dim\_number$]\texttt{)}
  }
Estimate derivative along dimension 
  $dim\_number$ (default is 0) of array 
  $a$. The result has the same shape as 
  $a$.
  \par Example (assuming 
  \texttt{vector} has dimension ($\backslash$\&amp; coordinate variable) 
  \texttt{time}:
  \\
  \begin{verbatim}
derivative(vector); # result is derivative with respect to time
\end{verbatim}

  \par Examples (assuming 
  \texttt{matrix} has dimensions 
  \texttt{latitude} $\backslash$\&amp; 
  \texttt{longitude}):
  \\
  \begin{verbatim}
derivative(matrix, 'latitude'); # result is derivative with
respect to latitude
derivative(matrix, 0);          # result is derivative with respect to
latitude
derivative(matrix);             # result is derivative with respect to
latitude
derivative(matrix,'longitude'); # result is derivative with
respect to longitude
derivative(matrix, 1);          # result is derivative with respect to
longitude
\end{verbatim}

  \par Based on quadratic through 3 points (provided size of dimension is
  $>$ 2 -- if only 2 then based on straight line). These always
  include the point corresponding to the result. For interior points,
  the other 2 are the closest neighbour on each side. For boundary
  points, these are the 2 closest neighbours.
  \par Let D(x) be the derivative of quadratic through points 
  \texttt{(x0,y0), (x1,y1), (x2,y2)}.
  \\
  \begin{verbatim}
D1 = D(x1) = a0 * y0 + a1 * y1 + a2 * y2
where the coefficients a0, a1, a2 are defined by:
        a0 = (x1 - x2) / ((x1 - x0) * (x2 - x0))
        a1 = 1 / (x1 - x0) - 1 / (x2 - x1)
        a2 = (x1 - x0) / ((x2 - x0) * (x2 - x1))
\end{verbatim}

  \subsection{
    \label{fill-holes} \texttt{fill\_holes(}$x$\texttt{,} [$max\_nloops$]\texttt{)} 
  }

    $x$
	is array to be filled.
    \\
    $max\_nloops$
	is maximum number of iterations.
    \\
    Replace missing values by estimates based on means of neighbours.
    If $max\_nloops$ is not specified then \textit{fill\_holes}
    continues until there are no missing values.
  \subsection{
    \label{fuzzy-floor} \texttt{fuzzy\_floor(}$x$[\texttt{,} $eps$]\texttt{)}
  }
Like 
  \texttt{floor()} except allow for rounding error.
  \\
  $eps$ is tolerance and defaults to 
  \texttt{1e-9}.
  \par Example:
  \\
  \texttt{\% [nap "fuzzy\_floor(\{4.998 4.9998\},1e-3)"]
  \\4 5}
  \subsection{
    \label{fuzzy-ceil} \texttt{fuzzy\_ceil(}$x$[\texttt{,} $eps$]\texttt{)}
  }
Like 
  \texttt{ceil()} except allow for rounding error.
  \\
  $eps$ is tolerance and defaults to 
  \texttt{1e-9}.
  \par Example:
  \\
  \texttt{\% [nap "fuzzy\_ceil(\{5.002 5.0002\},1e-3)"]
  \\6 5}
  \subsection{
    \label{gets-matrix} \texttt{gets\_matrix(}$filename$[\texttt{,} $n\_header\_lines$]\texttt{)}
  }
Read text file and return NAO matrix whose rows correspond to
  the lines in the file. Ignore:
  \begin{itemize}
    \item first 
    $n\_header\_lines$ (default 0) lines
    \item blank lines
    \item lines whose first non-whitespace character is '
    \texttt{\#}'
  \end{itemize}
  \par Examples:
  \\
  \texttt{nap "m1 = gets\_matrix('matrix1.txt')"
  \\nap "m2 = gets\_matrix('matrix2.txt', 3)"; \#
  Skip 1st 3 lines}
  \subsection{
    \label{head} \texttt{head(}$x$[\texttt{,} $n$]\texttt{)}
  }
If 
  $n$ $\ge$ 0 then result is 1st 
  $n$ elements of 
  $x$, cycling if 
  $n$ $>$ nels(
  $x$).
  \\
  $n$ defaults to 1.
  \\If 
  $n$ $<$  0 then result is 1st nels(
  $x$)+
  $n$ elements of 
  $x$ i.e. drop $-$
  $n$ from end
  \\
  \par Examples:
  \\
  \begin{verbatim}
% [nap "head({3 1 9 2 7})"]
3
% [nap "head({3 1 9 2 7}, 2)"]
3 1
% [nap "head({3 1 9 2 7}, -2)"]
3 1 9
\end{verbatim}

  \subsection{
    \label{hsv2rgb} \texttt{hsv2rgb(}$hsv$\texttt{)} 
  }
Convert colour in HSV form to RGB.
  \\
  $hsv$ is an array whose leading dimension has size 3.
  \par Layer 0 along this dimension corresponds to hue as an angle in
  degrees. Angles of any sign or magnitude are allowed.
  Red\ =\ 0, yellow\ =\ 60, green\ =\ 120,
  cyan\ =\ 180, blue\ =\ $-$120,
  magenta\ =\ $-$60.
  \par Layer 1 along this dimension corresponds to saturation in range
  0.0 to 1.0.
  \par Layer 2 along this dimension corresponds to "value".
  This has the same range as the RGB values, normally either 0.0 to 1.0
  or 0 to 255. If you are casting the result to an integer $\backslash$\&amp; want a
  maximum of 255 then set the maximum to say 255.999. Otherwise you
  will get few if any 255s.
  \par The result has the same shape as the argument (hsv).
  \par See Foley, vanDam, Feiner and Hughes, Computer Graphics 
  \textit{Principles and Practice}, Second Edition, 1990, ISBN
  0201121107 page 593.
  \\
  \par Example:
  \\
  \texttt{\% [nap "hsv2rgb \{180.0 0.5 100.0\}"]
  \\50 100 100}
  \subsection{
    \label{isMissing} \texttt{isMissing(}$x$\texttt{)} 
  }
1 if 
  $x$ missing, 0 if present.
  \\
  \par Example:
  \\
  \texttt{\% [nap "isMissing \{0 \_ 9\}"]
  \\0 1 0}
  \subsection{
    \label{isPresent} \texttt{isPresent(}$x$\texttt{)} 
  }
0 if 
  $x$ missing, 1 if present.
  \\
  \par Example:
  \\
  \texttt{\% [nap "isPresent \{0 \_ 9\}"]
  \\1 0 1}
  \subsection{
    \label{magnify-interp} \texttt{magnify\_interp(}$a$\texttt{,} $mag\_factor$\texttt{)}
  }
Magnify each dimension of array 
  $a$ by factor defined by the corresponding element of 
  $mag\_factor$ if this is a vector. If this is a scalar then
  every dimension is magnified by the same factor. The new values are
  estimated using multi-linear interpolation.
  \par This function can be used to make images larger or smaller.
  \par Example:
  \\
  \texttt{\% [nap "magnify\_interp(\{\{1 2 3\}\{4 5 6\}\}, \{1 3\})"]
  value
  \\1.00000 1.33333 1.66667 2.00000 2.33333 2.66667 3.00000
  \\4.00000 4.33333 4.66667 5.00000 5.33333 5.66667
  6.00000}
  \subsection{
    \label{magnify-nearest}\texttt{magnify\_nearest(}$a$\texttt{,} $mag\_factor$\texttt{)}
  }
This function is similar to 
  \texttt{magnify\_interp} except that the new values are defined by
  the nearest neighbour rather than interpolation.
  \par Example:
  \\
  \texttt{\% [nap "magnify\_nearest(\{\{1 2 3\}\{4 5 6\}\}, \{1 3\})"]
  value
  \\1 1 2 2 2 3 3
  \\4 4 5 5 5 6 6}
  \subsection{
    \label{mixed-base} \texttt{mixed\_base(}$x$\texttt{,} $b$\texttt{)} 
  }
Convert scalar value 
  $x$ to mixed base defined by vector 
  $b$.
  \\
  \par Following example converts 87 inches to yards, feet $\backslash$\&amp; inches:
  \\
  \texttt{\% [nap "mixed\_base(87, \{3 12\})"]
  \\2 1 3}
  \subsection{
    \label{nub} \texttt{nub(}$x$\texttt{)} 
  }
Result is vector of distinct values in argument (in same order).
  \subsection{
    \label{outer} \texttt{outer(} $dyad$\texttt{,} $y$[, $x$]\texttt{)}
  }
Tensor outer-product.
  \par 
  $dyad$ is name of either
  \begin{itemize}
    \item function with two arguments
    \item binary (dyadic) operator
    \\
    $x$ is vector
    \\
    $y$ is vector defaulting to 
    $x$
    \\Result is cross-product of 
    $x$ and 
    $y$, applying 
    $dyad$ to each combination of 
    $x$ $\backslash$\&amp; 
    $y$.
    \\
    $x$ $\backslash$\&amp; 
    $y$ are the coordinate variables of the result.
    \\
  \end{itemize}
  \par Following example produces a multiplication table:
  \\
  \begin{verbatim}
% [nap "outer('*', 1 .. 5)"]
 1  2  3  4  5
 2  4  6  8 10
 3  6  9 12 15
 4  8 12 16 20
 5 10 15 20 25
\end{verbatim}

  \subsection{
    \label{palette-interpolate} \texttt{palette\_interpolate(}$from$\texttt{,} $to$)
  }
Define a palette by interpolating around the HSV (hue,
  saturation, value) colour wheel with both 
  $S$ (saturation) and 
  $V$ (value) set to 1. The arguments 
  $from$ and 
  $to$ are angles in degrees which specify the range of of 
  $H$ (hue). Red is 0, green is -240 and blue is 240.
  \subsection{
    \label{scattered2grid} \texttt{scattered2grid(}$xyz$\texttt{,} $ycv$\texttt{,} $xcv$\texttt{)} 
  }

  \par Produce a matrix grid from scattered (
  $x,y,z$) data using triangulation. Grid points within each
  triangle are defined by interpolating using a plane through the three
  vertices of the triangle.
  \par 
  $xyz$ is an 
  $n$�
  $m$ matrix containing data corresponding to 
  $n$ points (
  $x,y,z$). The number of columns (
  $m$) must be at least 3. Columns 0, 1 and 2 contain x, y and
  z values respectively. Any further columns are ignored.
  \par 
  $ycv$ and 
  $xcv$ specify the coordinate-variables for the grid.
  \par The following example defines a grid from the four points (2,2,0),
  (6,4,0), (2,4,4) and (4,5,3). Note that the missing values in the
  result correspond to points which are outside of both the triangles
  produced by the triangulation. You could eliminate these missing
  values by defining values at all four corners of the grid.
  \begin{verbatim}
% nap "z = scattered2grid({{2 2 0}{6 4 0}{2 4 4}{4 5 3}}, 2 .. 5, 2 .. 6)"
::NAP::1020-1020
% $z
0.00    _    _    _    _
2.00 0.75 0.00    _    _
4.00 2.50 1.50 0.75 0.00
   _    _ 3.00    _    _
% [nap "z(@2, @2)"]; # Check value at x=2, y=2
0
% [nap "z(@4, @6)"]; # Check value at x=6, y=4
0
% [nap "z(@4, @2)"]; # Check value at x=2, y=4
4
% [nap "z(@5, @4)"]; # Check value at x=4, y=5
3
\end{verbatim}

  \subsection{
    \label{scaleAxis} \texttt{scaleAxis(}$xstart$\texttt{,} $xend$[\texttt{,} $nmax$[\texttt{,} $nice$]]\texttt{)}
  }
Find suitable values for axis of graph.
  \\
  $xstart$: 1st data value
  \\
  $xend$: Final data value
  \\
  $nmax$: Max. allowable number of elements in result
  (Default: 10)
  \\
  $nice$: Allowable increments (Default: 
  \texttt{\{1 2 5\}})
  \\Result is the arithmetic progression which:
  \begin{itemize}
    \item is within interval from 
    $xstart$ to 
    $xend$
    \item has same order (ascending/descending) as 
    $xstart$
    \texttt{//}
    $xend$
    \item has increment equal to element of 
    $nice$ times a power 
    \texttt{(-30 .. 30)} of 10
    \item has at least two elements
    \item has no more than 
    $nmax$ elements if possible
    \item has as many elements as possible. (Ties are resolved by
    choosing earlier element in 
    $nice$.)
  \end{itemize}
  \par Example:
  \\
  \texttt{\% [nap "axis = scaleAxis(-370, 580, 10, \{10 20 25
  50\})"] value
  \\-300 -200 -100 0 100 200 300 400 500}
  \subsection{
    \label{scaleAxisSpan} \texttt{scaleAxisSpan(}$xstart$\texttt{,} $xend$[\texttt{,} $nmax$[\texttt{,} $nice$]]\texttt{)}
  }
Find suitable values for axis of graph.
  \\
  $xstart$: 1st data value
  \\
  $xend$: Final data value
  \\
  $nmax$: Max. allowable number of elements in result
  (Default: 10)
  \\
  $nice$: Allowable increments (Default: 
  \texttt{\{1 2 5\}})
  \\Result is the arithmetic progression which:
  \begin{itemize}
    \item includes the interval from 
    $xstart$ to 
    $xend$
    \item has same order (ascending/descending) as 
    $xstart$
    \texttt{//}
    $xend$
    \item has increment equal to element of 
    $nice$ times a power 
    \texttt{(-30 .. 30)} of 10
    \item has at least two elements
    \item has no more than 
    $nmax$ elements if possible
    \item has as many elements as possible. (Ties are resolved by
    choosing earlier element in 
    $nice$.)
  \end{itemize}
  \par Example:
  \\
  \texttt{\% [nap "axis = scaleAxisSpan(-370, 580, 10, \{10 20 25
  50\})"] value
  \\-400 -200 0 200 400 600}
  \subsection{
    \label{range} \texttt{range(}$a$\texttt{)} 
  }
Result is 2-element vector containing minimum and maximum of
  array 
  $a$.
  \\Example:
  \\
  \texttt{\% [nap "range \{\{9 -1 -5\}\{2 9 3\}\}"]
  \\-5 9}
  \subsection{
    \label{tail} \texttt{tail(}$x$[\texttt{,} $n$]\texttt{)}
  }
If 
  $n$ $\ge$ 0 then result is final 
  $n$ elements of 
  $x$, cycling if 
  $n$ $>$ nels(
  $x$).
  \\
  $n$ defaults to 1.
  \\If 
  $n$ $<$  0 then result is final 
  \texttt{nels(}
  $x$
  \texttt{)+}
  $n$ elements of 
  $x$ i.e. drop $-$
  $n$ from start.
  \\
  \par Example:
  \\
  \begin{verbatim}
% [nap "tail({3 1 9 2 7})"]
7
% [nap "tail({3 1 9 2 7}, 2)"]
2 7
% [nap "tail({3 1 9 2 7}, -2)"]
9 2 7
\end{verbatim}

