%  $Id: ooc_meta.tex,v 1.1 2006/03/01 10:34:15 dav480 Exp $ 
    % OOC Methods: Metadata
    \section{OOC Methods which return Metadata}

  \subsection{
    \label{intro}Introduction
  }
The following code defines vectors `\texttt{x}' and `\texttt{y}' for use in the examples below:
  \begin{verbatim}
% nap "x = 0 .. 2 ... 0.5"
::NAP::58-58
% nap "y = x ** 2"
::NAP::61-61
% $y val -format %0.3f
0.000 0.250 1.000 2.250 4.000
\end{verbatim}

  \subsection{
    \label{coordinate}Method \texttt{coordinate}
  }

  $ooc\_name$ 
  \texttt{coordinate} ?
  $dim\_name$|
  $dim\_number$? ?
  $dim\_name$|
  $dim\_number$? $\ldots$
  \\
  

This returns the OOC-names of the coordinate variables of selected
  dimensions. If no dimensions are specified then the effect is the
  same as:
  \\
  $ooc\_name$ 
  \texttt{coo} 0 1 2 $\ldots$ rank-1'
  \\
  

Example:
  \begin{verbatim}
% $y set coo x
% $y coo
::NAP::58-58
% [$y coo]
0 0.5 1 1.5 2
\end{verbatim}

  \subsection{
    \label{count}Method \texttt{count}
  }

  


  $ooc\_name$ 
  \texttt{count} 
  \texttt{-keep}
  \\
  

This returns the reference count.
  

Example (using 
  \texttt{x} defined in previous example):
  \begin{verbatim}
% $x count
2
\end{verbatim}

Note that the reference count is 2 because this NAO is referenced
by both
  \begin{itemize}
    \item Tcl variable 
    \texttt{x}
    \item coordinate variable pointer of NAO 
    \texttt{::NAP::61-61}
  \end{itemize}
  \subsection{
    \label{datatype}Method \texttt{datatype}
  }

  


  $ooc\_name$ 
  \texttt{datatype}
  \\
  

This returns the data-type.
  

Example:
  \begin{verbatim}
% [nap "'hello'"] dat
c8
\end{verbatim}

  \subsection{
    \label{dimension}Method \texttt{dimension}
  }

  


  $ooc\_name$ 
  \texttt{dimension} ?
  $dim\_number$? ?
  $dim\_number$? $\ldots$
  \\
  

This returns the dimension names.
  \\
  


  $ooc\_name$ 
  \texttt{di}
  \\is equivalent to:
  \\
  $ooc\_name$ 
  \texttt{di} 0 1 2 $\ldots$ rank-1'
  \\
  

Example (again continuing above example):
  \begin{verbatim}
% $y dim
x
\end{verbatim}

  \subsection{
    \label{format}Method \texttt{format}
  }

  


  $ooc\_name$ 
  \texttt{format}
  \\
  

This returns the C format for printing the NAO.
  

Example:
  \begin{verbatim}
% $y set format %.4f
% $y format
%.4f
% $y value
0.0000 0.2500 1.0000 2.2500 4.0000
\end{verbatim}

  \subsection{
    \label{header}Method \texttt{header}
  }

  


  $ooc\_name$ 
  \texttt{header} 
  \texttt{-keep}
  \\
  

This returns similar information to the following (but using a
  different format):
  \\
  $ooc\_name$ 
  \texttt{ooc}
  \\
  $ooc\_name$ 
  \texttt{datatype}
  \\
  $ooc\_name$ 
  \texttt{missing}
  \\
  $ooc\_name$ 
  \texttt{count}
  \\
  $ooc\_name$ 
  \texttt{unit}
  \\
  $ooc\_name$ 
  \texttt{shape}
  \\
  $ooc\_name$ 
  \texttt{dimension}
  \\
  $ooc\_name$ 
  \texttt{coordinate}
  \\
  

Example (continuing above example):
  \begin{verbatim}
% $y header
::NAP::61-61  f64  MissingValue: NaN  References: 1  Unit: (NULL)
Dimension 0   Size: 5      Name: x         Coordinate-variable:
::NAP::58-58
% 
\end{verbatim}

  \subsection{
    \label{label}Method \texttt{label}
  }

  


  $ooc\_name$ 
  \texttt{label}
  \\
  

This returns the label (title, etc.) of the NAO.
  

Example:
  \begin{verbatim}
% $y set label "areas of squares"
% $y label
areas of squares
\end{verbatim}

  \subsection{
    \label{link}Method \texttt{link}
  }

  


  $ooc\_name$ 
  \texttt{link}
  \\
  

This returns the OOC-name of the link NAO.
  

Example:
  \begin{verbatim}
% $y set link [nap 42]
% [$y link]
42
\end{verbatim}

  \subsection{
    \label{missing}Method \texttt{missing}
  }

  


  $ooc\_name$ 
  \texttt{missing}
  \\
  

This returns the missing value. This is the value used to indicate
  null or undefined data.
  

Example:
  \begin{verbatim}
% $y miss
NaN
\end{verbatim}

  \subsection{
    \label{ooc}Method \texttt{ooc}
  }

  


  $ooc\_name$ 
  \texttt{ooc} 
  \texttt{-keep}
  \\
  

This returns the OOC-name of the NAO.
  

Example:
  \begin{verbatim}
$y ooc
::NAP::61-61
\end{verbatim}

  \subsection{
    \label{rank}Method \texttt{rank}
  }

  


  $ooc\_name$ 
  \texttt{rank}
  \\
  

This returns the rank (number of dimensions).
  

Example:
  \begin{verbatim}
% $y rank
1
\end{verbatim}

  \subsection{
    \label{sequence}Method \texttt{sequence}
  }

  


  $ooc\_name$ 
  \texttt{sequence} 
  \texttt{-keep}
  \\
  

This returns the sequence number of the NAO. E.g. 42 for 
  \texttt{nao.42-9}
  \\
  \\
  \texttt{-keep}: Do not delete NAO with reference count of 0
  

Example:
  \begin{verbatim}
% $y seq
61
\end{verbatim}

  \subsection{
    \label{shape}Method \texttt{shape}
  }

  


  $ooc\_name$ 
  \texttt{shape}
  \\
  

This returns the shape, which is a vector of dimension sizes.
  

Example:
  \begin{verbatim}
% $y shape
5
\end{verbatim}

  \subsection{
    \label{slot}Method \texttt{slot}
  }

  


  $ooc\_name$ 
  \texttt{slot} 
  \texttt{-keep}
  \\
  

This returns the slot number of the NAO. E.g. 9 for 
  \texttt{nao.42-9}
  \\
  

Example:
  \begin{verbatim}
% $y sl
61
\end{verbatim}

  \subsection{
    \label{step}Method \texttt{step}
  }

  


  $ooc\_name$ 
  \texttt{step}
  \\
  

This returns a code which indicates whether step sizes of a vector
  are equal, and if not, their sign. NAP uses this information for
  efficiency. It indicates whether a vector (not relevant for other
  ranks) is monotonically ascending/descending, and if so whether it is
  an arithmetic progression (AP). The result code is one of following
  strings:
  \begin{itemize}
    \item "\texttt{+-}': at least one positive step and one negative
    step
    \item "\texttt{$>$= 0}': all steps $>$= 0
    \item "\texttt{$<$= 0}': all steps $<$= 0
    \item "\texttt{AP}': equal steps (except final one which may be
    shorter)
  \end{itemize}
  

Example:
  \begin{verbatim}
% [nap "{3 5 7 7.1}"] step
AP
\end{verbatim}

  \subsection{
    \label{unit}Method \texttt{unit}
  }

  


  $ooc\_name$ 
  \texttt{unit}
  \\
  

This returns the unit of measure. This may be used in the future
  to support arithmetic with automatic unit conversion, but at the
  moment it is just descriptive information.
  

Example:
  \begin{verbatim}
% $y set unit seconds
% $y unit
seconds
\end{verbatim}

