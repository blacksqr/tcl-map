%  $Id: nap_get.tex,v 1.3 2006/03/01 21:19:06 dav480 Exp $ 
    % nap\_get
    \section{Reading Files using 
    \texttt{nap\_get} Command}

  \section{
    \label{Introduction}Introduction
  }

  

The 
  \texttt{nap\_get} command creates a NAO containing data read from
  a file. The first argument specifies the type of file, which can be 
  \texttt{binary}, 
  \texttt{hdf}, 
  \texttt{netcdf} or 
  \texttt{swap}.
  


  \href{http://hdf.ncsa.uiuc.edu}{HDF} and 
  \href{http://www.unidata.ucar.edu/packages/netcdf/index.html}{netCDF}
  are similar array-oriented file formats which are popular in earth
  sciences such as meteorology and oceanography. (The new HDF5 format
  is not currently supported.) Such files contain data referenced by
  symbol tables containing the names, data-types and dimensions of
  variables. Each variable can also have attributes such as a label, a
  format, a unit of measure and a missing-value. Note the similarity
  between these attributes and those in a NAO.
  

The 
  \texttt{netcdf} option also supports (but only on the Linux Intel
  386 platform at the time of writing) access to remote data provided
  by an 
  \href{http://www.opendap.org/}{OPeNDAP (a.k.a. DODS)} server.
  In this case the 
  $filename$ is a URL rather than the pathname of a local
  file. There is an OPeNDAP example at the end of 
  \ref{Reading-netCDF-Data}Reading netCDF Data.
  \section{
    \label{Reading-Binary-Data}Reading Binary Data
  }

  

Binary data is read using the command
  \\
  \texttt{nap\_get binary} 
  $channel$ [
  $datatype$ [
  $shape$]]
  \\where 
  $datatype$ defaults to 
  \texttt{u8} and 
  $shape$ defaults to a vector corresponding to the file
  size.
  

The command
  \\
  \texttt{nap\_get swap} 
  $channel$ [
  $datatype$ [
  $shape$]]
  \\is similar, except that bytes are swapped. This enables reading
  of data written on a machine with opposite byte-order within
  words.
  

The following example first writes six 32-bit floating-point
  values to a file using the OOC 
  \texttt{binary} method. It then reads them back into a NAO named
  `\texttt{in}' using `\texttt{nap\_get binary}':
  \begin{verbatim}
% set file [open float.dat w]
filee15170
% [nap "f32{1.5 -3 0 2 4 5}"] binary $file
% close $file
% set file [open float.dat]
filee15170
% nap "in = [nap_get binary $file f32]"
::NAP::22-22
% close $file
% $in all
::NAP::22-22  f32  MissingValue: NaN  References: 1
Dimension 0   Size: 6      Name: (NULL)    Coordinate-variable: (NULL)
Value:
1.5 -3 0 2 4 5
\end{verbatim}

Note that no shape was specified, giving a 6-element vector. The
following example reads the file again, this time specifying a shape of
  \texttt{\{2 3\}}. The NAO is displayed but not saved.
  \begin{verbatim}
% set file [open float.dat]
file6
% [nap_get binary $file f32 "{2 3}"] all
::NAP::32-32  f32  MissingValue: NaN  References: 0  Unit: (NULL)
Dimension 0   Size: 2      Name: (NULL)    Coordinate-variable: (NULL)
Dimension 1   Size: 3      Name: (NULL)    Coordinate-variable: (NULL)
Value:
 1.5 -3.0  0.0
 2.0  4.0  5.0
% close $file
\end{verbatim}

  \section{
    \label{Reading-netCDF-Data}Reading netCDF Data
  }

  

NetCDF data is read using the command
  \\
  \texttt{nap\_get netcdf} 
  $filename name$ [
  $index$ [
  $raw$]]
  \\
  


  $filename$ is the:
  \begin{itemize}
    \item pathname of a local file
    \item URL of a remote 
    \href{http://www.opendap.org/}{OPeNDAP (a.k.a. DODS)}
    server
  \end{itemize}
  


  $name$ is the name of a variable or attribute and has the
  form:
  \begin{itemize}
    \item 
    $varname$ for a variable
    \item 
    $varname$
    \texttt{:}
    $attribute$ for an attribute of a variable
    \item 
    \texttt{:}
    $attribute$ for a global attribute
    \\
  \end{itemize}
  

A single-element attribute gives a scalar. Other attributes give
  vectors. Neither 
  $index$ nor 
  $raw$ can be specified for attributes.
  

A variable with no 
  $index$ gives the entire variable. If 
  $index$ is specified then it selects using 
  \href{indexing.html\#Cross-product-index}{cross-product}
  indexing.
  

If 
  $raw$ is 1 then the result contains raw data read from the
  file. If 
  $raw$ is 0 (default) then this data is transformed using the
  attributes 
  \texttt{scale\_factor}, 
  \texttt{add\_offset}, 
  \texttt{\_FillValue}, 
  \texttt{valid\_min}, 
  \texttt{valid\_max} and 
  \texttt{valid\_range} if any of these are present.
  

The following example first creates a netCDF file using the netCDF
  utility 
  \texttt{ncgen}. There is one variable called 
  \texttt{vec}. It is a 3-element 32-bit integer vector with
  elements 6, -9 and 4. The data is read into a NAO called 
  \texttt{v} using 
  \texttt{nap\_get netcdf}.
  \begin{verbatim}
% exec ncgen -b << {
    netcdf int {
        dimensions:
            n = 3 ;
        variables:
            int vec(n) ;
        data:
            vec = 6, -9, 4 ;
    }
}
% nap "v = [nap_get netcdf int.nc vec]"
::NAP::52-52
% $v all
::NAP::52-52  i32  MissingValue: -2147483648  References: 1  Unit:
(NULL)
Dimension 0   Size: 3      Name: n         Coordinate-variable: (NULL)
Value:
6 -9 4
\end{verbatim}

  

Now let's specify the index 
  \texttt{\{0 2\}} to select the first and third elements:
  \begin{verbatim}
% [nap_get netcdf int.nc vec "{0 2}"] all
::NAP::58-58  i32  MissingValue: -2147483648  References: 0  Unit:
(NULL)
Dimension 0   Size: 2      Name: (NULL)    Coordinate-variable: (NULL)
Value:
6 4
\end{verbatim}

  

The following shows the different effects of a single-element
  vector index and a scalar index with the same value:
  \begin{verbatim}
% [nap_get netcdf int.nc vec "{1}"] all
::NAP::65-65  i32  MissingValue: -2147483648  References: 0
Dimension 0   Size: 1      Name: n         Coordinate-variable:
::NAP::69-69
Value:
-9
% [nap_get netcdf int.nc vec "1"] all
::NAP::83-83  i32  MissingValue: -2147483648  References: 0
Value:
-9
\end{verbatim}

  

The following is an 
  \href{http://www.opendap.org/}{OPeNDAP (a.k.a. DODS)} example.
  It displays the altitude of the south pole.
  \begin{verbatim}
% set url
http://www.marine.csiro.au/dods/nph-dods/dods-data/climatology-netcdf/etopo5.nc
% [nap_get netcdf $url height "@@-90, 0"]
2810
\end{verbatim}

  \section{
    \label{Reading-HDF-Data}Reading HDF Data
  }

  

HDF data is read using the command
  \\
  \texttt{nap\_get hdf} 
  $filename name$ [
  $index$ [
  $raw$]]
  \\
  


  $filename$ is the pathname of a local file.
  


  $name$ is the name of an SDS or attribute and has the
  form:
  \begin{itemize}
    \item 
    $sdsname$ for a SDS
    \item 
    $sdsname$
    \texttt{:}
    $attribute$ for an attribute of a SDS
    \item 
    \texttt{:}
    $attribute$ for a global attribute
    \\
  \end{itemize}
  

A single-element attribute gives a scalar. Other attributes give
  vectors. Neither 
  $index$ nor 
  $raw$ can be specified for attributes.
  

An SDS with no 
  $index$ gives the entire SDS. If 
  $index$ is specified then it selects using 
  \href{indexing.html\#Cross-product-index}{cross-product}
  indexing.
  

If 
  $raw$ is 1 then the result contains raw data read from the
  file. If 
  $raw$ is 0 (default) then this data is transformed using the
  attributes 
  \texttt{scale\_factor}, 
  \texttt{add\_offset}, 
  \texttt{\_FillValue}, 
  \texttt{valid\_min}, 
  \texttt{valid\_max} and 
  \texttt{valid\_range} if any of these are present.
  

The following example writes data to an HDF file using the OOC 
  \texttt{hdf} method. Then 
  \texttt{nap\_get hdf} is used with various index values (including
  default) to read the data back into temporary NAOs (which are deleted
  after being displayed):
  \begin{verbatim}
% [nap "f64{{1 0 9}{3 2 -1}}"] hdf mat.hdf mat64
% [nap_get hdf mat.hdf mat64] all; # default index giving whole SDS
::NAP::47-47  f64  MissingValue: NaN  References: 0
Dimension 0   Size: 2      Name: fakeDim0  Coordinate-variable: (NULL)
Dimension 1   Size: 3      Name: fakeDim1  Coordinate-variable: (NULL)
Value:
 1  0  9
 3  2 -1
% [nap_get hdf mat.hdf mat64 "1,0"] all; # select element
giving scalar
::NAP::78-78  f64  MissingValue: NaN  References: 0
Value:
3
% [nap_get hdf mat.hdf mat64 "{1},0"] all; # select element
giving vector
::NAP::102-102  f64  MissingValue: NaN  References: 0
Dimension 0   Size: 1      Name: fakeDim0  Coordinate-variable:
::NAP::92-92
Value:
3
% [nap_get hdf mat.hdf mat64 "0,"] all; # select row
::NAP::123-123  f64  MissingValue: NaN  References: 0
Dimension 0   Size: 3      Name: fakeDim1  Coordinate-variable: (NULL)
Value:
1 0 9
% [nap_get hdf mat.hdf mat64 ",2"] all; # select column
::NAP::147-147  f64  MissingValue: NaN  References: 0
Dimension 0   Size: 2      Name: fakeDim0  Coordinate-variable: (NULL)
Value:
9 -1
% [nap_get hdf mat.hdf mat64 ",{0 2}"] all; # select
sub-matrix
::NAP::154-154  f64  MissingValue: NaN  References: 0
Dimension 0   Size: 2      Name: fakeDim0  Coordinate-variable: (NULL)
Dimension 1   Size: 2      Name: fakeDim1  Coordinate-variable:
::NAP::161-161
Value:
 1  9
 3 -1
\end{verbatim}

  \section{
    \label{Listing-Names}Listing Names of Variables/SDSs and Attributes in HDF and netCDF Files
  }
One can list the names of variables/SDSs and attributes matching
  a regular expression 
  $RE$ using the command
  \\
  \texttt{nap\_get hdf -list} 
  $filename$ [
  $RE$]
  \\or
  \\
  \texttt{nap\_get netcdf -list} 
  $filename$ [
  $RE$]
  \\All variables/SDSs and attributes are listed if 
  $RE$ is omitted. For example, using the HDF file created
  above:
  \begin{verbatim}
% nap_get hdf -list mat.hdf
mat64
mat64:_FillValue
\end{verbatim}

  

Some useful regular expressions are

\begin{tabular}{|l|l|}
    \hline 
      \textbf{Regular Expression} & \textbf{Select all:}
    \\
    \hline 
        \texttt{\^[\^:]*\$} & variables
    \\
    \hline 
        \texttt{:} & attributes
    \\
    \hline 
        \texttt{\^:} & global attributes
    \\
    \hline 
        \texttt{.:} & non-global attributes
    \\
  \hline
\end{tabular}

  

Thus we can restrict the above list to SDSs only using:
  \begin{verbatim}
% nap_get hdf -list mat.hdf {^[^:]*$}
mat64
\end{verbatim}

  \section{
    \label{Metadata}Reading Metadata from HDF and netCDF Files
  }

  

The command
  \\
  \texttt{nap\_get hdf -datatype} 
  $filename$ $sdsname$
  \\or
  \\
  \texttt{nap\_get netcdf -datatype} 
  $filename$ $varname$
  \\returns the data-type of a specified variable/SDS in the
  specified file.
  

The command
  \\
  \texttt{nap\_get hdf -rank} 
  $filename$ $sdsname$
  \\or
  \\
  \texttt{nap\_get netcdf -rank} 
  $filename$ $varname$
  \\returns the rank (number of dimensions) of a specified
  variable/SDS in the specified file.
  

The command
  \\
  \texttt{nap\_get hdf -shape} 
  $filename$ $sdsname$
  \\or
  \\
  \texttt{nap\_get netcdf -shape} 
  $filename$ $varname$
  \\returns the shape (dimension sizes) of a specified variable/SDS
  in the specified file.
  

The command
  \\
  \texttt{nap\_get hdf -dimension} 
  $filename$ $sdsname$
  \\or
  \\
  \texttt{nap\_get netcdf -dimension} 
  $filename$ $varname$
  \\returns the dimension names of a specified variable/SDS in the
  specified file.
  

The command
  \\
  \texttt{nap\_get hdf -coordinate} $filename$ $sdsname$ $dim\_name|dim\_number$
  \\or
  \\
  \texttt{nap\_get netcdf -coordinate} $filename$ $varname$ $dim\_name|dim\_number$
  \\returns the coordinate variable NAO corresponding to the
  specified dimension of the specified variable/SDS in the specified
  file.
